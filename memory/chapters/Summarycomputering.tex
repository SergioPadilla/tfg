\section{Resumen}

En este trabajo estudiaremos teoría de conjuntos difusos para modelar matemáticamente información que utilizamos habitualmente para comunicarnos. A partir de ello, realizaremos una extensión de la funcionalidad de MongoDB para trabajar con información imprecisa.

Veremos una introducción a las bases de datos NoSQL, utilizadas frecuentemente para técnicas de Big Data, y su comparativa con los sistemas relacionales de bases de datos clásicos. Haremos una introducción a la teoría de conjuntos y las operaciones que se pueden realizar con ellos y veremos como incluir información imprecisa en la base de datos MongoDB.

Concluiremos con una propuesta software para extender los operadores de consulta y proyección de MongoDB.

\paragraph{Palabras clave} nosql, mongodb, conjuntos difusos, probabilidad, trapezoides, bases de datos, consulta difusa

\newpage
