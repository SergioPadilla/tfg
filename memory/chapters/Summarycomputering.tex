\section{Resumen}

En este trabajo estudiaremos teoría de conjuntos difusos para modelar matemáticamente información que utilizamos habitualmente para comunicarnos. A partir de ello, realizaremos una extensión de la funcionalidad de MongoDB, un sistema de bases de datos encuadrado en la categoría de Bases de Datos NoSQL, para trabajar con información imprecisa.

Veremos una introducción a las bases de datos NoSQL, utilizadas frecuentemente para técnicas de Big Data, y su comparativa con los sistemas relacionales de bases de datos clásicos. Haremos una introducción a la teoría de conjuntos y las operaciones que se pueden realizar con ellos y veremos cómo incluir información imprecisa en la base de datos MongoDB.

Concluiremos con una propuesta software para extender los operadores de consulta y proyección de MongoDB, para dotar a este sistema de bases de datos de capacidad para representar y consultar información de tipo impreciso.

\paragraph{Palabras clave} bases de datos, nosql, mongodb, conjuntos difusos, posibilidad, necesidad, número difusos trapezoidales, consulta difusa.

\newpage
