\section{Resumen}

En este trabajo se abordará el problema isoperimétrico en $\rtres$. Se demostrará que dado un volumen fijo en $\rtres$, la superficie que encierra dicho volumen con área mínima es la esfera $\unitsphere$ y que además, es la única. 

Se introducirá la teoría de integración en superficies en la que veremos que las propiedades que conocemos para las integrales en el sentido de Lebesgue se extienden para superficies. Haremos una introducción a los entornos tubulares y superficies paralelas para obtener las primeras variaciones del área y del volumen que necesitamos para ver que las superficies isoperimétricas tienen curvatura constante y se terminará el trabajo probando que las esferas son las únicas superficies isoperimétricas mediante la prueba de Wente.

\paragraph{Palabras clave} Geometría diferencial global, problema isoperimétrico, superficies, curvatura media, esferas, superficie isoperimétricas, desigualdad de Brunn-Minkowski, fórmulas de variación para el área y el volumen

\newpage
