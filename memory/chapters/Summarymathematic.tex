\section{Resumen}

En este trabajo se abordará el problema isoperimétrico en $\rtres$. Se demostrará que las esferas minimizan el área entre todas las superficies compactas y conexas que encierran un volumen dado. Probaremos además que las esferas son las únicas superficies que exhiben esta propiedad de minimización.

Se introducirá la teoría de integración en superficies en la que veremos cómo las propiedades que conocemos para las integrales en el sentido de Lebesgue se extienden para superficies. Haremos una introducción a los entornos tubulares y superficies paralelas para obtener las primeras variaciones del área y del volumen que necesitamos para probar que las superficies isoperimétricas tienen curvatura media constante, probaremos la desigualdad de Brunn-Minkowski, la desigualdad isoperimétrica y se terminará el trabajo probando que las esferas son las únicas superficies isoperimétricas mediante la prueba de Wente.

\paragraph{Palabras clave} Geometría diferencial global, problema isoperimétrico, superficies, curvatura media, esferas, superficie isoperimétricas, desigualdad de Brunn-Minkowski, fórmulas de variación para el área y el volumen.

\newpage
