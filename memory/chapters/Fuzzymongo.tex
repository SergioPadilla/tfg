\label{propuesta}
Una vez introducida la teoría de conjuntos difusos en el capítulo previo y de dar una introducción a las bases de datos no relaciones, vamos a tratar de dar una solución para la base de datos NoSQL, MongoDB, dotándola con algunas utilidades para poder trabajar con estos conjuntos de datos.

En la literatura podemos encontrar trabajos con propuestas para bases de datos NoSQL, véase \cite{fuzzyquerygraph, fuzzyquerygraph2, fuzzyquerygraph3}, donde se pueden encontrar propuestas para bases de datos basadas en grafos o \cite{fuzzyqueryhbase} donde se habla de modelado de conjuntos difusos en la base de datos HBase. En \cite{fuzzyquerymongo} podemos encontrar una propuesta para realizar consultas difusas en la base de datos MongoDB, el cuál hemos utilizado para sacar algunas ideas sobre las que se basan nuestra propuesta.

El objetivo de la propuesta es mantener la compatibilidad con el funcionamiento de MongoDB común, adaptar el lenguaje al tipo de consultas que se realiza de forma nativa y adaptar la representación de los datos difusos a los tipos de datos nativos utilizados por MongoDB.

\section{Fuzzy Find}

En \cite{tesismedina} se expuso un módulo para permitir extender la capacidad de un SGBDR clásico para que pueda representar y manipular información imprecisa. En este trabajo se ha realizado una prueba de concepto sobre MongoDB con la implementación de la función \textbf{fuzzy\_find}, que nos permitirá realizar consultas sobre una abase de datos que contenga datos de tipo difuso.

Para llevar a cabo esto, hemos utilizado la utilidad que nos provee MongoDB para \href{https://docs.mongodb.com/manual/tutorial/store-javascript-function-on-server/}{almacenar funciones javascript en el servidor de MongoDB}, de forma que se permite su uso en cualquier contexto javascript.

Comenzamos una prueba de concepto con el operador de agregación \texttt{map-reduce} descrito en \ref{mapreduce}, pero tras unas pruebas con una cantidad de datos relativamente grande, no obteníamos un el rendimiento que esperábamos, las consultas eran demasiado lentas y descartamos esta opción a favor de utilizar el operador de agregación \texttt{pipeline}, véase la sección \ref{pipeline}. Este operador nos permite utilizar índices y nos ofrece para este problema en concreto un rendimiento muy superior a la opción \textit{map-reduce}.

\subsection{Sintaxis}

La cabecera de la función \texttt{fuzzy\_find} es la siguiente:

\begin{lstlisting}[numbers=none]
fuzzy_find(collection, filter, projection, count_name=null)
\end{lstlisting}

donde los parámetros son:

\begin{itemize}
    \item collection: Nombre de la colección sobre la que se quiere ejecutar la consulta.
    \item filter: JSON con la query que se quiere realizar. Conserva el formato propuesto por MongoDB además de los operadores que introducimos en este trabajo, explicaremos estos enla sección posterior.
    \item projection: JSON para la etapa de proyección.
    \item count\_name: Campo opcional. Si se especifica una cadena de texto se añade la etapa de \textit{count} para devolver el número de documentos en lugar de los propios documentos.
\end{itemize}

Veamos la sintaxis propuesta mediante un ejemplo:

\begin{example}

Supongamos una base de datos \texttt{test}, con la colección \texttt{example} y la siguiente información:

\begin{lstlisting}
{ _id: 1, item: { name: "ab", code: "123" }, qty: 15, tags: [ "A", "B", "C" ] }
{ _id: 2, item: { name: "cd", code: "123" }, qty: 20, tags: [ "B" ] }
{ _id: 3, item: { name: "ij", code: "456" }, qty: 25, tags: [ "A", "B" ] }
{ _id: 4, item: { name: "xy", code: "456" }, qty: 30, tags: [ "B", "A" ] }
{ _id: 5, item: { name: "mn", code: "000" }, qty: 20, tags: [ [ "A", "B" ], "C" ] }
\end{lstlisting}

podemos ejecutar la consulta utilizando la función \texttt{fuzzy\_find} como sigue:

\begin{lstlisting}[numbers=none]
fuzzy_find('example', { qty: { \$eq: 20 } }, {})
\end{lstlisting}

y obtenemos:

\begin{lstlisting}
{ _id: 2, item: { name: "cd", code: "123" }, qty: 20, tags: [ "B" ] }
{ _id: 5, item: { name: "mn", code: "000" }, qty: 20, tags: [ [ "A", "B" ], "C" ] }
\end{lstlisting}

si añadimos una proyección a la función:

\begin{lstlisting}[numbers=none]
fuzzy_find('example', { qty: { \$eq: 20 } }, {tags: 1})
\end{lstlisting}

obtendríamos:

\begin{lstlisting}
{ "_id" : 2, "tags" : [ "B" ] }
{ "_id" : 5, "tags" : [ [ "A", "B" ], "C" ] }
\end{lstlisting}

\end{example}

\subsection{Representación de información}

Dependiendo del tipo de dato, acordamos su representación como sigue:

\begin{itemize}
    \item Datos precisos: Hacen referencia a los datos comunes de MongoDB, enteros, fechas, cade de texto, arrays, documentos... Estos datos se seguiran representando con el tipo de dato correspondiente y se trabajará con ellos de forma nativa.
    \item Datos difusos: Como ya adelantamos en los capítulos previos, la forma de trabajar con datos difusos que hemos utilizado es mediante los números difusos \ref{fuzzynumbers} de tipo \textbf{trapezoides} \ref{trapezoidrepresentation}. Para representar esto en MongoDB hemos utilizado \texttt{arrays} de 4 posiciones, por tanto, la etapa previa a cualquier operación con un operador difuso será transformar el valor que nos den a un trapezoide siguiendo lo descrito en \ref{notaciontrapezoide}.
\end{itemize}

