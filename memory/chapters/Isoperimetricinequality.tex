\section{Desigualdad Isoperimétrica}
Por último, en esta sección vamos a definir el resultado que íbamos buscando y  más importante, la desigualdad isoperimétrica en $\rtres$. Para su demostración, necesitamos de algunos elementos que aún no hemos definido. Vamos a verlos como preliminares a la desigualdad.

Vamos a ver las fórmulas de Minkowski que nos serán de utilidad en lo sucesivo.
\begin{theorem}[Fórmulas de Minkowski]
Sea $S$ una superficie compacta, $N$ el dominio interior dado por su aplicación de Gauss y sean $H$ y $K$ las curvaturas media y de Gauss de la superficie. Tenemos:

\begin{enumerate}
    \item $\int_S (1+<p, N(p)>H(p))dp = 0$
    \item $\int_S (H(p)+<p, N(p)>K(p))dp = 0$
\end{enumerate}
\end{theorem}
