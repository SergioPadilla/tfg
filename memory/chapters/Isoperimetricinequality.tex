En esta sección vamos a definir el resultado que íbamos buscando y  más importante, la desigualdad isoperimétrica en $\rtres$. Para su demostración, necesitamos de algunos elementos que aún no hemos definido. Vamos a verlos como preliminares a la desigualdad.

Vamos a ver las fórmulas de Minkowski que nos serán de utilidad en lo sucesivo.
\begin{theorem}[Fórmulas de Minkowski]
Sea $S$ una superficie compacta, $N$ el dominio interior dado por su aplicación de Gauss y sean $H$ y $K$ las curvaturas media y de Gauss de la superficie. Tenemos:

\begin{enumerate}
    \item $\int_S (1 + \langle p, N(p) \rangle H(p))dp = 0$
    \item $\int_S (H(p)+ \langle p, N(p) \rangle K(p))dp = 0$
\end{enumerate}
\end{theorem}

\begin{definition}
Si $A, B$ son dos conjuntos de $\mathbb{R}^n$, definimos la \textbf{suma conjuntista} de $A+B$ como:

\begin{equation*}
    A+B = \{a+b; a \in A, b \in B\}
\end{equation*}
\end{definition}

\begin{lemma}
Sea $A,B \subset \mathbb{R}^n$. Si uno de los dos conjuntos es abierto, entonces la suma es un abierto.
\end{lemma}
\begin{proof}
Supongamos que $A$ es abierto.
Podemos notas la suma de los conjuntos como:
\begin{equation*}
    A + B = \{a+b; a \in A, b \in B\} = \bigcup_{b \in B} \{a+b; a \in A\}
\end{equation*}

Sabemos que la unión de abiertos es abierta, luego solo nos falta ver que $\{a+b; a \in A\}$ es abierto.

En efecto, tenemos que el conjunto $\{a+b; a \in A\}$ es una traslación del conjunto $A$, que es un movimiento rígido y por tanto conserva las propiedades del conjunto $A$. Concluimos que $A+B$ es un abierto.

Análogo para $B$ abierto.
\end{proof}


\begin{lemma}
Sea $A,B \subseteq \mathbb{R}^n$. Si ambos conjuntos están acotados, entonces la suma está acotada.
\end{lemma}
\begin{proof}
La demostración de esta propiedad es consecuencia directa de la desigualdad triangular.

Sea $L,K > 0$ las cotas de $A$ y $B$ respectivamente. Luego,
\begin{equation*}
    |a+b| \leq |a|+|b| \leq L + K \qquad \forall a \in A, b\in B
\end{equation*}

Luego $A+B$ está acotado.
\end{proof}


\begin{lemma}
Sea $A,B \subset \mathbb{R}^3$. Si ambos conjuntos son arco-conexos, entonces la suma es arco-conexo.
\end{lemma}
\begin{proof}
Veamos que $\forall c,d \in A+B$, $\exists \sigma: [0,1] \longrightarrow A+B$ arco, tal que $\sigma(0)=c$ y $\sigma(1)=d$.

Sea $c=a_1 + b_1$, $d=a_2+b_2$ cualesquiera pertenecientes a $A+B$. Por ser $A$ y $B$ arco-conexos, tenemos que existen $\sigma_1: [0,1] \longrightarrow A$ con $\sigma_1(0)=a_1$ y $\sigma_1(1)=a_2$ y $\sigma_2: [0,1] \longrightarrow B$ con $\sigma_2(0)=b_1$ y $\sigma_2(1)=b_2$. Luego podemos construir, $\sigma: [0,1] \longrightarrow A+B$ tal que $\sigma(0)=\sigma_1(0) + \sigma_2(0) = a_1+b_1 = c$ y $\sigma(1)=\sigma_1(1) + \sigma_2(1) = a_2+b_2 = d$ arco.
\end{proof}

\begin{lemma}\label{paralelepipedoslemma}
Sean $I_1, I_2, I_3$ y $J_1, J_2, J_3$ intervalos abierto de $\rmath$. Los subconjuntos de $\rtres$ dados por:

\begin{equation*}
    A = I_1 \times I_2 \times I_3, \qquad B = J_1 \times J_2 \times J_3
\end{equation*}

son llamados \textbf{paralelepípedos con los ejes paralelos a los ejes coordenados}. Entonces tenemos:

\begin{equation*}
    A + B = (I_1 + J_1) \times (I_2 + J_2) \times (I_3 + J_3)
\end{equation*}

Además, si A y B son disjuntos, tenemos que existe un plano paralelos a uno de los ejes coordenado que separando $A$ y $B$.
\end{lemma}

Vamos a hacer dos observaciones para ver el área de una esfera y el volumen de una bola, que nos serán de utilidad en lo sucesivo.
\begin{remark}[Área de la esfera]
Veamos cuál es el área de una esfera, $\mathbb{S}^2(p_0, r)$. Sabemos que:

\begin{equation*}
    A(\mathbb{S}^2(p_0, r)) = \int_{\mathbb{S}^2(p_0, r)} 1dx
\end{equation*}

Consideremos la siguiente traslación: $T: \mathbb{S}^2(0, r) \longrightarrow \mathbb{S}^2(p_0, r)$ tal que $T(p) = p + p_0$, usando las fórmulas del cambio de variable vista en el capítulo anterior, tenemos que:

\begin{equation*}
    \int_{\mathbb{S}^2(p_0, r)} 1dx = \int_{\mathbb{S}^2(0, r)} |Jac T|dx
\end{equation*}

Además, $|JacT|=1$, luego:

\begin{equation*}
    \int_{\mathbb{S}^2(0, r)} |Jac T|dx = \int_{\mathbb{S}^2(0, r)} 1dx = A(\mathbb{S}^2(0, r))
\end{equation*}

Para calcular esta integral, consideramos la parametrización de la esfera centrada en 0 como una superficie de revolución:

\begin{align*}
    X: (\frac{-\pi}{2}, \frac{\pi}{2}) \times (-\pi, \pi) &\longrightarrow \mathbb{S}^2(0, r) - C \\
    (t, \theta) &\longrightarrow (r\cos t\cos\theta, r\cos t\sen\theta, r\sen t)
\end{align*}

con $C = \{X(t,\pi): t\in [-\pi/2, \pi/2]\}$. Por tanto, $\mathbb{S}^2(0, r) - C = X((-\pi/2, \pi/2) \times (-\pi, \pi))$, como $C$ es un conjunto de medida nula, tenemos que:

\begin{equation*}
    \int_{\mathbb{S}^2(0, r)} 1da = \int_{\mathbb{S}^2(0, r) - C} 1da
\end{equation*}

Utilizando de nuevo el cambio de variable:

\begin{equation*}
    \int_{\mathbb{S}^2(0, r) - C} 1da = \int_{(\frac{-\pi}{2}, \frac{\pi}{2}) \times (-\pi, \pi)} |JacX| dtd\theta
\end{equation*}

Calculemos el jacobiano de X, sea $(dX)_{(t, \theta)}: \rdos \longrightarrow T_{X(t,\theta)} \mathbb{S}^2$, con:

\begin{align*}
    (dX)_{(t, \theta)}(1,0) &= X_t(t, \theta) \\
    (dX)_{(t, \theta)}(0,1) &= X_\theta(t, \theta) \\
    X_t(t, \theta) &= (r\sen t\cos\theta, -r\sen t\sen\theta, r\cos t) \\
    X_\theta(t, \theta) &= (-r\cos t\sen\theta, r\cos t\cos\theta, 0) \\
    |X_t|^2 &= r^2 \\
    |X_\theta|^2 &= r^2\cos^2t
\end{align*}

Luego $\{ \frac{X_t}{r}, \frac{X_\theta}{r\cos t} \}$. Por definición:

\begin{equation*}
    |Jac X|(t,\theta) = r^2\cos t
\end{equation*}

y por tanto:

\begin{equation*}
    A(\mathbb{S}^2(0,r)) = \int_{(\frac{-\pi}{2}, \frac{\pi}{2}) \times (-\pi, \pi)} r^2\cos t da = 4\pi r^2
\end{equation*}

Finalmente:
\begin{equation*}
    A(\mathbb{S}^2(p_0,r)) = 4\pi r^2
\end{equation*}
\end{remark}

\begin{remark}[Volumen de una bola]
Partiendo como en el cálculo del área de la esfera, consideramos la traslación, $T: B(0, r) \longrightarrow B(p_0, r)$ tal que $T(p) = p + p_0$, aplicando la fórmula del cambio de variable y sabiendo que $|JacT|=1$, tenemos:

\begin{equation*}
    vol B(p_0,r) = \int_{B(p_0,r)} 1 dx = \int_{B(0,r)} |JacT|dx = \int_{B(0,r)} 1dx = vol B(0,r)
\end{equation*}

Utilizando el teorema de la divergencia con $S=\mathbb{S}^2(0,r)$ y $\Omega=B(0,r)$, tomando $X(p)=p$ tenemos que $(div X)(p)=3$, con $t\in \rtres$ y nos queda:

\begin{align*}
    vol B(0,r) &= \frac{-1}{3} \int_{\mathbb{S}^2(0,r)}  \langle p, \frac{-p}{r} \rangle da \\
    vol B(0,r) &= \frac{-1}{3r} \int_{\mathbb{S}^2(0,r)}  \langle p, -p \rangle da \\
    vol B(0,r) &= \frac{r}{3} \int_{\mathbb{S}^2(0,r)} 1da
\end{align*}

ya que $ \langle p,p \rangle  = |p|^2 = r^2$ con $p \in \mathbb{S}^2(0,r)$.

Por tanto tenemos que el $vol B(0,r) = \frac{r}{3}A(\mathbb{S}^2(0,r))$ que hemos calculado previamente:

\begin{equation*}
    vol B(p,r) = \frac{4}{3}\pi r^3
\end{equation*}
\end{remark}

\begin{remark}[Volumen encerrado por una superficie paralela]
Sea $S$ una superficie compacta y $\epsilon  \rangle  0$ tal que $N_\epsilon(S)$ sea un entorno tubular. Hemos notado a las superficies paralelas como $\{S_t\}_{t \in (-\epsilon, \epsilon)}$. Vamos a calcular el volumen comprendido entre $S$ y $S_t$ con $t\in (0, \epsilon)$. Sea $F: S \times (0, \epsilon) \longrightarrow V_t(S)$ definida como $F(p, t) = p + tN(p)$, donde $V_t(S)$ es el volumen que buscamos. Sean $\Omega$ y $\Omega_t$ los dominios interiores determinados por $S$ y $S_t$, respectivamente. Teniendo en cuenta que $F$ es un difeomorfismo, tenemos:

\begin{equation*}
    vol \Omega - vol \Omega_t = vol (V_t(S)) = \int_{V_t(S)} 1dx = \int_{S \times (0,t)} |JacF|(p,t)dSdt
\end{equation*}

Calculemos $|Jac F|$. Sean ${e_1, e_2}$ una base ortonormal de direcciones principales en $p \in S$, sabemos que:

\begin{align*}
    (dF)_{(p,t)}(e_i,0) &= (1-tk_i(p))e_i, \qquad i = 1,2 \\
    (dF)_{(p,t)}(0,1) &= N(p)
\end{align*}

Tomamos ahora $\{(e_1,0), (e_2,0), (0,1)\}$ base ortonormal de $T_pS \times \rmath$ y $\{e_1, e_2, N(p)\}$ como base ortonormal de $\rtres$, y obtenemos:

\begin{align*}
    |Jac F|(p,t) &= \left|
  det \left( {\begin{array}{ccc}
   1 - tk_1(p) & 0 & 0 \\
   0 & 1-tk_2(p) & 0 \\
   0 & 0 & 1 \\
  \end{array} } \right) \right| \\
  |Jac F|(p,t) &= 1 - 2tH(p) + t^2K(p)
\end{align*}

Y por último, utilizando el teorema de Fubini:

\begin{align*}
    vol \Omega - vol \Omega_t &= \int_0^t \left( \int_{S} 1-2tH(p)+t^2K(p) dp \right) dt \\
    vol \Omega - vol \Omega_t &= tA(S) - t^2\int_S HdS + \frac{t^3}{3}\int_S KdS
\end{align*}
\end{remark}

\begin{remark}[Área de una superficie paralela]
Sea $S$ una superficie compacta y $\epsilon > 0$ tal que $N_\epsilon(S)$ sea un entorno tubular. Recordamos que hemos notado a las superficies paralelas como $\{S_t\}_{t \in (-\epsilon, \epsilon)}$.

Sabemos que la función $F_t: S \longrightarrow S_t$ es un difeomorfismo, definido como $F_t(p, t) = p + tN(p)$, luego utilizando el cambio de variable, tenemos:

\begin{equation*}
    A(S_t) = \int_{S_t} 1dS_t = \int_S |Jac F_t|(p,t)dS
\end{equation*}

Hemos visto previamente:

\begin{align*}
    |Jac F_t|(p,t) &= \left|
  det \left( {\begin{array}{cc}
   1 - tk_1(p) & 0 \\
   0 & 1-tk_2(p) \\
  \end{array} } \right) \right| \\
  |Jac F_t|(p,t) &= 1-2tH(p) + t^2K(p)
\end{align*}

Por tanto, tenemos que:

\begin{equation*}
    A(S_t) = A(S) -2t\int_{S} H(p)dS + t^2\int_{S} K(p)dS
\end{equation*}
\end{remark}

Con estos preliminares, vamos a demostrar la desigualdad de Brunn-Minkowski basada en la del libro \cite{montielrosbook}, que a su vez será pieza fundamental de la demostración del resultado más importarte de este capítulo, la desigualdad isoperimétrica.
\begin{theorem}[Desigualdad de Brunn-Minkowski]
Sean $A, B$ dos abiertos acotados del espacio euclídeo $\rtres$. Se cumple:

\begin{equation*}
    (vol A)^{\frac{1}{3}} + (vol B)^{\frac{1}{3}} \leq (vol (A+B))^{\frac{1}{3}}
\end{equation*}
\end{theorem}
\begin{proof}
Esta demostración la vamos a hacer en tres pasos. Vamos a probar la desigualdad cuando $A,B$ son paralelepípedos en primer lugar, cuando son unión finita de paralelepípedos abiertos disjuntos acotados cuyos lados son paralelos a los ejes coordenados y por último el caso de dos abiertos acotados cualesquiera.

Supongamos entonces el caso de paralelepípedos, sea $A = I_1 \times I_2 \times I_3$ y $B = J_1 \times J_2 \times J_3$ donde $I_i,J_i \quad \forall i=1,2,3$ son intervalos abiertos y acotados de $\rmath$. Luego:

\begin{equation*}
    \frac{ \left(vol A \right)^{1/3} + \left(vol B \right)^{1/3}}{\left(vol (A+B) \right)^{1/3}} = \frac{\left(\displaystyle\prod_{i=1}^3 a_i \right)^{1/3} + \left(\displaystyle\prod_{i=1}^3 b_i \right)^{1/3}}{\left(\displaystyle\prod_{i=1}^3 a_i+b_i \right)^{1/3}}
\end{equation*}

y por tanto:

\begin{equation*}
    \frac{ \left(vol A \right)^{1/3} + \left(vol B \right)^{1/3}}{\left(vol (A+B) \right)^{1/3}} = \left(\displaystyle\prod_{i=1}^3 \frac{a_i}{a_i+b_i} \right)^{1/3} + \left(\displaystyle\prod_{i=1}^3 \frac{b_i}{a_i+b_i} \right)^{1/3}
\end{equation*}

Además, por la desigualdad de las medias, sabemos que:

\begin{equation*}
    \frac{ \left(vol A \right)^{1/3} + \left(vol B \right)^{1/3}}{\left(vol (A+B) \right)^{1/3}} \leq
    \frac{1}{3} \displaystyle\sum_{i=1}^3 \frac{a_i}{a_i+b_i} + \frac{1}{3} \displaystyle\sum_{i=1}^3 \frac{b_i}{a_i+b_i} = 1
\end{equation*}

Para este caso, la desigualdad queda probada.

Supongamos ahora el segundo caso, sean $A = \displaystyle\bigcup_{i=1}^n A_i$ y $B = \displaystyle\bigcup_{j=1}^n B_j$, donde $A_i$, $B_j$ son paralelepípedos iguales a los del primer caso.

Vamos a probar la desigualdad por inducción sobre el número total de paralelepípedos, $t = n + m$.
El caso $t=2$ es claro, pues es el caso previamente demostrado.
Supongamos cierta la desigualdad cuando $t < n + m$ y veamos que es cierta para $t \geq n + m$. Como ya hemos visto que $t=2$ es cierto, podemos suponer que $n + m \geq 3$ luego es claro que $n > 1$ or $m > 1$.

Supongamos el primero de ellos, por el lema \autoref{paralelepipedoslemma} tenemos que existe un plano, $P$, que separando $A_1$ y $A_2$. Sea $P^+$ y $P^-$ los dos espacios en los que $P$ divide a $\rtres$ y sean $A^+ = A \cap P^+$ y $A^- = A \cap P^-$, además podemos notar a estos como:

\begin{equation}\label{equationuniones}
    A^+ = \displaystyle\bigcup_{i=1}^{n^+} A_i^+ \qquad A^- = \displaystyle\bigcup_{i=1}^{n^-} A_i^-
\end{equation}

una unión finita de de paralelepípedos cuyos ejes son paralelos a los ejes coordenados, donde $n^+ < n$ y $n^- < n$ ya que $P$ divide a $A_1$ y $A_2$.

Cogemos ahora un plano $Q$ que cumpla:

\begin{equation}\label{equationvolumen}
    \frac{vol A^+}{vol A} = \frac{vol B^+}{vol B}
\end{equation}

esto es posible ya que $\frac{vol A^+}{vol A} = c$ con $0 < c < 1$. Luego podemos tomar $Q$ a la izquierda de todos los paralelepípedos $B$ y desplazarlo a la derecha hasta encontrar la igualdad, gracias a que $\frac{vol B^+}{vol B}$ es una función continua.
Dado que $vol A = vol A^+ + vol A^-$ tenemos la misma \autoref{equationvolumen} para $A^-$ y $B^-$.

Además, $B = B^+ + B^-$ cumplen también la \autoref{equationuniones} para $B^+$ y $B^-$ con $m^+ \leq m$ y $m^- \leq m$, en este caso, no podemos suponer que son estrictos debido a que no tenemos garantizado el hecho de que $Q$ separe a dos paralelepípedos de $B$.

Estamos en condiciones de aplicar la hipótesis de inducción a los subconjuntos $A^+$, $A^-$ y $B^+$, $B^-$ ya que el total de paralelepípedos cumple la condición $n^+ + m^+ < n + m$ y $n^- + m^- < n + m$. Por tanto tenemos:

\begin{equation*}
    vol (A^+ + B^+) \geq \left[ (vol A^+)^{1/3} + (vol B^+)^{1/3} \right]^3
\end{equation*}
\begin{equation*}
    vol (A^- + B^-) \geq \left[ (vol A^-)^{1/3} + (vol B^-)^{1/3} \right]^3
\end{equation*}

Seguimos la deducción sabiendo que $A^+ \subset P^+$ y $B^+ \subset Q^+$, luego $A^+ + B^+ \subset P^+ + Q^+ = (P+Q)^+$ y análogamente $A^- + B^- \subset (P+Q)^-$. Teniendo en cuenta que $P+Q$ es otro plano de $\rtres$ tenemos que $A^+ + B^+$ y $A^- + B^-$ son disjuntos. En consecuencia:

\begin{align*}
    vol (A+B) &\geq vol(A^+ + B^+) + vol(A^- + B^-) \\
    vol (A+B) &\geq \left[ (vol A^+)^{1/3} + (vol B^+)^{1/3} \right]^3 + \left[ (vol A^-)^{1/3} + (vol B^-)^{1/3} \right]^3
\end{align*}

Y teniendo en cuenta la \autoref{equationvolumen}:

\begin{align*}
    vol (A+B) &\geq vol A^+ \left[ 1 + \left( \frac{vol B}{vol A} \right)^{1/3}  \right]^3 + vol A^- \left[ 1 + \left( \frac{vol B}{vol A} \right)^{1/3}  \right]^3 \\
    vol (A+B) &\geq vol A \left[ 1 + \left( \frac{vol B}{vol A} \right)^{1/3}  \right]^3 = \left[ (vol A)^{1/3} + (vol B)^{1/3} \right]^3
\end{align*}

Luego la desigualdad se cumple para este segundo caso.

Consideremos finalmente el tercer caso, con $A$ y $B$ dos abiertos acotados cualesquiera de $\rtres$. Usando la teoría de integración de Lebesgue, existen dos sucesiones de conjuntos abiertos de $\rtres$ como los del caso anterior, notémoslas como $A_n$ y $B_n$, con $n \in \mathbb{N}$, $A_n \subset A$, $B_n \subset B$ y:

\begin{equation*}
    \lim_{x \to \infty} vol A_n = vol A \qquad \lim_{x \to \infty} vol B_n = vol B
\end{equation*}

Luego tenemos que $A_n + B_N \subset A + B$ y para cada $n \in \mathbb{N}$:

\begin{equation*}
    vol (A+B)^{1/3} \geq vol (A_n+B_n)^{1/3} \geq vol (A_n) ^{1/3} + vol(B_n)^{1/3}
\end{equation*}

Basta con tomar límite con $n$ tendiendo a infinito y tenemos probada la desigualdad para todos los casos.
\end{proof}

Con la desigualdad de Brunn-Minkowski demostrada, vamos a probar el resultado más importante de este capítulo y sobre el que trabajaremos en el capítulo siguiente.
\begin{theorem}[Desigualdad Isoperimétrica]
Sea $S$ una superficie conexa y compacta con dominio interior $\Omega$. Se cumple:

\begin{equation*}
    A(S)^3 \geq 36\pi(vol \Omega)^2
\end{equation*}
\end{theorem}
\begin{proof}
Para demostrarlo, tenemos que buscar primero cierta situación donde podemos aplicar la desigualdad de Brunn-Minkowski y demostrar la desigualdad isoperimétrica.

Tomamos $\epsilon > 0$, tal que $N_\epsilon(S)$ es un entorno tubular. Para cada $t \in (0, \epsilon)$, tenemos que $S_t$ es una superficie paralela incluida en $\Omega$ y $\Omega_t$ es el dominio interior determinado por $S_t$, y además, vamos a tomar $B_t$ como la bola abierta centrada en el origen de $\rtres$ y radio $t$.

Consideremos el conjunto $S_t + B_t$, $\forall t \in (0, \epsilon)$. Veamos primero que los puntos de este conjunto no están en la superficie. Sea $p \in (S_t + B_t) \cap S$. Por definición, $p = q + v$ con $q \in S_t$ y $v \in B_t$, luego se cumple que $|p - q| = |v| < t$, luego $dist(q, S) < t$ lo que es una contradicción ya que el mínimo de la distancia de $q$ a $S$ se alcanza en un punto $p_0$ tal que $q$ está en la recta normal de $p_0$ y $S$, y teniendo en cuenta que $q \in S_t$, tenemos que $|p_0 - q| \geq t$.

Por tanto, concluimos que $\nexists p \in (S_t + B_t) \cap S$. Por otro lado, sabemos que $S_t$ y $B_t$ son arco-conexos, y hemos probado que la suma de arco-conexos es arco-conexa, por lo que $(S_t + B_t)$ es arco-conexo y por tanto tiene que estar incluido en una de las componentes arco-conexas de $\rmath - S$. Teniendo en cuenta que $\Omega_t \subset \Omega$, tenemos que $\Omega_t + B_t \subset \Omega$, $\forall t \in (0, \epsilon)$. Y tenemos la situación que buscábamos.

%NOTE: (?) con la inclusión suficiente para afirmar
Aplicando ahora la ya conocida, desigualdad de Brunn-Minkowski, tenemos que:

\begin{equation*}
    vol \Omega \geq vol (\Omega + B_t) \geq \left[ (vol \Omega)^{1/3} + vol (B_t)^{1/3} \right]^3
\end{equation*}

$\forall t \in (0, \epsilon)$. Desarrollamos la parte de la derecha y obtenemos:

\begin{equation*}
    vol \Omega \geq vol \Omega_t + 3(vol \Omega_t)^{2/3}(vol B_t)^{1/3} + 3(vol \Omega_t)^{1/3}(vol B_t)^{2/3} + vol B_t
\end{equation*}

y simplificando:

\begin{equation*}
    vol \Omega \geq vol \Omega_t + 3(vol \Omega_t)^{2/3}(vol B_t)^{1/3}
\end{equation*}

Sustituimos ahora el valor del volumen de la bola que hemos visto previamente:

\begin{align*}
    vol \Omega &\geq vol \Omega_t + 3(vol \Omega_t)^{2/3} \left( \frac{4\pi}{3}t^3 \right)^{1/3} \\
    vol \Omega &\geq vol \Omega_t + 3t \left( \frac{4\pi}{3} \right)^{1/3} (vol \Omega_t)^{2/3}
\end{align*}

y obtenemos:

\begin{equation*}
    \frac{vol \Omega - vol \Omega_t}{t} \geq 3 \left( \frac{4\pi}{3} \right)^{1/3} (vol \Omega_t)^{2/3}
\end{equation*}

Conociendo ahora el volumen encerrado entre superficies paralelas, nos queda:

\begin{align*}
    \frac{tA(S) - t^2\int_S H + \frac{t^3}{3}\int_S K}{t} &\geq 3 \left( \frac{4\pi}{3} \right)^{1/3} (vol \Omega_t)^{2/3} \\
    A(S) - t\int_S H + \frac{t^2}{3}\int_S K &\geq 3 \left( \frac{4\pi}{3} \right)^{1/3} (vol \Omega_t)^{2/3}
\end{align*}

Tomando límites en ambos lados con $t$ tendiendo a 0, tenemos:

\begin{equation*}
    A(S) \geq 3 \left( \frac{4\pi}{3} \right)^{1/3} (vol \Omega)^{2/3}
\end{equation*}

y elevando al cubo tenemos la desigualdad que buscamos:

\begin{equation*}
    A(S)^3 \geq 36\pi(vol \Omega)^2
\end{equation*}
\end{proof}

\begin{proposition}
Las esferas son superficies isoperimétricas que minimizan el área entre todas las superficies compactas y conexas con su mismo volumen.
\end{proposition}
\begin{proof}
Veamos en primer lugar que las esferas son una superficie isoperimétrica con área mínima, para ello, basta ver que las esferas cumplen la igualdad isoperimétrica.

Sea $S=\mathbb{S}^2$ superficie compacta y conexa y $\Omega = B$, bola abierta, el dominio interior determinado por $S$. Sabemos que $A(S)=4\pi r^2$ y $vol \Omega = \frac{4\pi r^3}{3}$, luego sustituyendo en la desigualdad isoperimétrica, tenemos:

\begin{align*}
    64 \pi^3 r^6 &\geq 36\pi \frac{16\pi^2r^6}{9} \\
    64 \pi^3 r^6 &\geq 64\pi^3r^6 \\
    64 \pi^3 r^6 &= 64\pi^3r^6
\end{align*}

Veamos ahora que es la única. Supongamos que existe otra $S'$ tal que $vol \Omega' = vol \Omega$, donde $\Omega'$ es el dominio interior determinado por la superficie $S'$, aplicando la desigualdad isoperimétrica:

\begin{equation*}
    A(S')^3 \geq 36\pi (vol \Omega')^2 = 36\pi (vol \Omega)^2 = A(S)^3
\end{equation*}

Luego las esferas son las superficies isoperimétricas que minimizan el área.
\end{proof}

\begin{corolario}
Sea $S \subset \rtres$ una superficie compacta y conexa y $\Omega$ su dominio interior. Entonces $S$ es isoperimétrica si y solo si $A(S)^3 = 36\pi (vol \Omega)^2$.
\end{corolario}
