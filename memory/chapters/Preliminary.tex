En este capítulo haremos un repaso de los conceptos más importantes que nos serán de utilidad a lo largo del trabajo. Recordaremos definiciones y resultados de superficies regular, enunciaremos y demostraremos el teorema de Brower-Samelson e introduciremos los entornos tubulares y superficies paralelas.

\section{Superficie regular}

\begin{definition}
Sea un subconjunto $S \subseteq \rtres$, $S \neq \emptyset$, es una \textit{superficie regular} si:

$\forall p \in S$ $\exists V$ entorno abierto de $p$ en $S$ (con la topología inducida de $\rtres$) y una aplicación $\xmath: \utortres$ con $\umath \subseteq \rdos$ abierto, verificando:

\begin{enumerate}
    \item $\xmath \in \cinf(\umath, \rtres)$.
    \item $\dxq: \rdostortres$ es inyectiva $\forall q \in \umath$.
    \item $\xmath(\umath) = \vmath$ y $\xmath:\utov$ es un homeomorfismo.
\end{enumerate}
\end{definition}

Sea $S \subseteq \rtres$ una superficie. Un \textit{campo de vectores} (diferenciable) en $S$ es una aplicación diferenciable $\vmath: S \longrightarrow \rtres$. Si $\vmath_p := \vmath(p) \in \tmath_p S$ para todo $p \in S$, diremos que $\vmath$ es un \textit{campo tangente} a $S$. Si $\vmath_p \bot \tmath_p S$ para todo $p \in S$, diremos que $\vmath$ es un \textit{campo normal} a $S$. Un campo unitario es aquel que cumple $\parallel \vmath_p \parallel = 1$ para todo $p \in S$.

Sea $S \subseteq \rtres$ una superficie orientable, y $p \in S$. Se definen:

Se dice que $S$ es una \textbf{superficie orientable} si admite un campo normal unitario global $\nmath: S \longrightarrow \unitsphere$. A este campo $\nmath$ se le llama \textbf{aplicación de Gauss}.

Se llama \textbf{endomorfismo de Weingarten} de $S$ en $p$ al endomorfismo: $A_p = -(dN)_p$

\begin{definition}[Formas fundamentales]
La \textbf{primera forma fundamental} de $S$ en $p$ es: $I_p: T_pS \times T_pS \longrightarrow \rmath$ donde $I_p(u,v) =  \langle u,v \rangle $, $\forall u,v \in T_pS$.

La \textbf{segunda forma fundamental} de $S$ en $p$ es la forma bilineal: $II_p = \sigma_p: T_pS \times T_pS \longrightarrow \rmath$ donde $\sigma_p(u,v) =  \langle A_pu,v \rangle  = I_p(Ap_u,v) = - \langle (dN)_p(u), v \rangle $, $\forall u,v \in T_pS$.
\end{definition}

Notemos que $\sigma_p$ es simétrica y por tanto el endomorfismo de Weingarten ($A_p$) es autoadjunto.

Cómo $A_p$ es autoadjunto, entonces es diagonalizable mediante una base ortonormal y por tanto tiene valores propios reales, es decir, $\exists k_1(p), k_2(p) \in \rmath, k_1(p) \leq k_2(p)$, $\exists {e_1,e_2}$ base ortonormal en $(T_pS, I_p)$ de forma que $A_p(e_i) = k_i(p)e_i$, $\forall i = 1,2$

\begin{definition}[Curvaturas principales]
Los números $k_i(p)$ se llaman \textit{curvaturas principales} de $S$ en $p$.
\end{definition}
Los vectores propios, no nulos, de $A_p$ se llaman \textbf{direcciones principales} de $S$ en $p$.
Se define la \textbf{curvatura de Gauss} de $S$ en $p$ como el número real $K(p)=det A_p=k_1(p)k_2(p)$
Se define la \textbf{curvatura media} de $S$ en $p$ como el número real $H(p)=\frac{1}{2}tr A_p=\frac{k_1(p)+k_2(p)}{2}$
Se dice que $S$ es \textbf{una superficie llana} si $K(p)=0$, $\forall p \in S$

\begin{definition}[Superficie minimal y totalmente umbilical]
Se dice que $S$ es \textit{minimal} si $H(p)=0$, $\forall p \in S$

Un punto es \textit{umbilical} si $k_1(p)=k_2(p)$.

Sea $S \subseteq \rtres$, se dice que $S$ es \textit{totalmente umbilical} si todos son puntos son umbilicales.
\end{definition}

\begin{theorem}[Clasificación de las superficies totalmente umbilicales]\label{umbilicaltheorem}
Sea $S \subseteq \rtres$ superficie conexa, cerrada y totalmente umbilical. Entonces es un plano o una esfera.
\end{theorem}

La demostración de este teorema se basa en la que la conexión nos impide que salgan uniones de varios planos y esferas y el cierre impide que S sea un abierto de plano o esfera. No lo vamos a demostrar.

\begin{theorem}[Teorema de Hilbert-Liebmann]
Sea $S \subseteq \rtres$ una superficie compacta y conexa con curvatura de Gauss $K$ constante, entonces S es una esfera.
\end{theorem}


\begin{theorem}[Teorema de Alexandrov]
Sea $S \subseteq \rtres$ una superficie compacta y conexa con curvatura media $H$ constante, entonces S es una esfera.
\end{theorem}

Generalización del teorema de Hilbert-Liebmann a superficies cerradas.
\begin{theorem}[Teorema de Bonnet]
Sea $S \subseteq \rtres$ una superficie cerrada y conexa con curvatura de Gauss $K=c \rangle 0$, constante positiva, entonces S es una esfera.
\end{theorem}


\section{Teorema de Brower-Samelson}

En esta sección vamos a dar los preliminares necesarios para la demostración del Teorema de Brower-Samelson. Comenzaremos con el teore de Jordan-Brower, que nos permitirá hablar del volumen encerrado por una superficie compacta. Finalmente, entraremos en la definición y propiedades de los entornos tubulares.

Cómo ya sabemos por el Teorema de Jordan clásico en el caso de $\rdos$, una curva cerrada y simple, delimita el plano en dos superficies conexas, una de ellas acotada. Este teorema, nos va a permitir tener tener una extensión de este resultado para el caso de $\rtres$. De hecho, esta generalización, aunque no vamos a entrar a verla, es válida para todo $\rmath^n$ con un hiperplano suyo, puede verse en \cite{paperchicago}. Para este caso, veámoslo en n=3.

\begin{theorem}[Teorema de separación de Jordan-Brower]
Sea $S \subseteq \rtres$ superficie compacta y conexa. Entonces $\rtres - S$ tiene exactamente dos componentes conexas cuya frontera común es $S$.
\end{theorem}
\begin{proof}
Veamos, que si $S$ es compacta, luego cerrada, y conexa, entonces $\rtres - S$ tiene exactamente dos componentes conexas cuya frontera es común.

Sea $C$ una componente conexa de $\rtres - S$. Partiendo de la suposición que $S$ es cerrada, entonces $\rtres - S$ es localmente conexo, luego cada componente conexa $C$ es abierta. Además, la frontera de $C$ es distinta del vacío ($Fr(C) \neq \emptyset$). Veámoslo.

Supongamos $Fr(C) = \emptyset$, entonces $\bar{C} = int(C)\bigcup Fr(C)=int(C)=C$ luego, $C$ sería abierto y cerrado en $\rtres$, por tanto, $C=\rtres$. Lo que es una contradicción ya que definimos $C$ como una componente conexa de $\rtres - S$.

Notemos, $\rtres - S = C \bigcup C'$, donde $C'$ es la unión de todas las componentes conexas distintas de $C$. Entonces, $C'$ es un conjunto abierto de $\rtres$, $\rtres - C' = C \bigcup S$ es cerrado y por tanto $\bar{C} \subset C\bigcup S$. Entonces, $Fr(C)=\bar{C}-C \subseteq S$.


Veamos ahora que $Fr(C) = S$. Para ello, sabemos que $Fr(C)$ es un subconjunto cerrado no vacio de $S$, veamos que también es abierto en S.

Sea $p \in Fr(C) \subseteq S$, sea $W$ un entorno abierto de $p$ en $\rtres$, de forma que $W-S$ tiene exactamente dos componentes conexas, $C_1, C_2$ cuya frontera común es $S \bigcap W$. Sabemos que $W-S \subseteq \rtres - S$ luego, $C_1, C_2$ están contenidas en componentes conexas distintas de $\rtres - S$. Esto es, o $C_1, C_2$ están contenidas en $C$, o no lo cortan. Si ninguna componente conexa corta a $C$, tenemos que $W\bigcap C = \emptyset$ y por tanto, $p \notin \bar{C}$, luego $p \notin Fr(C)$ y contradice nuestra suposición. Así, tenemos que al menos una de las componentes $C_1, C_2$ están contenidas en $C$. Supongamos que $C_1 \subseteq C$, análogo para $C_2$. Tenemos: $W \bigcap S = Fr(C_1) \bigcap W \subseteq \bar{C_1} \subseteq \bar{C}$, luego $W \bigcap S= Fr(C)$. Con esto tenemos que $p$ es un punto interior de la frontera de $C$ y por tanto, la frontera de $C$ es un abierto en $S$, como queríamos probar.

Además hemos probado, que $C$ contiene una de las dos componentes conexas de $W-S$, y como estas componentes son disjuntas, no pueden existir más de dos, por tanto, queda probado que $C'$ es la otra componente conexa de $\rtres - S$.

\end{proof}

\begin{definition}[Dominios interior y exterior]
Llamamos \textit{dominio interior} y notamos como $\Omega$ a la componente conexa acotada limitada por la superficie $S$. Llamamos \textit{dominio exterior} a la componente no acotada $\rtres - \Omega=\Omega_{*}$.
\end{definition}

\begin{lemma}
Sea S una superficie y $p \in S$. $\exists W \subset \rtres$ entorno abierto y conexo de $p$ y $\exists G: W \longrightarrow B$ difeomorfismo que cumple $G(W\cap S) = B\cap P$ con $B$ una bola abierta y $P$ un plano, ambos de $\rtres$.
\end{lemma}

\begin{definition}
Sea $S$ superficie conexa y compacta y sea $v \in (T_pS)^{\perp}$ con $\parallel v \parallel=1$. Consideremos la recta afín $\alpha: \mathbb{R} \longrightarrow \rtres$ definida como $\alpha(t) = p + tv$ con $p \in S$. Entonces, $\alpha es regular$ y $\alpha(0)=p$ y $\alpha'(0) = v$. Existe entonces un $\epsilon  \rangle  0$ tal que $\alpha(-\epsilon, \epsilon) \cap S = {p}$. Así, los conexos $\alpha(-\epsilon, 0)$ y $\alpha(0, \epsilon)$ están contenidos en $\rtres - S$. Diremos que $v$ es \textbf{interior} si $\alpha(0, \epsilon) \subset \Omega$, en caso contrario diremos que es \textbf{exterior}. 
\end{definition}

\begin{theorem}[Teorema de Brower-Samelson]
Toda superficie compacta en un espacio euclideo es orientable.
\end{theorem}
\begin{proof}
La idea de esta demostración es encontrar la orientación de esta superficie $S$ compacta. Para ello comencemos viendo que para $v \in (T_pS)^{\bot}$ con $\parallel v \parallel=1$ tenemos que $v$ es interior o bien $-v$ es interior.

Supongamos que $v$ no es interior, luego por definición, $\alpha(\epsilon, 0) \subset \Omega^{*}$ con $\alpha$ la recta afín considerada en la definición previa. Consideremos ahora la recta afín $\beta: \mathbb{R} \longrightarrow \rtres$ dada por $\beta(t)=p + t(-v)$ con $p \in S$. Luego $\beta(0, \epsilon) = \alpha(-\epsilon, 0) \subset \Omega$ y por tanto $-v$ es interior. Ya hemos probado que si $v$ no es interior, lo es $-v$, veamos ahora que no pueden serlo ambos. 

Supongamos que $v$ y $-v$ son interiores. Esto implicaría que $\alpha(0,\epsilon) \subset \Omega$ y $\beta(0, \epsilon) = \alpha(-\epsilon, 0) \subset \Omega$, y por definición, $\alpha(-\epsilon, \epsilon) \cap S \neq \emptyset$.

Tomemos ahora $v \in (T_pS)^{\bot}$ con $\parallel v \parallel=1$ interior (si no fuese interior, tomamos -v). Vamos a buscar un entorno abierto de $p \in S$ para construir la orientación $N: V \longrightarrow \rtres$. Donde $N$ es un campo unitario normal diferenciable con $N(q)$ interior $\forall q \in V.$

Por el lema previo, tenemos que $\exists W$ entorno de abierto y conexo de $p$. Definimos el entorno $V=W\cap S$ entorno de $p$, homeomorfo a $B\cap P$, luego conexo. Como toda superficie es localmente orientable, podemos suponer que $V$ es orientable.

Sea $N: V \longrightarrow \rtres$ la orientación de $V$ tal que $N(p)$ es interior.

%%%% Tomando un entorno coordenado en p e intersecando con V
\end{proof}

\section{Entornos tubulares}

En este sección, vamos a definir los entornos tubulares. Veremos que dada una superficie, bajo determinadas hipótesis, existe un entorno que envuelve la superficie.

Sea $S$ una superficie de $\rtres$, como $\rtres$ es un espacio métrico, los entornos más sencillos de definir son los conocidos como \textbf{entornos métricos}. Definidos como los puntos cuya distancia a la superficie es menor que un delta dado, notamos:

\begin{equation*}
    B_\delta(S)=\{p\in \rtres | dist(p,S)  \langle  \delta\}
\end{equation*}

donde, $dist(p,S) = inf_{q\in S}|p-q|$.

\begin{lemma}
Sea $S$ una superficie cerrada de $\rtres$, el conjunto $B_\delta(S)$ definido anteriormente, coincide con el conjunto $N_\delta(S)=\bigcup_{p\in S}N_\delta(p)$, definido como el los segmentos abiertos en las normales a la superficie S con centro p ($p \in S$) y radio $\delta$.
\end{lemma}
\begin{proof}
Veámoslo por doble inclusión:

Sea $p \in S$, y sea $q \in N_\delta(p)$ para el $p$ dado y $\delta  \rangle  0$. Es directo que, $dist(q,S) \leq |q-p|  \langle  \delta$, luego $q \in B_\delta(S)$.

Supongamos ahora $q \in B_\delta(S)$. Tomamos la función distancia del punto $q$ a $S$. Como $S$ es cerrado, sabemos que existe un mínimo en un punto $p \in S$, además por la caracterización de los punto críticos de la función distancia al cuadrado, sabemso que el punto $q$ está en la normal a $S$ en $p$. Así, $|p-q| = dist(q,S)  \langle  \delta$, luego $q \in N_\delta(p)$.
\end{proof}

Sea $S$ una superficie orientable con la aplicación de Gauss $N: S \longrightarrow \mathbb{S}^2 \subset \rtres$, definimos:

\begin{align*}
    F: S x \mathbb{R} &\longrightarrow \rtres \\
    (p,t) &\longrightarrow p + tN(p)
\end{align*}

Esta aplicación,claramente diferenciable, envía cada punto $p$ de la superficie a la distancia $t$ en dirección de la recta normal a la superficie en el punto $p$. Luego tenemos:

\begin{equation*}
    F(S \times (-\delta, \delta)) = N_\delta(S)=\bigcup_{p\in S} N_\delta(p), \qquad \forall \delta  \rangle  0
\end{equation*}

\begin{definition}[Entornos tubulares]
La unión $N_\delta(S)$ de todos los segmentos normales de radio $\delta  \rangle  0$ centrados en los puntos de una superficie $S$ orientable es llamada \textit{entorno tubular} de radio $\delta$ si es un abierto como subconjunto de $\rtres$ y la función $F: S x (-\delta, \delta) \longrightarrow N(p)$ definida previamente es un difeomorfismo.
\end{definition}

% TODO: Demostrar
\begin{lemma}
Sea $S$ una superficie, $\forall p \in S$, $\exists V_p$, entorno abierto y orientable y $\delta  \rangle  0$ de forma que el conjunto $N_\delta(V_p)$ es un entorno tubular de $V$.
\end{lemma}

Veamos ahora la existencia de entornos tubulares para superficies compactas.
\begin{theorem}[Existencia de entornos tubulares]
Sea $S$ una superficie orientable y $R \subset S$ un subconjunto abierto relativamente compacto. Entonces $\exists \epsilon  \rangle  0$ tal que, el conjunto $N_\epsilon(R)$ es un entorno tubular de la superficie $R$, esto es, es un abierto de $\rtres$ y la función:

\begin{align*}
    F: S \times (-\epsilon, \epsilon) &\longrightarrow N_\epsilon(R) \\
    (p,t) &\longrightarrow p + tN(p)
\end{align*}

es un difeomorfismo.

En particular, cuando la superficie $S$ es compacta, $\exists \epsilon  \rangle  0$ tal que
$B_\epsilon(S)=N_\epsilon(S)$ es un entorno tubular de $S$.
\end{theorem}
\begin{proof}
Como tenemos que $R$ es relativamente compacto, esto es, $\bar{R}$ es compacto, existe un recubrimiento finito por abierto de $\bar{R}$ cada uno con un entorno tubular, por el lema previo. Sea $\delta  \rangle  0$ el menor radio de todos ellos. Si tomamos la función $F$ definida previamente, restringida al intervalo de definición $Rx(-\delta, \delta)$ es un difeomorfismo local. Vamos a buscar un $\epsilon \in (0,\delta)$ tal que la función $F$ restringida al intervalo $Rx(-\epsilon, \epsilon)$ es inyectiva. Veámoslo por reducción al absurdo.

Supongamos que $\exists \epsilon \in (0,\delta)$ tal que $F$ restringida al intervado de definición $Rx(-\epsilon, \epsilon)$ no es inyectiva, o lo que es lo mismo, los segmentos $N_\epsilon(p)$ de normales con $p \in R$ intersecan entre ellos. Tomamos $\epsilon=\frac{1}{n}$ y $\forall p \in \mathbb{N}$, $\exists p_n,q_n \in S$ con $p_n \neq q_n$ tal que $N_{\frac{1}{n}}(p_n) \bigcap N_{\frac{1}{n}}(q_n) \neq \emptyset$.
Como $R$ es relativamente compacto, podemos tomar sucesiones parciales que convergen en $\bar{R}$. Sean $\{p_n\}_{n \in \mathbb{N}}, \{q_n\}_{n \in \mathbb{N}}$ estas sucesiones y sean $p,q \in \bar{R}$ los puntos donde convergen, esto es:

\begin{equation*}
    \lim_{n\to\infty} p_n = p \qquad \lim_{n\to\infty} q_n = q
\end{equation*}

Sea $r_n \in N_{\frac{1}{n}}(p_n) \bigcap N_{\frac{1}{n}}(q_n)$, entonces:

\begin{equation*}
    |p_n - q_n| = |p_n - q_n + r_n - r_n| \leq |p_n-r_n| + |r_n-q_n|  \langle  \frac{1}{n}+\frac{1}{n} = \frac{2}{n}
\end{equation*}

Por tanto los límites coinciden.

Aplicando el lema previo al punto $p = q \in S$, tenemos $V$ y $\rho  \rangle  0$ tal que $N_\rho(V)$ es un entorno tubular. Además, $\exists N_0 \in \mathbb{N}$ tal que, $\forall n  \rangle  N_0$, $p_n,q_n \in V$ y $1/n  \langle  \rho$. Por tanto llegamos a una contradicción:

\begin{equation*}
    N_{\frac{1}{n}}(p_n) \bigcap N_{\frac{1}{n}}(q_n) \subset N_\rho(p_n)\bigcap N_\rho(q_n) = \emptyset
\end{equation*}

ya que $N_\rho(V)$ es un entorno tubular. Por tanto, $F$ restringida al intervalo de definición $V \times (-\rho, \rho)$ es inyectiva y localmente difeomorfismo.

\end{proof}

A partir de los entornos tubulares, introduciremos el concepto de superficie paralela.

\begin{definition}[Superficie paralela]
Sea $\rho  \rangle  0$ y sea $N_\rho(S)$ un entorno tubular de la superficie $S$ compacta con $N$ su aplicación de Gauss. $\forall t \in (-\rho, \rho)$ se define $S_t={p + tN(p); p \in S}$ superficie compacta y la apliación $F_t: S \longrightarrow S_t$ dada por $F_t(p)=p+tN(p)$ es un difeomorfismo.
Llamamos a $S_t$ \textit{superficie paralela} a S a una distancia $t$.
\end{definition}