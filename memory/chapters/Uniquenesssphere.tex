\label{chapter:uniquenesssphere}
En el capítulo previo, hemos probado que las esferas son superficies isoperimétricas, es decir,  encierran un volumen dado con área mínima. En este capítulo, vamos a mostrar que son las únicas. Para esto, vamos a comenzar demostrando que todas las superficies isoperimétricas tienen curvatura media constante. Nos ayudaremos de técnicas variacionales que introduciremos a continuación.

\section{Propiedad variacional de las superficies isoperimétricas}

Sea $S$ una superficie compacta y conexa de $\rtres$, sea $\epsilon > 0$ suficientemente pequeño tal que $N_\epsilon(S)$ sea un entorno tubular. Recordemos que la aplicación $F: S \times (-\epsilon, \epsilon) \longrightarrow N_\epsilon(S)$ dada por $F(p,t)=p + tN(p)$ es un difeomorfismo, donde $N: S \longrightarrow \unitsphere$ es el normal unitario interior sobre $S$.

Sea $f: S \longrightarrow \rmath$ diferenciable. Como $S$ es compacta entonces $f$ alcanza su máximo absoluto sobre $S$. Sea $M=\max \{ |f(p)| \enspace | \enspace p \in S \}$. Definimos $\delta = \frac{\epsilon}{M} > 0$. Tenemos que, si $t \in (-\epsilon, \epsilon)$ y $p\in S$, entonces $|tf(p)| \leq tM < \epsilon$. Esto nos indica que la aplicación $\Phi: S \times (-\delta, \delta) \longrightarrow \rtres$ dada por $\Phi(p,t) = F(p, tf(p)) = p + tf(p)N(p)$ está bien definida.

Dado $t \in (-\delta, \delta)$, sea:
%
\begin{equation*}
    S_t(f) = \{ p + tf(p)N(p) \enspace | \enspace p \in S \}
\end{equation*}
%
Vamos a demostrar que $S_t(f)$ es una superficie compacta y $\Phi_t: S \longrightarrow S_t(f)$ es un difeomorfismo. Por un resultado previo basta probar que la aplicación $\Phi_t: S \longrightarrow \rtres$ dada por $\Phi_t(p)=p+tf(p)N(p)$ es diferenciable, inyectiva, y tiene diferencial inyectiva en cada punto.

Tenemos que $\Phi_t$ es inyectiva. Si $p,q \in S$ con $\Phi_t(p) = \Phi_t(q)$, entonces $F(p, tf(p)) = F(q, tf(q))$ lo que implica que $p=q$ al ser $F: S \times (-\epsilon, \epsilon) \longrightarrow \rtres$ inyectiva.

Es claro que $\Phi_t: S \longrightarrow \rtres$ es diferenciable,  veamos ahora que $(d\Phi_t)_p: T_pS \longrightarrow \rtres$ es inyectiva $\forall p \in S$. Dado $v \in T_pS$, se tiene:
%
\begin{align*}
    (d\Phi_t)_p(v) &= v + t \left( (df)_p(v)N(p) + f(p)(dN)_p(v) \right) \\
    &= \left( v - tf(p)A_p(v) \right) + t(df)_p(v)N(p)
\end{align*}

Nótese que tenemos la expresión dividida en dos partes, la primera es tangente a $S$ y la segunda es normal. Supongamos que $(d\Phi_t)_p(v) = 0$. Entonces ambas partes son $0$ y, en particular, $v - tf(p)A_p(v) = 0$.

Recordemos que $F_u: S \longrightarrow \rtres$ dada por $F_u(p) = p + uN(p)$ tiene diferencial inyectiva si $|u| < \epsilon$. Tomando $u=tf(p)$ tenemos que, $v - tf(p)A_p(v) = (dF_u)_p(v)$, luego $v=0$. Con esto tenemos que $ker((d\Phi_t)_p) = \{0\}$ y por tanto la diferencial de $\Phi_t$ es inyectiva en cada $p\in S$. Concluimos que $S_t(f)$ es una superficie compacta y $\Phi_t: S \longrightarrow S_t(f)$ es un difeomorfismo.

\begin{definition}[Variación]
Se llama \textbf{variación de $S$ correspondiente a la función $f$} a la familia de superficies $S_t(f)$ con $t < \delta$.
\end{definition}

\begin{proposition}[Primera variación del área]
Sea $S_t(f)$, con $t \in (-\delta, \delta)$, la variación de la superficie compacta $S$ para la función diferenciable $f: S \longrightarrow \rmath$. Entonces, la función dada por:
%
\begin{equation*}
    t \longrightarrow A(t) = A(S_t(f))
\end{equation*}
%
es diferenciable, y
%
\begin{equation}\label{variacionarea}
    A'(0) = \diff{}{t}{t=0} A(S_t(f)) = -2 \int_S f(p)H(p) \, dS
\end{equation}
\end{proposition}
\begin{proof}
Sea $\{e_1, e_2\}$ una base ortonormal de curvaturas principales de $S$ en el punto $p \in S$. Por el cálculo antes realizado para $(d\Phi_t)_p$ tenemos:
%
\begin{equation*}
    (d\Phi_t)_p(e_i) = (1 - tf(p)k_i(p))e_i + t(df)_p(e_i)N(p), \qquad i=1,2.
\end{equation*}
%
Un cálculo sencillo muestra que el producto vectorial de estos dos vectores esta dado por:
%
\begin{multline}\label{productoescalar}
    (d\Phi_t)_p(e_1) \wedge (d\Phi_t)_p(e_2) = (1 - 2tf(p)H(p))N(p) - \\ t(\triangledown f)_p + t^2G(p,t)
\end{multline}
%
donde $\triangledown f$ es el gradiente de la función $f$ y $G$ es una función diferenciable definida en $S \times (-\delta, \delta)$ tomando valores en $\rtres$. Aplicando el teorema del cambio de variable, tenemos:
%
\begin{equation*}
    A(t) = A(S_t(f)) = \int_{S_t(f)} 1 = \int_S |\text{Jac} \, \Phi_t| \, dS
\end{equation*}
%
Derivando bajo el signo integral en $t=0$ llegamos a:

\begin{equation*}
    A'(0) = \int_S \left( \diff{}{t}{t=0} |(d\Phi_t)_p(e_1) \wedge (d\Phi_t)_p(e_2)| \right) \, dS
\end{equation*}

Por otro lado, la \autoref{productoescalar} nos dice que:
%
\begin{equation*}
    \diff{}{t}{t=0} |(d\Phi_t)_p(e_1) \wedge (d\Phi_t)_p(e_2)| = -2f(p)H(p),
\end{equation*}
%
lo que concluye la demostración.
\end{proof}


\begin{proposition}[Primera variación del volumen]
Sea $S_t(f)$, con $t \in (-\delta, \delta)$, la variación de la superficie compacta $S$ para la función diferenciable $f: S \longrightarrow \rmath$. Denotemos por $\Omega_t(f)$ al dominio interior encerrado por $S_t(f)$. Entonces, la función dada por:

\begin{equation*}
    t \longrightarrow V(t) = vol \, \Omega_t (f)
\end{equation*}
%
es diferenciable, y:
%
\begin{equation}\label{variacionvolumen}
    V'(0) = \diff{}{t}{t=0} vol \, \Omega_t (f) = - \int_S f(p) \, dS
\end{equation}
\end{proposition}
\begin{proof}
Por la ecuación que hay en la definición \ref{volumensuperficiecompacta}, sabemos que:
%
\begin{equation*}
    V(t) = - \frac{1}{3} \int_{S_t}  \langle p, N_t(p) \rangle \, dS_t,
\end{equation*}
%
donde $N_t$ es el normal interior de $S_t(f)$. Usando de nuevo el teorema del cambio de variable, tenemos:
%
\begin{equation*}
    V(t) = - \frac{1}{3} \int_{S}  \langle N_t \circ \Phi_t, \Phi_t \rangle |\text{Jac} \, \Phi_t| \, dS
\end{equation*}

Como el difeomorfismo $\Phi_t$ es la identidad cuando $t=0$, se sigue que:
%
\begin{equation*}
    (N_t \circ \Phi_t)(p) = \frac{(d\Phi_t)_p(e_1) \wedge (d\Phi_t)_p(e_2)}{|(d\Phi_t)_p(e_1) \wedge (d\Phi_t)_p(e_2)|} = \frac{(d\Phi_t)_p(e_1) \wedge (d\Phi_t)_p(e_2)}{|\text{Jac} \, \Phi_t|(p)}
\end{equation*}
%
donde $\{e_1,e_2\}$ es una base ortonormal en $T_pS$. Sustituyendo en la ecuación previa para $V(t)$ y usando la ecuación \ref{productoescalar}, tenemos:
%
\begin{align*}
    V(t) = - \frac{1}{3} \int_S \left( 1-2tf(p)H(p) \right) \langle N(p), p \rangle \, dS  \\ - \frac{1}{3} \int_S \left( tf(p) - t \langle (\triangledown f)_p, p \rangle  + t^2D(p,t) \right) \, dS,
\end{align*}
%
siendo $D$ una función diferenciable definida en $S \times (-\delta, \delta)$. Derivando cuando $t=0$, llegamos a:
%
\begin{align*}
    V'(0) = \frac{1}{3} \int_S [-f(p) + 2f(p)H(p) \langle N(p),p \rangle  +  \langle (\triangledown f)_p, p \rangle ] \, dS
\end{align*}

Utilizando el corolario \ref{corolariogradiente}:
%
\begin{align*}
    V'(0) &= \frac{1}{3} \int_S -f(p) \, dp + \frac{1}{3} \int_S [2f(p)H(p) \langle N(p),p \rangle  +  \langle (\triangledown f)_p, p \rangle ] \, dS \\ &= - \int_S f(p) \, dS
\end{align*}
\end{proof}

Gracias a la desigualdad isoperimétrica del capítulo anterior y a las fórmulas de variación del área y el volumen recién calculadas, podemos deducir el siguiente resultado.

\begin{theorem}[Las superficies isoperimétricas tienen curvatura media constante]\label{meancurvaturecte}
Sea $S$ una superficie compacta y conexa. Si $S$ es isoperimétrica, entonces $S$ tiene curvatura media constante.
\end{theorem}
\begin{proof}
Sea $S$ una superficie isoperimétrica. Tomamos una función $f: S \longrightarrow \rmath$ diferenciable y la variación $S_t(f)$ de $S$ con respecto a $f$, definida para $t \in (-\delta, \delta)$. Sea $\Omega_t(f)$ al dominio interior encerrado por $S_t(f)$. Definimos la función $h: (-\delta, \delta) \longrightarrow \rmath$ dada por $h(t) = A(S_t(f))^3 - 36\pi(vol \, \Omega_t(f))^2$. LA función $h$ es diferenciable y no negativa gracias a la desigualdad isoperimétrica en $\rtres$. Como $S_0=S$ y asumimos que $S$ es una superficie isoperimétrica, el corolario \ref{cor2} nos dice que $h(0)=0$. Así, $h$ alcanza en $t=0$ su mínimo absoluto, por lo que $h'(0)=0$. Nótese que:
%
\begin{equation*}
    h'(0) = 3A(S_0(f))^2 \diff{}{t}{t=0} A(S_t(f)) - 72\pi vol \, \Omega_0(f) \diff{}{t}{t=0} vol \, \Omega_t(f)
\end{equation*}

Utilizando la \autoref{variacionarea} y \autoref{variacionvolumen} llegamos a:

\begin{equation*}
    h'(0) = 3A(S)^2 \left( -2 \int_S f(p)H(p) \, dS \right) - 72\pi vol \, \Omega \left( - \int_S f(p) \, dS \right)
\end{equation*}

Sacando factor común obtenemos:

\begin{equation*}
    h'(0) = 0 = 6 \int_S f(p)(-A(S)^2H(p) + 12 \pi vol \, \Omega)
\end{equation*}

Al estar igualado a $0$, podemos omitir el producto por un número positivo. Además, esta igualdad se tiene que cumplir para toda función $f$ diferenciable en $S$. En particular, si tomamos $f(p) = 12 \pi vol \, \Omega - A(S)^2H(p)$ se sigue que:
%
\begin{equation*}
    0 = \int_S \left( 12 \pi vol \, \Omega - A(S)^2H(p) \right)^2 \, dS.
\end{equation*}

Luego $12 \pi vol \, \Omega - A(S)^2H(p) = 0$ y despejando $H$:
%
\begin{equation*}
    H(p) = \frac{12 \pi vol \, \Omega}{A(S)^2}, \qquad \forall p \in S
\end{equation*}

Esto prueba que $H$ es constante y concluye la prueba.
\end{proof}

\section{Unicidad de las esferas como superficies isoperimétricas}

En la sección anterior hemos probado que todas las superficies isoperimétricas tienen curvatura media constante. A partir de esto, estamos en condiciones de probar que las esferas son las únicas superficies isoperimétricas, es decir, dado un volumen fijo, cualquier superficie isoperimétrica que encierre dicho volumen es necesariamente una esfera.

Demostrar la unicidad de las esferas como superficies isoperimétricas una vez sabido que las superficies isoperimétricas tienen curvatura media constante es directo mediante el \textbf{teorema de Alexandrov} que nos asegura que si una superficie compacta y conexa tiene curvatura media constante, entonces es una esfera. Sin embargo, para concluir este trabajo, se decide mostrar una demostración alternativa al teorema de Alexandrov, más novedosa, basada en un artículo de Wente \cite{wenteproof} publicado en 1991.

Para llevar la prueba a cabo son necesarios algunos preliminares.Comencemos viendo como se comportan el volumen y el área respecto de homotecias en $\rtres$.

\begin{lemma}
Sea $S$ una superficie compacta y conexa, y $\Omega$ su dominio interior. Dado $\lambda \in \rmath$ positivo, sea $h_\lambda: \rtres \longrightarrow \rtres$ la homotecia de centro $0$ y razón $\lambda$ definida como $h_\lambda(p)=\lambda p$, $\forall p \in \rtres$. Entonces, tenemos que:
%
\begin{align*}
    A(h_\lambda(S)) &= \lambda^2A(S), \\
    vol \, h_\lambda(S) &= \lambda^3 vol \, \Omega.
\end{align*}
\end{lemma}
\begin{proof}
Consideramos la aplicación $h_\lambda: S \longrightarrow h_\lambda(S)$. Como $h_\lambda(S)$ es una homotecia, es un difeomorfismo de $\rtres$ y, por tanto, $h_\lambda(S)$ es una superficie compacta y conexa con $T_{h_\lambda (p)} h_\lambda (S) = T_pS$, $\forall p \in S$.

Veamos primero como cambia el área. Sabemos que:
%
\begin{equation*}
    A(h_\lambda(S)) = \int_{h_\lambda(S)} 1,
\end{equation*}
%
y utilizando el cambio de variable:
%
\begin{equation*}
    A(h_\lambda(S)) = \int_{S} |\text{Jac} \, h_\lambda| \, dS
\end{equation*}

Calculemos $|\text{Jac} \, h_\lambda|$. Tenemos que $(dh_\lambda)_p: T_pS \longrightarrow T_pS$ es el endomorfismo dado por $(dh_\lambda)_p(v) = \lambda v$, $\forall v \in T_pS$. Tomando $\{e_1, e_2\}$ una base ortonormal de $T_pS$, tenemos:
%
\begin{align*}
    (dh_\lambda)_p(e_1) &= \lambda e_1, \\
    (dh_\lambda)_p(e_2) &= \lambda e_2,
\end{align*}
%
luego: 
%
\begin{equation*}
|\text{Jac} \, h_\lambda|(p) = \begin{vmatrix}
\lambda & 0 \\ 
0 & \lambda
\end{vmatrix} = \lambda^2.
\end{equation*}

Volviendo a la expresión para $A(h_\lambda(S))$, tenemos:
%
\begin{equation*}
    A(h_\lambda(S)) = \int_{h_\lambda(S)} 1dp = \int_{S} |\text{Jac} \, h_\lambda| \, dS = \int_{S} \lambda^2 \, dS = \lambda^2A(S).
\end{equation*}
%
Veamos ahora como cambia el volumen. Consideremos la aplicación $h_\lambda: \Omega \longrightarrow h_\lambda(S)$ y utilicemos el cambio de variable:
%
\begin{equation*}
    vol \, h_\lambda(\Omega) = \int_{h_\lambda(\Omega)} 1 = \int_{\Omega} |\text{Jac} \, h_\lambda| \, d\Omega
\end{equation*}

Calculemos $|\text{Jac} \, h_\lambda|$. En este caso, $h_\lambda$ está definida entre abiertos de $\rtres$. Como la diferencial $(dh_\lambda)_p: \rtres \longrightarrow \rtres$ viene dada por $(dh_\lambda)_p(v) = \lambda v$, tomamos la base usual de $\rtres$ como base ortonormal, obtenemos:
%
\begin{equation*}
  |\text{Jac} \, h_\lambda|(p) = \begin{vmatrix}
                        \lambda & 0 & 0 \\ 
                        0 & \lambda & 0 \\
                        0 & 0 & \lambda \\
                        \end{vmatrix} = \lambda^3,
\end{equation*}
%
y podemos concluir que:
%
\begin{equation*}
    vol \, h_\lambda(\Omega) = \int_{h_\lambda(\Omega)} 1 = \int_{\Omega} |\text{Jac} \, h_\lambda| \, d\Omega = \int_{\Omega} \lambda^3 \, d\Omega = \lambda^3 vol \, \Omega.
\end{equation*}
\end{proof}


\begin{theorem}[Unicidad de las superficies isoperimétricas]
Si $S \subset \rtres$ es una superficie isoperimétrica, entonces $S$ es una esfera.
\end{theorem}
\begin{proof}
Vamos a construir deformaciones de $S$ que preserven su volumen mediante homotecias de la variación por paralelas. Sea $\Omega$ el dominio interior encerrado por $S$ y sea $\epsilon > 0$ tal que $N_\epsilon(S)$ es un entorno tubular de $S$. Para cada $t\in (-\epsilon, \epsilon)$ sea $S_t$ la superficie paralela correspondiente y $\Omega_t$ el dominio interior encerrado por $S_t$. Para cada $t\in (-\epsilon, \epsilon)$ sea $\lambda(t) > 0$ tal que $vol \, h_{\lambda(t)} (\Omega_t) = vol \, \Omega$. Como acabamos de probar que $vol \, h_{\lambda(t)}(\Omega_t) = \lambda(t)^3 vol \, \Omega_t$, igualando ambas expresiones nos queda:

\begin{equation*}\label{lambdafunction}
    \lambda(t) = \left( \frac{vol \, \Omega}{vol \, \Omega_t} \right) ^{1/3}, \qquad \forall t \in (-\epsilon, \epsilon).
\end{equation*}

Esto define una aplicación diferenciable $\lambda: (-\epsilon, \epsilon) \longrightarrow \rmath$ con $\lambda(0) = 1$ ya que $\Omega_0 = \Omega$. 

Consideremos ahora la aplicación $A: (-\epsilon, \epsilon) \longrightarrow \rmath$ dada por:
%
\begin{equation}\label{funcA}
    A(t) = A(h_{\lambda(t)} (S_t)) = \lambda(t)^2 A(S_t).
\end{equation}
%
Utilizando la hipótesis de que $S$ es una superficie isoperimétrica, sabemos que $A'(0) = 0$ y $A''(0) \geq 0$ ya que $A$ alcanza su mínimo global en $t=0$.

Vamos a calcular las derivadas $A'(0)$ y $A''(0)$. Como demostramos al final del \autoref{chapter:isoperimetricinequality} tenemos:

\begin{equation*}
    A(S_t) = A(S) -2t\int_{S} H \, dS + t^2\int_{S} K \, dS
\end{equation*}
%
si definimos la función $a: (-\epsilon, \epsilon) \longrightarrow \rmath$ dada por $a(t) = A(S_t)$, la \autoref{funcA} nos queda como:
%
\begin{equation}
    A(t) = \lambda(t)^2 a(t)
\end{equation}

Calculemos las derivadas de $a$:
%
\begin{align*}
    a'(t) &= -2\int_{S} H \, dS + 2t\int_{S} K \, dS, \\
    a''(t) &= 2\int_{S} K \, dS.
\end{align*}
%
evaluando en $t=0$ y teniendo en cuenta que hemos probado en el \autoref{meancurvaturecte} que las superficies isoperimétricas tienen curvatura media constante:
%
\begin{align*}
    a'(0) &= -2\int_{S} H \, dS = -2HA(S), \\
    a''(0) &= 2\int_{S} K \, dS.
\end{align*}

Ahora queremos calcular las derivadas $\lambda'(0)$ y $\lambda''(0)$ de $\lambda(t) = \left( \frac{vol \, \Omega}{vol \, \Omega_t} \right) ^{1/3}$. Por lo probado en la \autoref{volumeparallelsurface}:
%
\begin{equation*}
    vol \, \Omega_t = vol \, \Omega - tA(S) + t^2\int_S H \, dS - \frac{t^3}{3}\int_S K \, dS.
\end{equation*}
%
consideremos la función $v: (-\epsilon, \epsilon) \longrightarrow \rmath$ dada por $v(t) = vol \, \Omega_t$. Calculemos las derivadas de $v$:
%
\begin{align*}
    v'(t) &= - A(S) + 2t\int_S H \, dS - t^2\int_S K \, dS, \\
    v''(t) &= 2\int_{S} H \, dS - 2t\int_S K \, dS.
\end{align*}
%
De nuevo, evaluando en $t=0$, y sabiendo que $H$ es constante:
%
\begin{align*}
    v'(0) &= -A(S), \\
    v''(0) &= 2HA(S).
\end{align*}

Nótese que podemos escribir:
%
\begin{equation*}
    \lambda(t) = (vol \, \Omega)^{1/3} v(t)^{-1/3},
\end{equation*}
%
y al derivar obtenemos:
%
\begin{align*}
    \lambda'(t) &= \frac{-1}{3} (vol \, \Omega)^{1/3} v(t)^{-4/3} v'(t) \\
    \lambda''(t) &= \frac{-1}{3} (vol \, \Omega)^{1/3} \left[ \frac{-4}{3}v(t)^{-7/3} v'(t)^2 + v(t)^{-4/3}v''(t) \right]
\end{align*}
%
sustituyendo las derivadas de $v$ calculadas previamente y evaluando en $t=0$:
%
\begin{align*}
    \lambda'(0) &= \frac{A(S)}{3 vol \, \Omega}, \\
    \lambda''(0) &= \frac{4 A(S)^2}{9 (vol \, \Omega)^{2}} - \frac{2HA(S)}{3 vol \, \Omega} .
\end{align*}

Vamos a simplificar ahora las expresiones conseguidas. Utilizando el \autoref{formulaminkowski}, la \textbf{fórmula de Minkowski}, y teniendo en cuenta de nuevo que $H$ es constante, nos queda:
%
\begin{equation}\label{minkowskiauxfunc}
    A(S) + H\int_S \langle p, N(p) \rangle \, dS = 0.
\end{equation}

Por otro lado, la expresión para $vol \, \Omega$ en la definición \autoref{volumensuperficiecompacta} nos dice que:
%
\begin{equation*}
    vol \, \Omega = \frac{-1}{3}\int_S \langle p, N(p) \rangle \, dS
\end{equation*}
%
sustituyendo esta información en la \autoref{minkowskiauxfunc}, se deduce que:
%
\begin{equation*}
    A(S) - 3H vol \, \Omega = 0
\end{equation*}
%
de donde:
%
\begin{equation*}
    vol \, \Omega = \frac{A(S)}{3H}.
\end{equation*}

Ahora, sustituyendo la igualdad previa en las derivadas de $\lambda$, llegamos a:
%
\begin{align*}
    \lambda'(0) &= H, \\
    \lambda''(0) &= 2H^2.
\end{align*}

Ya podemos calcular las derivadas $A'(0)$ y $A''(0)$ a partir de la \autoref{funcA}. Tenemos:
%
\begin{align*}
    A'(t) &= 2\lambda(t)a(t)\lambda'(t) + a'(t)\lambda(t)^2, \\
    A''(t) &= 2\left [ \lambda'(t)^2a(t) + 2\lambda(t)\lambda'(t)a'(t) + a(t)\lambda(t)\lambda''(t) \right] + a''(t)\lambda(t)^2.
\end{align*}
%
sustituyendo los valores calculados de $a'(0)$, $a''(0)$, $\lambda'(0)$ y $\lambda''(0)$, nos queda:
%
\begin{align*}
    A''(0) &= 2\left[ H^2A(S) -4H^2A(S) + 2H^2A(S) \right] + 2\int_S K \, dS \\
    &= -2H^2A(S) + 2\int_S K \, dS \\
    &= -2\int_S (H^2-K) \, dS.
\end{align*}

Sabemos que $A''(0) \geq 0$, luego $\int_S (H^2-K) \, dS \leq 0$. Como por otro lado toda superficie cumple $H^2-K \geq 0$, deducimos que:

\begin{equation*}
    0 \leq -2\int_S (H^2-K) \, dS \leq 0,
\end{equation*}
%
lo que implica que $H^2-K = 0$ en S. Por tanto, tenemos que $S$ es una superficie compacta, conexa y totalmente umbilical, y por el \autoref{umbilicaltheorem}, \textbf{teorema de clasificación de superficies totalmente umbilicales}, podemos concluir que $S$ es una esfera.
\end{proof}

\paragraph{NOTA}{Una demostración similar funciona en dimensión más alta y permite probar que toda hipersuficie compacta y conexa en $\rmath^n$ minimizando el área (n-$1$)-dimensional con volumen fijo debe ser una esfera.}
