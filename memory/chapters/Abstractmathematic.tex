\begin{otherlanguage}{american}
\pdfbookmark[1]{Abstract}{Abstract}
\section{Abstract}

This work studies the isoperimetric problem in $\rtres$. It prove that a  per given fixed volume in $\rtres$, sphere is the smallest surface area that enclosed this volume and furthemore, sphere is the unique. 

It introduces the surfaces integration theory where we will extend the properties for Lebesgue's integrations to a surfaces. This work shows us the tubular neighbourhood and the parallel surfaces to get the first variation of area and volumen. We use these variations to prove that sphere has mean curvature constant and are the unique isoperimetric surface.


\paragraph{Keywords} Global differential geometry, isoperimetric problem, surfaces, mean curvature, spheres, isoperimetric surfaces, Brunn-Minkowski inequality, first variation formula for area and volume

\end{otherlanguage}

\newpage
