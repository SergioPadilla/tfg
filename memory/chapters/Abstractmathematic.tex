\begin{otherlanguage}{american}
\pdfbookmark[1]{Abstract}{Abstract}
\section{Abstract}

This work studies the isoperimetric problem in $\rtres$. It proves that spheres are compact and connected surfaces that have the minimum area per a given volume. We will prove, moreover, spheres are the unique surfaces that have this property.

It introduces the surfaces integration theory which we will extend the properties for Lebesgue's integrations to surfaces. This work shows us the tubular neighborhood and the parallel surfaces to get the first variation of area and volume. We use these variations to prove that sphere has to mean curvature constant. We will prove the Brunn-Minkowski inequality, the isoperimetric inequality and we will finish this work proving that surfaces are the unique isoperimetric surfaces through Wente proof.


\subsection{Compact and connected surfaces}

One of the first result we could see in this work is that connected compact surfaces separate euclidean space into exactly two domains. That is, we show a three-dimensional version of the famous Jourdan curve theorem, the Jordan-Brower theorem. This theorem allows us to talk about the domain closed by a compact surface in $\rtres$. Using this theorem, we could prove the Brower-Samelson theorem that gives us an orientation for all compact surface in $\rtres$.

\subsection{Tubular Neighbourhoods and parallel surfaces}

This work continues proving that compact surfaces, in the euclidean space, have special neighbourhoods whose allow us to prove some global properties. This special neighbourhoods we show in this work are the tubular neighbourhoods. Using tubular neighbourhoods we will define parallel surfaces.

\subsection{Integration surfaces}

We already know some integration properties and results in the sense of Lebesgue, in a chapter of this work, we will extend this properties for surfaces. We will introduce the absolute value of the jacobian, it will be fundamental to solve some integrations along this work. We will show classic integration theorems like change of variables theorem and the Fubini theorem for surfaces. With the help of these properties of integrations we will calculate the area of sphere, the volume of a open ball in $\rtres$, for this, we will need to use the diverge theorems, and the volume and area closed by a parallel surface.

\subsection{Isoperimetric surfaces}

The isoperimetric surfaces are those that have minimun area per a given volume. These surfaces are the solution for one of the most famous classic problem of optimization, among all the compact surfaces in $\rtres$ whose inner domains have a given volume, which has the least area?. In this work we will give a unique solution for this problem, the speheres. We will prove the Brunn-Minkowski inequality giving previously an introduction to the sum sets and its properties. We will prove the isoperimetric inequality and some direct consequences like sphere is a isoperimetric surface.

Finally, we will introduces the variation tools to calculate the first variation of the area and volume functionals to prove spheres are the unique isoperimetric surfaces.

\paragraph{Keywords} Global differential geometry, isoperimetric problem, surfaces, mean curvature, spheres, isoperimetric surfaces, Brunn-Minkowski inequality, first variation formula for area and volume

\end{otherlanguage}

\newpage
