\section{Conclusiones y vías futuras}

En conclusión, hemos visto cómo se puede extender la funcionalidad de MongoDB para trabajar con conjuntos difusos, que era el objetivo propuesto para este trabajo. Se ha dado una solución con la implementación de la función \texttt{fuzzy\_find} que permite realizar consultas con cláusulas difusas y proyectar el grado de pertenencia a la consulta pedida, pero aún con ciertas limitaciones.

Esta versión de \texttt{fuzzy\_find} solo acepta una cláusula difusa por atributo, además no se pueden combinar cláusulas difusas para el mismo atributo.

Las vías futuras para extender esta funcionalidad pueden resumirse:

\begin{itemize}
    \item Mejorar restricciones: Hay que ampliar la funcionalidad de la función \texttt{fuzzy\_find} para permitir varias cláusulas difusas para un mismo atributo. Además, si se consigue esto, hay que modificar la proyección para poder seleccionar cuales de las condiciones difusas se quiere proyectar.
    \item Implementación de operadores lógicos: Hay que implementar los operadores lógicos para utilizar con cláusulas difusas.
    \item Mejorar indexación: MongoDB permite la indexación en campos de tipo array, un posible estudio de la indexación podría mejorar los tiempos de consulta.
    \item Extender funcionalidad en clientes MongoDB: Ahora mismo, la función \texttt{fuzzy\_find} está probada desde la shell de MongoDB. Un posible trabajo futuro sería extender esta funcionalidad a las librerías para los distintos lenguajes de programación.
    \item Aumentar funcionalidad MongoDB: En esta propuesta nos hemos centrado en el operador de agregación \textit{pipeline}. Se podría estudiar para implementar el trabajo con información imprecisa con el resto de operadores.
\end{itemize}