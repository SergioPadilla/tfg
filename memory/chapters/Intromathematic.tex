\section{Introducción}

En este trabajo se abordará el problema isoperimétrico en $\rtres$. Se demostrará que dado un volumen fijo en $\rtres$, la superficie que encierra dicho volumen con área mínima es la esfera $\unitsphere$. Para llevarlo a cabo, se introducirá la teoría de integración en superficies, las técnicas variacionales y se terminará el trabajo probando que las esferas son las únicas superficies isoperimétricas mediante la prueba de Wente.

\subsection{Contextualización y descripción del trabajo}

El \textbf{Cálculo variacional} es un problema matemático que estudio la solución a problemas de optimización clásicos. Es sabido, y así se estudia en las asignaturas vistas durante el doble grado, que el problema de encontrar una curva plana con perímetro mínimo para encerrar un área dada tiene cómo solución la circunferencia. Pero, ¿qué pasa en espacios de mayor dimensionalidad?. Cuando estudiamos este problema en $\rtres$, esto es, dado un volumen fijo, ¿Cuál es la superficie de todas las que encierran el volumen con área mínima?, los cálculos se complican, pero es más, este problema se entienda a dimensionalidad $\rmath^n$.

Los griegos ya estudiaron el problema isoperimétrico en $\rtres$, sabían que la solución tenía que ser la esfera, pero el primero en probarlo fue H.A. Schwarz usando herramientas de simetrización. Posteriormente, se han realizado otras demostraciones de este problema, pero sin duda, cabe destacar la prueba de A. D. Alexandrov, la cuál se apoya en las propiedades de las soluciones de ciertas ecuaciones en derivadas parciales. Para dar un enfoque más moderno, ya que la demostración de Alexandrov se puede encontrar en casi toda la literatura actual, en este trabajo se dará una demostración alternativa, basada en la prueba de H. Wente en \cite{wenteproof}. Esta demostración tiene un enfoque más geométrico y utiliza técnicas como la clasificación de superficies totalmente umbilicales vista en la asignatura de Curvas y Superficies durante el doble grado.

\subsection{Estructura del trabajo}


