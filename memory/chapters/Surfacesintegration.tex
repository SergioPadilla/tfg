En este capítulo introduciremos los conceptos de integración en superficies. Vamos a ver como los conceptos que conocemos para las integrales en el sentido de Lebesgue son extensibles a superficies. Enunciaremos y probaremos algunos de los teoremas más utilizados en la teoría de integración como los teoremas como el cambio de variable o el teorema de Fubini. Por último, veremos los teoremas de divergencia que aplicaremos en capítulos posteriores.

\section{Integración en Superficies}

Comenzamos esta sección con la definición de Jacobiano definido entre superficies, que nos será fundamental en la última parte del trabajo y utilizaremos en muchas demostraciones.

\begin{definition}(Valor absoluto del Jacobiano)
Sea $\Phi: S \longrightarrow S'$ un difeomorfismo entre dos superficies. Se define el \textit{valor absoluto del jacobiano de $\Phi$} como la aplicación
\begin{align*}
    |Jac \Phi|: S &\longrightarrow \rmath \\
    p &\longrightarrow |det(d\Phi)_p| = |det(M_p)|
\end{align*}

donde $M_p=M((d\Phi)_p, (B_1)_p, (B_2)_{\Phi(p)})$, siendo $(B_1)_p$ base ortonormal de $T_pS$ y $(B_2)_{\Phi(p)}$ base ortonormal de $T_{\Phi(p)}S'$.
\end{definition}

\begin{definition}[Integral en $S \times \rmath$]
Sea $O$ un subconjunto abierto de $\rtres$, y sea el difeomorfismo $\phi: S \longrightarrow R \times (a,b)$ con $a < b$. Podemos decir que la función $h: R \times (a,b) \longrightarrow \rmath$ es integrable en su dominio cuando $(h \circ \phi)|Jac \phi|$ sea integrable en $O$ en el sentido de Lebesgue.

En ese caso, llamaremos \textbf{integral de h en $R \times (a,b)$} al número real dado por:

\begin{equation*}
    \int_{R \times (a,b)} h = \int_{R \times (a,b)} h(p, t)dp dt = \int_{O} (h \circ \phi)|Jac \phi|
\end{equation*}
\end{definition}

\begin{definition}[Integral en $S$]
Sea $R$ una region de una superficie orientable $S$ y sea $f:R \longrightarrow \rmath$ una función cualquiera. Diremos que $f$ es integrable en $R$ cuando la función $(p,t) \in R \longrightarrow (0,1) \mapsto f(p)$ sea integrable en $R \times (0,1)$. Llamaremos \textbf{integral de $f$ en $R$} al número real dado por:

\begin{equation*}
    \int_{R} f = \int_{R \times (0,1)} f(p,t)dp dt
\end{equation*}
\end{definition}

\begin{theorem}[Teorema del cambio de variable]
Sea $\phi:R_1 \longrightarrow R_2$ un difeomorfismo entre dos regiones de dos superficies orientables, y sea $f: R_2 \longrightarrow \rmath$ una función integrable. Entonces la función $(f \circ \phi)|Jac \phi|$ es integrable en $R_1$ y además:

\begin{equation*}
    \int_{R_2} f = \int_{R_1} (f \circ \phi)|Jac \phi|
\end{equation*}
\end{theorem}

\begin{theorem}[Teorema de Fubini]
Sea $R$ una región de una superficie orientable, $a,b \in \rmath, a < b$ y $h:R \times (a,b) \longrightarrow \rmath$ una función integrable en $R \times (a,b)$. Entonces, para casi todo $t \in (a,b)$, la función $p \in R \mapsto h(p,t)$ es integrable en $R$ y para casi todo $p \in R$ la función $t \in (a,b) \mapsto h(p,t)$ es integrable en $(a,b)$. Además, las funciones

\begin{equation*}
    p \in R \mapsto \int_a^b h(p,t), \quad  t \in (a,b) \mapsto \int_S h(p,t) dp
\end{equation*}

son integrales en $R$ y $(a,b)$ respectivamente.

Finalmente, tenemos que:

\begin{equation*}
    \int_{R \times (a,b)} h(p,t) dpdt = \int_R \left( \int_a^b h(p,t)dt \right) dp = \int_a^b \left( \int_R h(p,t)dp \right) dt
\end{equation*}
\end{theorem}


\section{Teorema de Divergencia}
Para finalizar este capítulo, en esta sección vamos a enunciar el teorema de divergencia y algunas consecuencias directas de este que, de nuevo, necesitaremos para demostrar algunos de los resultados más importantes en los capítulos finales.

\begin{definition}[Campo vectorial diferenciable]
Sea $A$ un subconjunto de $\rtres$, se define un \textbf{campo vectorial diferenciable} como una aplicación diferenciable $X: A \longrightarrow \rtres$.
\end{definition}

\begin{definition}[Divergencia de un campo vectorial diferenciable]
Se define la \textbf{divergencia de X} como la función $div X: A \longrightarrow \rmath$ dada por $divX(p) = Traza(dX)_p$ con $p \in A$.
\end{definition}

\begin{theorem}[Teorema de divergencia clásico]\label{divergencetheorem}
Sea $S$ una superficie conexa y compacta y $\Omega$ el dominio interior determinado por $S$. Si $X: \overline{\Omega} \longrightarrow \rtres$ es un campo vectorial diferenciable, entonces:

\begin{equation*}
    \int_\Omega div X = -\int_S  \langle X,N \rangle
\end{equation*}

donde $N: S \longrightarrow \mathbb{S}^2$ es el campo normal interior a lo largo de $S$.
\end{theorem}

\begin{definition}[Volumen encerrado por una superficie compacta]\label{volumensuperficiecompacta}
Si $S$ es una superficie compacta y $\Omega$ su dominio interior. Sea el campo vectorial diferenciable identidad $X: \overline{\Omega} \longrightarrow \rtres$ definido por $X(x) = x \quad \forall p \in \overline{\Omega}$. Su divergencia es la función constante 3. Entonces por el teorema de divergencia:
\begin{equation*}
    vol \Omega = - \frac{1}{3} \int_S  \langle p, N(p) \rangle  dp
\end{equation*}

donde $N$ es el interior normal unitario de $S$.
\end{definition}

\begin{theorem}[Teorema de divergencia en superficies]\label{divergesurfaces}
Sea $S$ una superficie compacta y $V: S \longrightarrow \rtres$ un campo vectorial diferenciable en $S$. Entonces, tenemos:

\begin{enumerate}
    \item $\int_S div V = -2 \int_S \langle V,N \rangle H$
    \item $\int_S [k_2(p) \langle (dV)_p(e_1),e_1 \rangle  + k_1(p) \langle (dV)_p(e_2),e_2 \rangle ]dp = -2 \int_S \langle V,N \rangle K$ donde $\{e_1,e_2\}$ es una base ortonormal de direcciones principales en $p \in S$.
\end{enumerate}

donde $H$, $k_1$ y $k_2$ son las curvaturas media y principales de la superficie.
\end{theorem}

\begin{definition}[Gradiente]
Sea $f$ una función diferenciable sobre una superficie $S$ y $p \in S$. Consideremos las diferencial $(df)_p: T_pS \longrightarrow \rmath$. Entonces por teoría de espacios vectoriales euclídeos, existe un único $x \in T_pS$ tal que $(df)_p(v)=  \langle x,v \rangle $, $\forall v \in T_pS$.

Denotaremos $x= \gradientef $ y lo llamaremos \textbf{gradiente de f en p}. Así:

\begin{equation*}
    \langle \gradientef, v \rangle = (df)_p(v), \qquad \forall v \in T_pS
\end{equation*}

En particular, si $\{e_1, e_2\}$ es una base ortonormal se tiene:

\begin{equation*}
     \gradientef = \langle  \gradientef , e_1 \rangle e_1 + \langle \gradientef, e_2 \rangle e_2 = (df)_p(e_1)e_1 + (df)_p(e_2)e_2
\end{equation*}
\end{definition}

\begin{corolario}\label{corolariogradiente}
Si $f: S \longrightarrow \rmath$ es una función diferenciable sobre una superficie compacta, entonces:

\begin{equation*}
    \int_S \langle \gradientef, p \rangle dS + 2 \int_S f(p)dS = -2 \int_S f(p)H(p) \langle p, N(p) \rangle dS
\end{equation*}
\end{corolario}
\begin{proof}
Aplicando el \textbf{teorema de divergencia para superficies} [\ref{divergesurfaces}] a $V(p)=f(p)p$. Por la definición de divergencia, tenemos:

\begin{align*}
    div_S V &= \sum_{i=1}^2 \langle (dV)_p(e_i), e_i \rangle \\
    &= \sum_{i=1}^2 (df)_p(e_i) \langle p, e_i \rangle + 2f(p)
\end{align*}

calculando el producto escalar de $p$ con el gradiente de $f$ en $p$:

\begin{equation*}
    \langle p, \gradientef \rangle = \langle p, (df)_p(e_1)e_1 + (df)_p(e_2)e_2 \rangle = \sum_{i=1}^2 (df)_p(e_i) \langle p, e_i \rangle
\end{equation*}

y por tanto:

\begin{equation*}
    (div_S V)(p) = \langle \gradientef, p \rangle + 2f(p)
\end{equation*}
\end{proof}
