\section{Integración en Superficies}

En este capítulo introduciremos los conceptos de integración en superficies. Veremos la fórmula del área y los teoremas de divergencia que aplicaremos en capítulos posteriores.

\begin{definition}(Valor absoluto del Jacobiano)
Sea $\Phi: S \longrightarrow S'$ un difeomorfismo entre dos superficies. Se define el \textit{valor absoluto del jacobiano de $\Phi$} como la aplicación
\begin{align*}
    |Jac \Phi|: S &\longrightarrow \mathbb{R} \\
    p &\longrightarrow |det(d\Phi)_p| = |det(M_p)|
\end{align*}

donde $M_p=M((d\Phi)_p, (B_1)_p, (B_2)_{\Phi(p)})$, siendo $(B_1)_p$ base ortonormal de $T_pS$ y $(B_2)_{\Phi(p)}$ base ortonormal de $T_{\Phi(p)}S'$.
\end{definition}

\begin{definition}[Integral en $S \times \mathbb{R}$]
Sea $O$ un subconjunto abierto de $\rtres$, y sea el difeomorfismo $\phi: S \longrightarrow R \times (a,b)$ con $a < b$. Podemos decir que la función $h: R \times (a,b) \longrightarrow \mathbb{R}$ es integrable en su dominio cuando $(h \circ \phi)|Jac \phi|$ sea integrable en $O$ en el sentido de Lebesgue.

En ese caso, llamaremos \textit{integral de h en Rx(a,b)} al número real dado por:

\begin{equation*}
    \int_{R \times (a,b)} h = \int_{R \times (a,b)} h(p, t)dp dt = \int_{O} (h \circ \phi)|Jac \phi|
\end{equation*}
\end{definition}

\begin{definition}[Integral en $S$]
Sea $R$ una region de una superficie orientable $S$ y sea $f:R \longrightarrow \mathbb{R}$ una función. Diremos que $f$ es integrable en $R$ cuando la función $(p,t) \in Rx(0,1) \mapsto f(p)$ sea integrable en $R \times (0,1)$. Llamaremos \textit{integral de $f$ en $R$} al número real dado por:

\begin{equation*}
    \int_{R} f = \int_{R \times (0,1)} f(p)dp dt
\end{equation*}
\end{definition}

\begin{theorem}[Teorema del cambio de variable]
Sea $\phi:R_1 \longrightarrow R_2$ un difeomorfismo entre dos regiones de dos superficies orientables, y sea $f: R_2 \longrightarrow \mathbb{R}$ una función integrable. Entonces la función $(f \circ \phi)|Jac \phi|$ es integrable en $R_1$ y además:

\begin{equation*}
    \int_{R_2} f = \int_{R_1} (f \circ \phi)|Jac \phi|
\end{equation*}
\end{theorem}

\begin{theorem}[Teorema de Fubini]
Sea $R$ una región de una superficie orientable, $a,b \in \mathbb{r}, a < b$ y $h:R \times (a,b) \longrightarrow \mathbb{R}$ una función integrable en $R \times (a,b)$. Entonces, para casi todo $t \in (a,b)$, la función $p \in R \mapsto h(p,t)$ es integrable en $R$ y para casi todo $p \in R$ la función $t \in (a,b) \mapsto h(p,t)$ es integrable en $(a,b)$. Además, las funciones 

\begin{equation*}
    p \in R \mapsto \int_a^b h(p,t), \quad  t \in (a,b) \mapsto \int_S h(p,t) dp
\end{equation*}

son integrales en $R$ y $(a,b)$ respectivamente.

Finalmente, tenemos que:

\begin{equation*}
    \int_{R \times (a,b)} h(p,t) dpdt = \int_R \left( \int_a^b h(p,t)dt \right) dp = \int_a^b \left( \int_R h(p,t)dp \right) dt
\end{equation*}
\end{theorem}


\section{Teorema de Divergencia}
En esta sección vamos a demostrar el Teorema de divergencia, para ello, vamos a utilizar la generalización del teorema del cambio de variable para las integrales de Lebesgue en el caso de que sea una función diferenciable cualquiera en lugar de un difeomorfismo. Esta generalización es la llamada fórmula del área.

\begin{theorem}[Fórmula del área]
Sea $O$ un abierto de $\rtres$ relativamente compacto, $\phi: O \longrightarrow \rtres$ una función diferenciable y $f: O \longrightarrow \mathbb{R}$ una función cualquiera. Supongamos que el producto $f|Jac(\Phi)|$ es integrable en $O$. Entonces tenemos que la función $n(\phi, f)$ definida como:

\begin{align*}
    n(\phi, f): \rtres - \phi(N) &\longrightarrow \mathbb{R} \\
    x &\longrightarrow \sum_{p \in \phi^{-1}(x)} f(p)
\end{align*}

donde $N \subset O$ es el subconjunto de punto críticos de $\phi$ en $O$ (esto es donde se anula el jacobiano), es integrable en $\rtres$ y 

\begin{equation*}
    \int_{\rtres} n(\phi, f) = \int_O f(x)|Jac \phi|(x)dx
\end{equation*}
\end{theorem}

\begin{theorem}[Fórmula del área en productos]
Sea $\phi: S \times (a,b) \longrightarrow \rtres$ una apliación diferenciable, con $S$ una superficie compacta y $a,b \in \mathbb{R}$, $a < b$. Sea $f$ una función integrable en $S \times (a,b)$. Entonces, la función $n(\phi, f)$ dada por:

\begin{equation*}
    n(\phi, f) = \sum_{(p,t) \in \phi^{-1}(x)} f(p,t)
\end{equation*}

está bien definida para cada $x \in \rtres$ excepto en un conjunto de medida nula y:

\begin{equation*}
    \int_{\rtres} n(\phi, f) = \int_{S \times (a,b)} f(p,t)|Jac \phi|(p,t)dpdt
\end{equation*}
\end{theorem}

\begin{definition}[Campo vectorial diferenciable]
Sea $A$ un subconjunto de $\rtres$, se define un campo vectorial diferenciable como una aplicación diferenciable $X: A \longrightarrow \rtres$.
\end{definition}

\begin{definition}[Divergencia de un campo vectorial diferenciable]
Se define la divergencia de X como la función $div X: A \longrightarrow \mathbb{R}$ dada por $divX(p) = Traza(dX)_p$ con $\forall \in A$.
\end{definition}

\begin{theorem}[Teorema de divergencia]
Sea $S$ una superficie conexa y compacta y $\Omega$ el dominio interior determinado por $S$. Si $X: \bar{\Omega} \longrightarrow \rtres$ es un campo vectorial diferenciable, entonces:

\begin{equation*}
    \int_\Omega div X = -\int_S <X,N>
\end{equation*}

donde $N: S \longrightarrow \mathbb{S}^2$ es el campo normal interior a lo largo de $S$.
\end{theorem}

\begin{definition}[Volumen encerrado por una superficie compacta]
Si $S$ es una superficie compacta y $\Omega$ su dominio interior. Sea campo vectorial diferenciable identidad $X: \bar{\Omega} \longrightarrow \rtres$ definido por $X(x) = x \quad \forall p \in \bar{\Omega}$. Su divergencia es la función constante. Entonces por el teorema de divergencia:
\begin{equation*}
    vol \Omega = - \frac{1}{3} \int_S <p, N(p)> dp
\end{equation*}

donde $N$ es el interior normal unitario de $S$.
\end{definition}

\begin{theorem}[Teorema de divergencia en superficies]
Sea $S$ una superficie compacta y $V: S \longrightarrow \rtres$ un campo vectorial diferenciable en $S$. Entonces, tenemos:

\begin{enumerate}
    \item $\int_S div V = -2 \int_S<V,N>H$
    \item $\int_S [k_2(p)<(dV)_p(e_1),e_1> + k_1(p)<(dV)_p(e_2),e_2>]dp = -2 \int_S<V,N>K$ donde ${e_1,e_2}$ es una base ortonormal de direcciones principales en $p \in S$.
\end{enumerate}

donde $H, k_1 y k_2$ son las curvaturas media y principales de la superficie.
\end{theorem}