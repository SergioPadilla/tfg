En este capítulo, haremos una introducción a la teoría de conjuntos difusos. Veremos qué son y las principales operaciones con estos conjuntos. Esta permite nos permite modelar matemáticamente información imperfecta, que es usada frecuentemente por los humanos en su comunicación y razonamiento.

Vamos a ver que bases de datos difusas existen y sus principales características para adentrarnos en el mundo de la representación difusa en base de datos. El objetivo de esta es dotar a los sistemas automáticos de una poderosa forma de toma de decisiones, similares a las que tomamos las personas, basada en información que no pretende ser del todo exacta.

\section{Teoría de conjuntos difusa}

La principal teoría para habla de datos imprecisos fue introducida por L.A Zadeh \cite{fuzzysetszadeh} en 1965. Vamos a basarnos en ella para explicar los conjuntos difusos.

La teoría de conjuntos difusos (\textit{fuzzy sets}) hace un generalización de la teoría de conjuntos clásicas, en la que se define un conjunto como un grupo de elementos que pertenecen o no a este conjunto. En la teoría de conjuntos difusa, se añade un elemento más a tener en cuenta, el grado de pertenencia al conjunto, esto es, cada elemento pertenece a un conjunto con un grado de pertenencia determinado, este grado de pertenencia suele representarse con un número $x \in [0,1]$. Haciendo una analogía, en la teoría de conjuntos clásica, los elementos pertenecen con solo dos posibles valores $0$ ó $1$, que indicarían si pertenece o no al conjunto.

Basándonos en los descrito anteriormente, vamos a dar una definición de conjunto difuso.

\begin{definition}[Conjunto difuso]
Sea $\Omega$ un dominio (de objetos), notemos a los elementos de $\Omega$ por $x$. Sea $f: \Omega \longrightarrow [0,1]$ una función. Definimos un \textbf{conjunto difuso}, $F$, como:

\begin{equation*}
    F = \left\{ f(x)/x \enspace | \enspace x\in\Omega, f(x) \in [0,1] \right\}
\end{equation*}
\end{definition}

Es común no dar una lista exhaustiva de elementos del conjunto difuso, basta con dar una definición de la función $f$, que se denomina \textbf{función de pertenencia}.

Estos conceptos suelen ser muy subjetivos y en la práctica, suelen depender del contexto en el que se encuentren, esto es, un mismo valor, depende del contexto, puede pertenecer a un conjunto u otro. Veamos un mismo ejemplo en dos contextos distintos para ilustrarlo, aprovecharemos el ejemplo para afianzar la definición de conjunto difuso.

\begin{example}
Supongamos que tenemos los datos de la tabla \ref{datatable} con la altura de 5 personas:

\begin{table}[h]
\centering
\begin{tabular}{|l|l|}
\hline
\textbf{Nombre} & \textbf{Altura} \\ \hline
Juan            & 175             \\ \hline
Pepe            & 195             \\ \hline
Marco           & 170             \\ \hline
Felipe          & 180             \\ \hline
Antonio         & 185             \\ \hline
\end{tabular}
\caption{Datos de altura}
\label{datatable}
\end{table}

Vamos a definir el conjunto difuso que representa a los jugadores \textit{altos} en dos deportes distintos:

Supongamos como primer deporte el baloncesto, y podríamos tener una estimación del conjunto difuso como sigue:
%TODO Mira el pdf, algún paquete te transforma los puntos decimales en comillas. Corrígelo!.
\begin{equation*}
    altos = \{0.8/195, 0.3/180, 0.35/185\}
\end{equation*}

sin embargo, para un deporte como el fútbol, podríamos tener el siguiente conjunto:

\begin{equation*}
    altos = \{0.2/175, 0.1/170, 1/195, 0.7/180, 0.82/185\}
\end{equation*}

\end{example}

Aprovechando el ejemplo previo, vamos a introducir un nuevo concepto:

\begin{definition}[Etiqueta lingüística]
Llamamos \textbf{etiqueta lingüística} a aquella palabra en lenguaje natural que describe un conjunto difuso.
\end{definition}

\subsection{Conceptos sobre conjuntos difusos}

Similar a los conceptos de la teoría clásica de conjuntos, vamos a dar una introducción a los conceptos sobre conceptos difusos, veremos como todos ellos depende de la función de pertenencia.

Sean $\Omega$ un dominio, y sean $F_1$ y $F_2$ dos conjuntos difusos con $f_1$ y $f_2$ sus funciones de pertenencia, respectivamente. Entonces, se tiene:

\begin{itemize}
    \item $F_1$ y $F_2$ son \textbf{iguales} si, y solo si, las funciones de pertenencia son iguales. Esto es, $F_1 = F_2 \Leftrightarrow f_1(x) = f_2(x) \forall x \in \Omega$.
    \item $F_1$ está \textbf{incluido} en $F_2$ si, y solo si, $f_1$ es menor o igual que $f_2$. Esto es, $F_1 \subseteq F_2 \Leftrightarrow f_1(x) \leq f_2(x) \forall x \in \Omega$.
    \item Se define el \textbf{soporte} de $F_1$ como el subconjunto de valores de $\Omega$ tal que la función de pertenencia es mayor que cero. Esto es, $sop(F_1) = \{ x \in \Omega \enspace | \enspace f_1(x) > 0 \}$
    \item Se define el \textbf{$\alpha$-corte} de $F_{1_\alpha}$ como el subconjunto de $\Omega$ tal que la función de pertenencia es mayor o igual que un valor dado, $\alpha$. Esto es, $F_{1_\alpha} = \{ x \in \Omega \enspace | \enspace f_1(x) \geq \alpha \}$.
    \item Se define el \textbf{núcleo} de $F_1$ como el subconjunto de $\Omega$ tal que el grado de pertenencia es 1. Esto es, $ker(F_1) = \{ x\in \Omega \enspace | \enspace f_1(x) = 1 \}$.
    \item Se define la \textbf{altura} de $F_1$ como el supremo de todos los grados de pertenencia. Esto es, $hgt(F_1) = \sup_{x\in \Omega} f_1(x)$.
    \item Se dice que un conjunto $F_1$ difuso está \textbf{normalizado} si, y solo si, la altura es igual a 1. Esto es, $\exists x \in \Omega$ tal que $f_1(x) = 1$.
\end{itemize}

\subsection{Números difusos}\label{fuzzynumbers}

Los números difusos fueron introducidos por Zadeh \cite{fuzzynumberszadeh} para poder trabajar con información imprecisa de forma práctica, posteriormente, otros trabajos han ido refinando y redefiniendo este concepto. En esta sección, vamos a definirlos y vamos a ver cómo trabajar con ellos. Es una sección muy importante para nuestro trabajo, ya que será con números difusos como representaremos la información en la base de datos. 

\begin{definition}[Número difuso]
Sea $F$ un conjunto difuso en $\Omega$ con $f$ su función de pertenencia. Diremos que $F$ es un \textbf{número difuso} si cumple:

\begin{enumerate}
    \item f es convexa
    \item f es semi-continua superiormente
    \item El soporte de $F$ está acotado 
\end{enumerate}
\end{definition}

Veamos ahora la forma general de un número difuso. Sean $\alpha, \beta, \gamma, \delta \in \Omega$ con $\alpha \leq \beta \leq \gamma \leq \delta$, entonces la \textbf{forma de pertenencia general de un número difuso} es:

\begin{equation*}
    f(x) = \left\{ { \begin{array}{cc}
                    0 & x < \alpha \\ 
                    r(x) & x\in [\alpha,\beta) \\
                    h & x\in [\beta,\gamma] \\
                    s(x) & x\in (\gamma,\delta] \\
                    0 & x > \delta \\ 
                    \end{array}  } \right.
\end{equation*}

donde $r: \Omega \longrightarrow [0,1]$ es no decreciente, $s: \Omega \longrightarrow [0,1]$ es no creciente y $h \in (0,1]$ con $r(\beta) = h = s(\gamma)$.

Los números difusos que nos interesan a nosotros especialmente, y que utilizaremos para dar nuestra solución son aquellos en los que $r,s$ son lineales. Estos son llamados los \textbf{trapezoides} y tienen la siguiente forma de pertenencia de número difuso.

\begin{equation}\label{trapezoidrepresentation}
    f(x) = \left\{ { \begin{array}{cc}
                    0 & x < \alpha \\ 
                    h + \frac{(x-\beta)h}{\beta-\alpha} & x\in [\alpha,\beta) \\
                    h & x\in [\beta,\gamma] \\
                    h + \frac{(x-\gamma)h}{\delta-\gamma} & x\in (\gamma,\delta] \\
                    0 & x > \delta \\ 
                    \end{array}  } \right.
\end{equation}

Además, habitualmente estará normalizado ($h=1$) y por tanto nos basta con una tupla $(\alpha, \beta, \gamma, \delta)$ para definir un número difuso.

\begin{remark}\label{notaciontrapezoide}
Notemos que esta representación nos vale incluso cuando se tengan solo 1,2 o 3 valores. Estos serán llamados \textit{crisp}, \textit{intervalo} y \textit{triangulares} respectivamente. Con un valor $a$, tendremos $\alpha = \beta = \gamma = \delta = a$, con dos valores, $a,b$ tendremos $\alpha = \beta = a$ y $\gamma = \delta = b$ y con tres valores, $a,b,c$ tendremos $\beta = \gamma = b$.
\end{remark}
