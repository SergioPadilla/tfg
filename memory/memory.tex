% **************************************************************************************************************
% A Classic Thesis Style
% An Homage to The Elements of Typographic Style
%
% Copyright (C) 2015 André Miede http://www.miede.de
%
% If you like the style then I would appreciate a postcard. My address 
% can be found in the file ClassicThesis.pdf. A collection of the 
% postcards I received so far is available online at 
% http://postcards.miede.de
%
% License:
% This program is free software; you can redistribute it and/or modify
% it under the terms of the GNU General Public License as published by
% the Free Software Foundation; either version 2 of the License, or
% (at your option) any later version.
%
% This program is distributed in the hope that it will be useful,
% but WITHOUT ANY WARRANTY; without even the implied warranty of
% MERCHANTABILITY or FITNESS FOR A PARTICULAR PURPOSE.  See the
% GNU General Public License for more details.
%
% You should have received a copy of the GNU General Public License
% along with this program; see the file COPYING.  If not, write to
% the Free Software Foundation, Inc., 59 Temple Place - Suite 330,
% Boston, MA 02111-1307, USA.   
%
% **************************************************************************************************************
\RequirePackage{fix-cm} % fix some latex issues see: http://texdoc.net/texmf-dist/doc/latex/base/fixltx2e.pdf
\documentclass[ oneside,openany,titlepage,numbers=noenddot,headinclude,%1headlines,% letterpaper a4paper
                footinclude=true,cleardoublepage=empty,abstractoff, % <--- obsolete, remove (todo)
                BCOR=5mm,paper=a4,fontsize=11pt,%11pt,a4paper,%
                spanish,american%
                ]{scrreprt}

%********************************************************************
% Note: Make all your adjustments in here
%*******************************************************
\input{classicthesis-config}
%----------------------------------------------------------------------------------------
%   ALIAS
%----------------------------------------------------------------------------------------
\newcommand{\rmath}{\mathbb{R}}
\newcommand{\rdos}{\mathbb{R}^2}
\newcommand{\rtres}{\mathbb{R}^3}
\newcommand{\rdostortres}{\rdos \longrightarrow \rtres}
\newcommand{\rtrestordos}{\rtres \longrightarrow \rdos}
\newcommand{\umath}{\mathbb{U}}
\newcommand{\vmath}{\mathbb{V}}
\newcommand{\tmath}{\mathbb{T}}
\newcommand{\rtrestou}{\rtres \longrightarrow \umath}
\newcommand{\utortres}{\umath \longrightarrow \rtres}
\newcommand{\rtrestov}{\rtres \longrightarrow \vmath}
\newcommand{\vtortres}{\vmath \longrightarrow \rtres}
\newcommand{\utov}{\umath \longrightarrow \vmath}
\newcommand{\xmath}{\mathbb{X}}
\newcommand{\cinf}{\mathbb{C}^\infty}
\newcommand{\dx}{(d\mathbb{X})}
\newcommand{\dxq}{(d\mathbb{X})_{q}}
\newcommand{\cn}{\mathbb{C}^n}
\newcommand{\nmath}{\mathbb{N}}
\newcommand{\unitsphere}{\mathbb{S}^2(1)}
\newcommand{\gradientef}{(\triangledown f)_p}

%----------------------------------------------------------------------------------------
%   Formats
%----------------------------------------------------------------------------------------
\usepackage{amsthm}
% \newtheorem{satz}{Satz}[chapter]
% \newtheorem*{satz*}{Satz}
% \newtheorem{lemma}[satz]{Lemma}
% \newtheorem{corollar}[satz]{Korollar} 
% \newcommand{\satzautorefname}{Satz}
\newtheorem{theorem}{Teorema}[chapter]
\newtheorem{lemma}{Lema}
\newtheorem{proposition}{Proposición}
\newtheorem{corolario}{Corolario}
%\newtheorem{lemma}[theorem]{Lema}
%\newtheorem{prop}[theorem]{Proposición}
%\newtheorem{cor}[theorem]{Corolario}

\theoremstyle{definition}
\newtheorem{definition}{Definición}[chapter]
\newtheorem{example}{Ejemplo}[chapter]
\newtheorem{exca}{Ejercicio}[chapter]

\theoremstyle{remark}
\newtheorem{remark}{Observación}[chapter]

%********************************************************************
% Bibliographies
%*******************************************************
\addbibresource{./bibliography.bib}

%********************************************************************
% Hyphenation
%*******************************************************
%\hyphenation{put special hyphenation here}


% ********************************************************************
% GO!GO!GO! MOVE IT!
%*******************************************************
\begin{document}
\frenchspacing
\raggedbottom
\selectlanguage{spanish} % american ngerman
%\renewcommand*{\bibname}{new name}
%\setbibpreamble{}
\pagenumbering{roman}
\pagestyle{plain}
%********************************************************************
% Frontmatter
%*******************************************************
%\include{FrontBackmatter/DirtyTitlepage}
%*******************************************************
% Titlepage
%*******************************************************
\begin{titlepage}
    % if you want the titlepage to be centered, uncomment and fine-tune the line below (KOMA classes environment)
    \begin{addmargin}[-3.45cm]{-3cm}
    \begin{center}
        \large  

        \hfill

        \includegraphics[width=8cm]{gfx/ugr_icon} \\ \medskip

        \vfill

        \begingroup
            \color{NavyBlue}\spacedallcaps{\myTitle} \\ \bigskip
        \endgroup

        \spacedlowsmallcaps{\mySubtitle}

        \vfill

        \myWork \\ 
        \myDegree \\ \bigskip
        \textbf{Autor} \\
        \myName \\ \medskip
        \textbf{Tutores} \\
        \myProf \\
        \myOtherProf \\ \bigskip
        %\myDepartment \\                            
        \spacedlowsmallcaps{\myFaculty} \\ \medskip
        \spacedlowsmallcaps{\myOtherFacultyA} \\
        \spacedlowsmallcaps{\myOtherFacultyB} \\ \bigskip

        \myLocation, \myTime %\ -- \myVersion

        \vfill                      

    \end{center}  
  \end{addmargin}       
\end{titlepage}   
\include{FrontBackMatter/Titleback}
\cleardoublepage\include{FrontBackMatter/Libraryconsent}
\cleardoublepage%*******************************************************
% Declaration
%*******************************************************
%\refstepcounter{dummy}
%\pdfbookmark[0]{Declaration}{declaration}
\chapter*{}
\thispagestyle{empty}
D. \textbf{\myProf} y D. \textbf{\myOtherProf}, profesores del Departamento de Geometría y Topología y del departamento de Ciencias de la Computación e Inteligencia
Artificial de la Universidad de Granada, respectivamente.

\vspace{0.5cm}

\textbf{Informan:}

\vspace{0.5cm}

Que el presente trabajo, titulado \textbf{\myTitle}, ha sido realizado bajo su supervisión por \textbf{\myName}
y se autoriza la defensa de dicho trabajo ante el tribunal que corresponda.

\vspace{0.5cm}

Y para que conste, se expide el presente informe, en Granada, a %TODO - Fecha.

\vspace{3cm}

\begin{flushright}
 \begin{tabular}{m{6cm}m{6cm}}
     % TODO: poner firma real
     % \\ \hline
     \myProf & \myOtherProf \\
 \end{tabular}
\end{flushright}
%\cleardoublepage\include{FrontBackmatter/Foreword}
%\cleardoublepage\include{FrontBackmatter/Publications}
\cleardoublepage\include{FrontBackMatter/Acknowledgments}
\pagestyle{scrheadings}
\cleardoublepage\include{FrontBackMatter/Contents}
%********************************************************************
% Mainmatter
%*******************************************************
\cleardoublepage\pagenumbering{arabic}
%\setcounter{page}{90}
% use \cleardoublepage here to avoid problems with pdfbookmark
\cleardoublepage

\part{El problema isoperimétrico en el espacio euclídeo}

\chapter{Descripción}
\section{Resumen}

En este trabajo se abordará el problema isoperimétrico en $\rtres$. Se demostrará que dado un volumen fijo en $\rtres$, la superficie que encierra dicho volumen con área mínima es la esfera $\unitsphere$. Para llevarlo a cabo, se introducirá la teoría de integración en superficies, las técnicas variacionales y se terminará el trabajo probando que las esferas son las únicas superficies isoperimétricas mediante la prueba de Wente.

\paragraph{Palabras clave} Geometría diferencial global, problema isoperimétrico, superficies, curvatura media, esferas, superficie isoperimétricas, desigualdad de Brunn-Minkowski, fórmulas de variación para el área y el volumen

\newpage

\section{Introducción}
%TODO: Justificar WENTE sobre Alexandrov
\begin{otherlanguage}{american}
\pdfbookmark[1]{Abstract}{Abstract}
\section{Abstract}

This work studies the isoperimetric problem in $\rtres$. It prove that a  per given fixed volume in $\rtres$, sphere is the smallest surface area that enclosed this volume and furthemore, sphere is the unique. 

It introduces the surfaces integration theory where we will extend the properties for Lebesgue's integrations to a surfaces. This work shows us the tubular neighbourhood and the parallel surfaces to get the first variation of area and volumen. We use these variations to prove that sphere has mean curvature constant and are the unique isoperimetric surface.


\paragraph{Keywords} Global differential geometry, isoperimetric problem, surfaces, mean curvature, spheres, isoperimetric surfaces, Brunn-Minkowski inequality, first variation formula for area and volume

\end{otherlanguage}

\newpage


\chapter{Preliminares}
En este capítulo haremos un repaso de los conceptos más importantes que nos serán de utilidad a lo largo del trabajo. Recordaremos definiciones y resultados de superficies regular, enunciaremos y demostraremos el teorema de Brower-Samelson e introduciremos los entornos tubulares y superficies paralelas.

\section{Superficie regular}

\begin{definition}
Sea un subconjunto $S \subseteq \rtres$, $S \neq \emptyset$, es una \textit{superficie regular} si:

$\forall p \in S$ $\exists V$ entorno abierto de $p$ en $S$ (con la topología inducida de $\rtres$) y una aplicación $\xmath: \utortres$ con $\umath \subseteq \rdos$ abierto, verificando:

\begin{enumerate}
    \item $\xmath \in \cinf(\umath, \rtres)$.
    \item $\dxq: \rdostortres$ es inyectiva $\forall q \in \umath$.
    \item $\xmath(\umath) = \vmath$ y $\xmath:\utov$ es un homeomorfismo.
\end{enumerate}
\end{definition}

Sea $S \subseteq \rtres$ una superficie. Un \textbf{campo de vectores} (diferenciable) en $S$ es una aplicación diferenciable $\vmath: S \longrightarrow \rtres$. Si $\vmath_p := \vmath(p) \in \tmath_p S$ para todo $p \in S$, diremos que $\vmath$ es un \textbf{campo tangente} a $S$. Si $\vmath_p \bot \tmath_p S$ para todo $p \in S$, diremos que $\vmath$ es un \textbf{campo normal} a $S$. Un campo unitario es aquel que cumple $\parallel \vmath_p \parallel = 1$ para todo $p \in S$.

Se dice que $S$ es una \textbf{superficie orientable} si admite un campo normal unitario global $\nmath: S \longrightarrow \unitsphere$. A este campo $\nmath$ se le llama \textbf{aplicación de Gauss}.

Se llama \textbf{endomorfismo de Weingarten} de $S$ en $p$ al endomorfismo: $A_p = -(dN)_p$

\begin{definition}[Formas fundamentales]
La \textbf{primera forma fundamental} de $S$ en $p$ es: $I_p: T_pS \times T_pS \longrightarrow \rmath$ donde $I_p(u,v) =  \langle u,v \rangle $, $\forall u,v \in T_pS$.

La \textbf{segunda forma fundamental} de $S$ en $p$ es la forma bilineal: $II_p = \sigma_p: T_pS \times T_pS \longrightarrow \rmath$ donde $\sigma_p(u,v) =  \langle A_pu,v \rangle  = I_p(Ap_u,v) = - \langle (dN)_p(u), v \rangle $, $\forall u,v \in T_pS$.
\end{definition}

Notemos que $\sigma_p$ es simétrica y por tanto el endomorfismo de Weingarten ($A_p$) es autoadjunto.

Cómo $A_p$ es autoadjunto, entonces es diagonalizable mediante una base ortonormal y por tanto tiene valores propios reales, es decir, $\exists k_1(p), k_2(p) \in \rmath, k_1(p) \leq k_2(p)$, $\exists {e_1,e_2}$ base ortonormal en $(T_pS, I_p)$ de forma que $A_p(e_i) = k_i(p)e_i$, $\forall i = 1,2$

Los números $k_i(p)$ se llaman \textbf{curvaturas principales} de $S$ en $p$.
Los vectores propios, no nulos, de $A_p$ se llaman \textbf{direcciones principales} de $S$ en $p$.
Se define la \textbf{curvatura de Gauss} de $S$ en $p$ como el número real $K(p)=det A_p=k_1(p)k_2(p)$
Se define la \textbf{curvatura media} de $S$ en $p$ como el número real $H(p)=\frac{1}{2}tr A_p=\frac{k_1(p)+k_2(p)}{2}$
Se dice que $S$ es \textbf{una superficie llana} si $K(p)=0$, $\forall p \in S$

\begin{definition}[Superficie minimal y totalmente umbilical]
Se dice que $S$ es \textbf{minimal} si $H(p)=0$, $\forall p \in S$

Un punto es \textbf{umbilical} si $k_1(p)=k_2(p)$.

Sea $S \subseteq \rtres$, se dice que $S$ es \textbf{totalmente umbilical} si todos son puntos son umbilicales.
\end{definition}

\begin{theorem}[Clasificación de las superficies totalmente umbilicales]\label{umbilicaltheorem}
Sea $S \subseteq \rtres$ superficie conexa, cerrada y totalmente umbilical. Entonces es un plano o una esfera.
\end{theorem}

La demostración de este teorema se basa en la que la conexión nos impide que salgan uniones de varios planos y esferas y el cierre impide que S sea un abierto de plano o esfera. No lo vamos a demostrar.

\begin{theorem}[Teorema de Hilbert-Liebmann]
Sea $S \subseteq \rtres$ una superficie compacta y conexa con curvatura de Gauss $K$ constante, entonces S es una esfera.
\end{theorem}

Veamos ahora una generalización del teorema de Hilbert-Liebmann para superficies cerradas.
\begin{theorem}[Teorema de Bonnet]
Sea $S \subseteq \rtres$ una superficie cerrada y conexa con curvatura de Gauss $K=c > 0$, entonces S es una esfera.
\end{theorem}


\section{Teorema de Brower-Samelson}

En esta sección vamos a dar los preliminares necesarios para la demostración del Teorema de Brower-Samelson. Comenzaremos con el teore de Jordan-Brower, que nos permitirá hablar del volumen encerrado por una superficie compacta y terminaremos el capítulo con la definición y propiedades de los entornos tubulares.

Cómo ya sabemos por el Teorema de Jordan clásico en el caso de $\rdos$, una curva cerrada y simple, delimita el plano en dos superficies conexas, una de ellas acotada. Este teorema, nos va a permitir tener tener una extensión de este resultado para el caso de $\rtres$. De hecho, esta generalización, que no veremos porque se escapa del objetivo de este trabajo, es válida para todo $\rmath^n$ con un hiperplano suyo, puede verse en \cite{paperchicago}. En este trabajo, demostraremos que es cierto para n=3.

\begin{theorem}[Teorema de separación de Jordan-Brower]
Sea $S \subseteq \rtres$ superficie compacta y conexa. Entonces $\rtres - S$ tiene exactamente dos componentes conexas cuya frontera común es $S$.
\end{theorem}

Aprovechamos este teorema para definir los dominios interior y exterior delimitados por una superficie.

\begin{definition}[Dominios interior y exterior]
Llamamos \textbf{dominio interior} y notamos como $\Omega$ a la componente conexa acotada limitada por la superficie $S$. Llamamos \textbf{dominio exterior} a la componente no acotada $\rtres - \Omega=\Omega_{*}$.
\end{definition}

\begin{lemma}
Sea S una superficie y $p \in S$. $\exists W \subset \rtres$ entorno abierto y conexo de $p$ y $\exists G: W \longrightarrow B$ difeomorfismo que cumple $G(W\cap S) = B\cap P$ con $B$ una bola abierta y $P$ un plano, ambos de $\rtres$.
\end{lemma}

\begin{definition}
Sea $S$ superficie conexa y compacta y sea $v \in (T_pS)^{\perp}$ con $\lVert v\rVert=1$. Consideremos la recta afín $\alpha: \mathbb{R} \longrightarrow \rtres$ definida como $\alpha(t) = p + tv$ con $p \in S$. Entonces, $\alpha$ es regular con $\alpha(0)=p$ y $\alpha'(0) = v$. Existe entonces un $\epsilon  >  0$ tal que $\alpha(-\epsilon, \epsilon) \cap S = {p}$. Así, los conexos $\alpha(-\epsilon, 0)$ y $\alpha(0, \epsilon)$ están contenidos en $\rtres - S$. Diremos que $v$ es \textbf{interior} si $\alpha(0, \epsilon) \subset \Omega$, en caso contrario diremos que es \textbf{exterior}.
\end{definition}

Con estos preliminares, estamos en condiciones de demostrar el teorema de Brower-Samelson.

\begin{theorem}[Teorema de Brower-Samelson]\label{browersamelson}
Toda superficie compacta en un espacio euclideo es orientable.
\end{theorem}
\begin{proof}
La idea de esta demostración es encontrar la orientación de esta superficie $S$ compacta. Para ello comencemos viendo que para $v \in (T_pS)^{\bot}$ con $\lVert v \rVert=1$ tenemos que $v$ es interior o bien $-v$ es interior.

Supongamos que $v$ no es interior, luego por definición, $\alpha(\epsilon, 0) \subset \Omega^{*}$ con $\alpha$ la recta afín considerada en la definición previa. Consideremos ahora la recta afín $\beta: \mathbb{R} \longrightarrow \rtres$ dada por $\beta(t)=p + t(-v)$ con $p \in S$. Luego $\beta(0, \epsilon) = \alpha(-\epsilon, 0) \subset \Omega$ y por tanto $-v$ es interior. Ya hemos probado que si $v$ no es interior, lo es $-v$, veamos ahora que no pueden serlo ambos.

Supongamos que $v$ y $-v$ son interiores. Esto implicaría que $\alpha(0,\epsilon) \subset \Omega$ y $\beta(0, \epsilon) = \alpha(-\epsilon, 0) \subset \Omega$, y por definición, $\alpha(-\epsilon, \epsilon) \cap S \neq \emptyset$, luego llegamos a contradicción y no pueden ser ambos interiores.

Tomemos ahora $v \in (T_pS)^{\bot}$ con $\lVert v \rVert=1$ interior (si no fuese interior, tomamos $-v$). Vamos a buscar un entorno abierto de $p \in S$ para construir la orientación $N: V \longrightarrow \rtres$. Donde $N$ es un campo unitario normal diferenciable con $N(q)$ interior $\forall q \in V.$

Por el lema previo, tenemos que $\exists W$ entorno de abierto y conexo de $p$ y $G: W \longrightarrow B$ difeomorfismo. Definimos el entorno $V_1=W\cap S$ entorno de $p$, homeomorfo a $B\cap P$, luego conexo. Además, podemos tomar un entorno coordenado, por ser superficie localmente orientable, $V_2$ de $p$ de forma que $V=V_1 \cap V_2$ es orientable.

Sea $N: V \longrightarrow \rtres$ la orientación de $V$ tal que $N(p)$ es interior. Vamos a denotar como $\tilde{\Omega}$ a la componente conexa de $B-P$ tal que $G(W\cap P) = \tilde{\Omega}$ y sea $n \in \rtres$ al normal unitario que apunta hacia $\tilde{\Omega}$.

Definimos la función $f: V \longrightarrow \rmath$ como $f(q) = \langle (dG)_q(N(q)), n \rangle = \langle \beta'_q(0), n \rangle$, donde $\beta_q(t) = G(q + tN(q))$.

Como $N(q) \not\in T_qS$ y los difeomorfismos conservan la transversalidad se tiene que $\beta'_q(0) \not\in T_{G(p)}P = n^\perp$. En particular, $f(q)$ nunca se anula en $V$. Por continuidad y conexión tenemos tenemos que se conserva el signo:

\begin{equation*}
    signo(f(q)) = signo(f(p)) = signo(\langle (dG)_q(N(q)), n \rangle) = signo(\langle \beta'_q(0), n \rangle)
\end{equation*}

Veamos que este signo es positivo. Tenemos que $\alpha_p(0, \epsilon_p) \subset \Omega$ donde $\alpha_q(t) = q + tN(q)$, luego $\beta_p(0, \epsilon_p) \subset \tilde{\Omega}$ y por tanto, $\langle \beta_p(t), n \rangle > 0$, para todo $t \in (0, \epsilon_p)$ y $\langle \beta_p(0), n \rangle = 0$. Derivando:

\begin{equation*}
    \diff{}{t}{t=0} \langle \beta_p(t), n \rangle \geq 0 \Rightarrow \langle \beta'_p(t), n \rangle \geq 0
\end{equation*}

como $f$ nunca se anula:

\begin{equation*}
    \langle \beta'_p(t), n \rangle > 0
\end{equation*}

Ya hemos probado que el signo es positivo. Veamos que $f(q) > 0$, $\forall q \in V$ implica que $N(q)$, $\forall q \in V$. Hagámoslo por reducción al absurdo.

Supongamos que no ocurre, entonces $\exists q \in V$ tal que $N(q)$ no es interior. Entonces se tiene que $\alpha_q(0, \epsilon_q) \subset \Omega_*$ y, por tanto, $\beta_q(0, \epsilon_q) \subset \tilde{\Omega}_*$ y tendríamos que $\langle \beta_q(t), n \rangle < 0$, para todo $t \in (0, \epsilon_q)$ y $\langle \beta_q(0), n \rangle = 0$. Por tanto, análogo al razonamiento anterior, podríamos obtener que $f(q) = \langle \beta'_q(0), n \rangle < 0$ y esto es una contradicción.

\end{proof}

\section{Entornos tubulares}

En este sección, vamos a definir los entornos tubulares. Veremos que dada una superficie, bajo determinadas hipótesis, existe un entorno que envuelve la superficie.

Sea $S$ una superficie de $\rtres$, como $\rtres$ es un espacio métrico, los entornos más sencillos de definir son los conocidos como \textbf{entornos métricos}, definidos como los puntos cuya distancia a la superficie es menor que un delta dado, notamos:

\begin{equation*}
    B_\delta(S)=\{p\in \rtres | dist(p,S) < \delta\}
\end{equation*}

donde, $dist(p,S) = inf_{q\in S}|p-q|$.

\begin{lemma}
Sea $S$ una superficie cerrada de $\rtres$, el conjunto $B_\delta(S)$ definido anteriormente, coincide con el conjunto $N_\delta(S)=\bigcup_{p\in S}N_\delta(p)$, definido como el los segmentos abiertos en las normales a la superficie S con centro p ($p \in S$) y radio $\delta$.
\end{lemma}
\begin{proof}
Veámoslo por doble inclusión:

Sea $p \in S$, y sea $q \in N_\delta(p)$ para el $p$ dado y $\delta > 0$. Es directo que, $dist(q,S) \leq |q-p| < \delta$, luego $q \in B_\delta(S)$.

Supongamos ahora $q \in B_\delta(S)$. Tomamos la función distancia del punto $q$ a $S$. Como $S$ es cerrado, sabemos que existe un mínimo en un punto $p \in S$, además por la caracterización de los punto críticos de la función distancia al cuadrado, sabemos que el punto $q$ está en la recta normal a $S$ en el punto $p$. Así, $|p-q| = dist(q,S) < \delta$, luego $q \in N_\delta(p)$.
\end{proof}

Sea $S$ una superficie orientable con la aplicación de Gauss $N: S \longrightarrow \mathbb{S}^2 \subset \rtres$, definimos:

\begin{align*}
    F: S \times \mathbb{R} &\longrightarrow \rtres \\
    (p,t) &\longrightarrow p + tN(p)
\end{align*}

Esta aplicación, claramente diferenciable, envía cada punto $p$ de la superficie a la distancia $t$ en dirección de la recta normal a la superficie en el punto $p$. Luego tenemos:

\begin{equation*}
    F(S \times (-\delta, \delta)) = N_\delta(S)=\bigcup_{p\in S} N_\delta(p), \qquad \forall \delta > 0
\end{equation*}

\begin{definition}[Entornos tubulares]
La unión $N_\delta(S)$ de todos los segmentos normales de radio $\delta > 0$ centrados en los puntos de una superficie $S$ orientable es llamada \textbf{entorno tubular} de radio $\delta$ si es un abierto como subconjunto de $\rtres$ y la función $F: S \times (-\delta, \delta) \longrightarrow N(p)$ definida previamente es un difeomorfismo.
\end{definition}

\begin{lemma}
Sea $S$ una superficie, $\forall p \in S$, $\exists V_p$, entorno abierto y orientable y $\delta > 0$ de forma que el conjunto $N_\delta(V_p)$ es un entorno tubular de $V$.
\end{lemma}
\begin{proof}
Vamso a empezar la demostración calculando la apliación diferencial de $F$ en un punto $(p,t) \in S \times \rmath$:

\begin{align*}
    (dF)_{(p,t)}(v,0) &= \diff{}{s}{s=0} F(\alpha(s), t) \\
    &= \diff{}{s}{s=0} (\alpha(s) + tN(\alpha(s)) \\
    &= \alpha'(0) + t(dN)_{\alpha(0)}(\alpha'(0)) \\
    &= v + t(dN)_p(v)
\end{align*}

donde $\alpha: (-\epsilon, \epsilon) \longrightarrow S$ es una curva con $\alpha(0) = p$ y $\alpha'(0) = v$.

Por otro lado, tenemos:

\begin{align*}
    (dF)_{(p,t)}(0,1) &= \diff{}{s}{s=0} F(p, t+s) \\
    &= \diff{}{s}{s=0} (p + (t+s)N(p)) \\ &= N(p)
\end{align*}

En particular, para $t=0$, tenemos:

\begin{align*}
    (dF)_{(p,0)}(v,0) &= v \\
    (dF)_{(p,0)}(0,1) &= N(p)
\end{align*}

Por tanto, $(dF)_{(p,0)}$ es un isomorfismo. Si tomamos una base $\{v_1,v_2\}$ de $T_pS$, $(dF)_{(p,0)}$ transforma la base $\{(v_1,0),(v_2,0),(0,1)\}$ de $T_pS \times \rmath$ en una base de $\rtres$, $\{v_1,v_2,N(p)\}$.

Concluimos utilizando el teorema de la función inversa, que nos asegura que existe $V_p$ entorno abierto de $p$ en $S$ y $\delta_p > 0$, tal que, $F: V_p \times (-\delta_p, \delta_p) \longrightarrow F(V_p \times (-\delta_p, \delta_p))$ es un difeomorfismo.
\end{proof}

Veamos ahora la existencia de entornos tubulares para superficies compactas, utilizando el lema previo para demostrarlo.

\begin{theorem}[Existencia de entornos tubulares]
Sea $S$ una superficie orientable y $R \subset S$ un subconjunto abierto relativamente compacto. Entonces $\exists \epsilon > 0$ tal que, el conjunto $N_\epsilon(R)$ es un entorno tubular de la superficie $R$, esto es, es un abierto de $\rtres$ y la función:

\begin{align*}
    F: S \times (-\epsilon, \epsilon) &\longrightarrow N_\epsilon(R) \\
    (p,t) &\longrightarrow p + tN(p)
\end{align*}

es un difeomorfismo.

En particular, cuando la superficie $S$ es compacta, $\exists \epsilon  \rangle  0$ tal que
$B_\epsilon(S)=N_\epsilon(S)$ es un entorno tubular de $S$.
\end{theorem}
\begin{proof}
Como tenemos que $R$ es relativamente compacto, esto es, $\overline{R}$ es compacto, existe un recubrimiento finito por abierto de $\overline{R}$ cada uno con un entorno tubular, por el lema previo. Sea $\delta > 0$ el menor radio de todos ellos. Si tomamos la función $F$ definida previamente, restringida al intervalo de definición $R \times (-\delta, \delta)$ es un difeomorfismo local. Vamos a buscar un $\epsilon \in (0,\delta)$ tal que la función $F$ restringida al intervalo $R \times (-\epsilon, \epsilon)$ es inyectiva. Veámoslo por reducción al absurdo.

Supongamos que $\exists \epsilon \in (0,\delta)$ tal que $F$ restringida al intervado de definición $Rx(-\epsilon, \epsilon)$ no es inyectiva, o lo que es lo mismo, los segmentos $N_\epsilon(p)$ de normales con $p \in R$ intersecan entre ellos. Tomamos $\epsilon=\frac{1}{n}$ y $\forall p \in \mathbb{N}$, $\exists p_n,q_n \in S$ con $p_n \neq q_n$ tal que $N_{\frac{1}{n}}(p_n) \bigcap N_{\frac{1}{n}}(q_n) \neq \emptyset$.
Como $R$ es relativamente compacto, podemos tomar sucesiones parciales que convergen en $\overline{R}$. Sean $\{p_n\}_{n \in \mathbb{N}}, \{q_n\}_{n \in \mathbb{N}}$ estas sucesiones y sean $p,q \in \overline{R}$ los puntos donde convergen, esto es:

\begin{equation*}
    \lim_{n\to\infty} p_n = p \qquad \lim_{n\to\infty} q_n = q
\end{equation*}

Sea $r_n \in N_{\frac{1}{n}}(p_n) \bigcap N_{\frac{1}{n}}(q_n)$, entonces:

\begin{equation*}
    |p_n - q_n| = |p_n - q_n + r_n - r_n| \leq |p_n-r_n| + |r_n-q_n| < \frac{1}{n}+\frac{1}{n} = \frac{2}{n}
\end{equation*}

Por tanto los límites coinciden.

Aplicando el lema previo al punto $p = q \in S$, tenemos $V$ y $\rho > 0$ tal que $N_\rho(V)$ es un entorno tubular. Además, $\exists N_0 \in \mathbb{N}$ tal que, $\forall n > N_0$, $p_n,q_n \in V$ y $1/n < \rho$. Por tanto llegamos a una contradicción:

\begin{equation*}
    N_{\frac{1}{n}}(p_n) \bigcap N_{\frac{1}{n}}(q_n) \subset N_\rho(p_n)\bigcap N_\rho(q_n) = \emptyset
\end{equation*}

ya que $N_\rho(V)$ es un entorno tubular. Por tanto, $F$ restringida al intervalo de definición $V \times (-\rho, \rho)$ es inyectiva y localmente difeomorfismo.

\end{proof}

A partir de los entornos tubulares, introduciremos el concepto de superficie paralela para concluir el capítulo.

\begin{definition}[Superficie paralela]
Sea $\rho > 0$ y sea $N_\rho(S)$ un entorno tubular de la superficie $S$ compacta con $N$ su aplicación de Gauss. Para todo $t \in (-\rho, \rho)$ se define el conjunto $S_t=\{p + tN(p); p \in S\}$ superficie compacta y la aplicación $F_t: S \longrightarrow S_t$ dada por $F_t(p)=p+tN(p)$ es un difeomorfismo.
Llamamos a $S_t$ \textbf{superficie paralela} a $S$ a una distancia $t$.
\end{definition}


\chapter{Integración en superficies}
\label{chapter:surfacesintegration}
En este capítulo introduciremos las herramientas de integración en superficies. Vamos a ver cómo los conceptos que conocemos para las integrales en el sentido de Lebesgue son extensibles a superficies. Enunciaremos y probaremos algunos de los resultados más utilizados en la teoría de integración como el teorema de cambio de variable o el teorema de Fubini. Por último, estudiaremos los teoremas de la divergencia que aplicaremos en capítulos posteriores.

\section{Integración en Superficies y fórmulas de integración}

Comenzamos esta sección con la definición integral para funciones sobre superficies.

\begin{definition}[Valor absoluto del Jacobiano]
Sea $\Phi: S \longrightarrow S'$ un difeomorfismo entre dos superficies. Se define el \textbf{valor absoluto del jacobiano de $\Phi$} como la aplicación
\begin{align*}
    |\text{Jac} \, \Phi|: S &\longrightarrow \rmath \\
    p &\longrightarrow |det(d\Phi)_p| = |det(M_p)|
\end{align*}
%
donde $M_p=M((d\Phi)_p, (B_1)_p, (B_2)_{\Phi(p)})$, siendo $(B_1)_p$ base ortonormal de $T_pS$ y $(B_2)_{\Phi(p)}$ base ortonormal de $T_{\Phi(p)}S'$. Esto no depende de las bases elegidas.
\end{definition}

\begin{definition}[Valor absoluto del Jacobiano para difeomorfismos en $S \times I$]
Sea $S$ una superficie, $I \subseteq \rmath$ un intervalo, $O$ abierto de $\rtres$ y sea $\Phi: S \times I \longrightarrow O$ un difeomorfismo. Se define el \textit{valor absoluto del jacobiano de $\Phi$} como la aplicación
\begin{align*}
    |\text{Jac} \, \Phi|: S \times I &\longrightarrow O \\
    (p,t) &\longrightarrow |det(d\Phi)_{(p,t)}| = |det(M_{(p,t)})|
\end{align*}
%
donde $M_{(p,t)}=M((d\Phi)_{(p,t)}, (B_1)_{(p,t)}, (B_2)_{\Phi(p,t)})$, siendo $(B_1)_{(p,t)}$ base ortonormal de $T_pS \times \rmath$ y $(B_2)_{\Phi(p,t)}$ base ortonormal de $\rtres$.
\end{definition}

\begin{definition}[Integral en $S \times \rmath$]
Sea $S$ una superficie y sea $O$ un subconjunto abierto de $\rtres$, y sea el difeomorfismo $\phi: O \longrightarrow R \times (a,b)$ con $a < b$ y $R \subset S$ un abierto relativamente compacto. Decimos que la función $h: R \times (a,b) \longrightarrow \rmath$ es integrable cuando $(h \circ \phi)|\text{Jac} \, \phi|$ sea integrable en $O$ en el sentido de Lebesgue.

En ese caso, llamaremos \textbf{integral de $h$ en $R \times (a,b)$} al número real dado por:

\begin{equation*}
    \int_{R \times (a,b)} h = \int_{R \times (a,b)} h(p, t) \, dp dt = \int_{O} (h \circ \phi)|\text{Jac} \, \phi|
\end{equation*}
\end{definition}

Esta definición es correcta, en el sentido de que no depende del difeomorfismo $\Phi: O \longrightarrow R \times (0,1)$.

\begin{definition}[Integral en $S$]
Sea $R$ una región (abierto relativamente compacto) de una superficie orientable $S$, y sea $f:R \longrightarrow \rmath$ una función cualquiera. Diremos que $f$ es integrable en $R$ cuando la función $(p,t) \in R \times (0,1) \mapsto f(p)$ sea integrable en $R \times (0,1)$. Llamaremos \textbf{integral de $f$ en $R$} al número real dado por:

\begin{equation*}
    \int_{R} f = \int_{R \times (0,1)} f(p,t) \, dp dt.
\end{equation*}
\end{definition}

\begin{definition}[Área de una superficie compacta]
Sea $S$ una superficie compacta, se define su área como:

\begin{equation*}
    A(S) = \int_S 1
\end{equation*}
\end{definition}

Algunos resultados de integración que necesitaremos son los siguientes. Se puede consultar su prueba en el capítulo 5 de \cite{montielrosbook}.

\begin{theorem}[Teorema del cambio de variable]
Sea $\phi:R_1 \longrightarrow R_2$ un difeomorfismo entre dos regiones de dos superficies orientables, y sea $f: R_2 \longrightarrow \rmath$ una función integrable. Entonces la función $(f \circ \phi)|\text{Jac} \, \phi|$ es integrable en $R_1$ y además:

\begin{equation*}
    \int_{R_2} f = \int_{R_1} (f \circ \phi)|\text{Jac} \, \phi|
\end{equation*}
\end{theorem}

\begin{theorem}[Teorema de Fubini]
Sea $R$ una región de una superficie orientable, $a,b \in \rmath, a < b$ y $h:R \times (a,b) \longrightarrow \rmath$ una función integrable en $R \times (a,b)$. Entonces, para casi todo $t \in (a,b)$, la función $p \in R \mapsto h(p,t)$ es integrable en $R$, y para casi todo $p \in R$ la función $t \in (a,b) \mapsto h(p,t)$ es integrable en $(a,b)$. Además, las funciones

\begin{equation*}
    p \in R \mapsto \int_a^b h(p,t), \quad  t \in (a,b) \mapsto \int_S h(p,t) dp
\end{equation*}
%
son integrables en $R$ y $(a,b)$ respectivamente. Finalmente, tenemos que:

\begin{equation*}
    \int_{R \times (a,b)} h(p,t) \, dpdt = \int_R \left( \int_a^b h(p,t) \, dt \right) \, dp = \int_a^b \left( \int_R h(p,t) \, dp \right) dt.
\end{equation*}
\end{theorem}


\section{Teoremas de la divergencia}
En esta sección vamos a enunciar el teorema de la divergencia y algunas consecuencias directas de este que, de nuevo, necesitaremos para demostrar algunos de los resultados más importantes en los capítulos finales.

\begin{definition}[Campo vectorial diferenciable]
Dado $A$ un subconjunto de $\rtres$, se define un \textbf{campo vectorial diferenciable} como una aplicación diferenciable $X: A \longrightarrow \rtres$.
\end{definition}

\begin{definition}[Divergencia de un campo vectorial diferenciable]
Se define la \textbf{divergencia de X} como la función $div \, X: A \longrightarrow \rmath$ dada por 
%
\begin{equation*}
    (div \, X)(p) = tr (dX)_p
\end{equation*}
%
con $p \in A$. Si $X$ se expresa como $X(p) = (X_1(p), X_2(p), X_3(p))$, entonces
%
\begin{equation*}
    (div \, X)(p) = \frac{\partial X_1}{\partial x}(p) + \frac{\partial X_2}{\partial y}(p) + \frac{\partial X_3}{\partial z}(p)
\end{equation*}
%
\end{definition}

\begin{theorem}[Teorema de la divergencia clásico]\label{divergencetheorem}
Sea $S$ una superficie conexa y compacta, y $\Omega$ el dominio interior determinado por $S$. Si $X: \overline{\Omega} \longrightarrow \rtres$ es un campo vectorial diferenciable, entonces:
%
\begin{equation*}
    \int_\Omega div \, X = -\int_S  \langle X,N \rangle
\end{equation*}
%
donde $N: S \longrightarrow \mathbb{S}^2$ es el campo unitario normal interior a lo largo de $S$.
\end{theorem}

Se puede consultar una demostración en la sección 5.7 de \cite{montielrosbook}.

\begin{definition}[Volumen encerrado por una superficie compacta]\label{volumensuperficiecompacta}
Sea $S$ una superficie compacta y $\Omega$ su dominio interior. Sea el campo vectorial diferenciable identidad $X: \overline{\Omega} \longrightarrow \rtres$ definido por $X(x) = x, \quad \forall p \in \overline{\Omega}$. Su divergencia es la función constante 3. Entonces por el teorema de divergencia:
%
\begin{equation*}
    vol \, \Omega = - \frac{1}{3} \int_S  \langle p, N(p) \rangle  dp,
\end{equation*}
%
donde $N$ es el normal unitario interior de $S$.
\end{definition}

\begin{definition}[Campo diferenciable sobre una superficie y campo diferenciable tangente]
    Sea $S$ una superficie de $\rtres$, un \textbf{campo diferenciable sobre $S$} es una aplicación diferenciable $V: S \longrightarrow \rtres$. Además, si $V(p) \in T_pS$ para cada $p \in S$ diremos que es un \textbf{campo diferenciable tangente}.
\end{definition}

\begin{definition}[Divergencia de un campo vectorial sobre una superficie]
    Sea $S$ una superficie compacta, y sea $V: S \longrightarrow \rtres$ un campo diferenciable y tangente sobre $S$. Se define la \textbf{divergencia de $V$} para $p \in S$ y $\{e_1,e_2\}$ base ortonormal de $T_pS$ como la aplicación $div_S \, V: S \longrightarrow \rmath$ definida por $(div_S \, V)(p) := \langle (dV)_p(e_1),e_1 \rangle + \langle (dV)_p(e_2),e_2 \rangle$. Esta definición no depende de la base ortonormal elegida.
\end{definition}

\begin{theorem}[Teorema de divergencia en superficies]\label{divergesurfaces}
Sea $S$ una superficie compacta y $V: S \longrightarrow \rtres$ un campo vectorial diferenciable en $S$. Entonces, tenemos:
%
\begin{equation*}
    \int_S div_S \, V = -2 \int_S \langle V,N \rangle H,
\end{equation*}
%
donde $N$ es el normal interior sobre $S$ y $H$ es la curvatura media de $S$ calculada con respecto a $N$.
\end{theorem}

Una demostración se puede encontrar en la sección 6.3 de \cite{montielrosbook}.

\begin{definition}[Gradiente]
Sea $f$ una función diferenciable sobre una superficie $S$ y $p \in S$. Consideremos la diferencial $(df)_p: T_pS \longrightarrow \rmath$. Entonces por teoría de espacios vectoriales euclídeos, existe un único $x \in T_pS$ tal que $(df)_p(v)=  \langle x,v \rangle $, $\forall v \in T_pS$.

Denotaremos $x= \gradientef $ y lo llamaremos \textbf{gradiente de $f$ en p}. Así:
%
\begin{equation*}
    \langle \gradientef, v \rangle = (df)_p(v), \qquad \forall v \in T_pS.
\end{equation*}
%
En particular, si $\{e_1, e_2\}$ es una base ortonormal se tiene:
%
\begin{equation*}
     \gradientef = \langle  \gradientef , e_1 \rangle e_1 + \langle \gradientef, e_2 \rangle e_2 = (df)_p(e_1) \, e_1 + (df)_p(e_2) \, e_2
\end{equation*}
\end{definition}

\begin{corolario}\label{corolariogradiente}
Si $f: S \longrightarrow \rmath$ es una función diferenciable sobre una superficie compacta, entonces:
%
\begin{equation*}
    \int_S \langle \gradientef, p \rangle dS + 2 \int_S f(p)dS = -2 \int_S f(p)H(p) \langle p, N(p) \rangle dS
\end{equation*}
\end{corolario}
\begin{proof}
Consideramos sobre $S$ el campo diferenciable dado por $V(p)=f(p)p$. Por la definición de divergencia, tenemos:
%
\begin{align*}
    (div_S V)(p) &= \sum_{i=1}^2 \langle (dV)_p(e_i), e_i \rangle \\
    &= \sum_{i=1}^2 (df)_p(e_i) \langle p, e_i \rangle + 2f(p),
\end{align*}
%
donde $\{e_1,e_2\}$ es cualquier base ortonormal en $T_pS$.

Por otro lado, calculando el producto escalar de $p$ con el gradiente de $f$ en $p$:
%
\begin{equation*}
    \langle p, \gradientef \rangle = \langle p, (df)_p(e_1) \, e_1 \rangle + \langle (df)_p(e_2) \, e_2 \rangle = \sum_{i=1}^2 (df)_p(e_i) \langle p, e_i \rangle 
\end{equation*}
%
y por tanto:
%
\begin{equation*}
    (div_S V)(p) = \langle \gradientef, p \rangle + 2f(p)
\end{equation*}

La demostración concluye sin más que aplicar el teorema de la divergencia para superficies \ref{divergesurfaces} al campo $V$.
\end{proof}

\section{Algunos cálculos necesarios}

Vamos calcular a partir de lo obtenido previamente el área de una esfera, el volumen de una bola, y el área y volumen encerrado por una superficie paralela.

\begin{remark}[Área de una esfera]
Veamos cuál es el área de una esfera $\mathbb{S}^2(p_0, r)$. Sabemos que:

\begin{equation*}
    A(\mathbb{S}^2(p_0, r)) = \int_{\mathbb{S}^2(p_0, r)} 1
\end{equation*}

Consideremos la traslación $T: \mathbb{S}^2(0, r) \longrightarrow \mathbb{S}^2(p_0, r)$ tal que $T(p) = p + p_0$. Usando la fórmula del cambio de variable, tenemos que:

\begin{equation*}
    \int_{\mathbb{S}^2(p_0, r)} 1 = \int_{\mathbb{S}^2(0, r)} |\text{Jac} \, T|.
\end{equation*}

Además, $|\text{Jac} \, T|=1$ por ser $T$ un movimiento rígido, luego:

\begin{equation*}
    \int_{\mathbb{S}^2(0, r)} |\text{Jac} \, T| = \int_{\mathbb{S}^2(0, r)} 1 = A(\mathbb{S}^2(0, r))
\end{equation*}

Para calcular esta integral, consideramos una parametrización de la esfera $\unitsphere (0,r)$ como una superficie de revolución: 
%
\begin{align*}
    X: \left( \frac{-\pi}{2}, \frac{\pi}{2} \right) \times (-\pi, \pi) &\longrightarrow \mathbb{S}^2(0, r) - C \\
    (t, \theta) &\longrightarrow (r\cos t\cos\theta, r\cos t\sen\theta, r\sen t)
\end{align*}
%
con $C = \{X(t,\pi) \enspace | \enspace t\in [-\pi/2, \pi/2]\}$. Por tanto, $\mathbb{S}^2(0, r) - C = X \big( (-\pi/2, \pi/2) \times (-\pi, \pi) \big)$. Como $C$ es un conjunto de medida nula, tenemos que:
%
\begin{equation*}
    \int_{\unitsphere(0, r)} 1 = \int_{\unitsphere(0, r) - C} 1.
\end{equation*}
%
Utilizando de nuevo el cambio de variable:
%
\begin{equation*}
    \int_{\unitsphere(0, r) - C} 1 = \int_{ \left( \frac{-\pi}{2}, \frac{\pi}{2} \right) \times (-\pi, \pi)} |\text{Jac} \, X| \, dtd\theta.
\end{equation*}

Calculemos el jacobiano de $X$ a partir de la definición, sea $(dX)_{(t, \theta)}: \rdos \longrightarrow T_{X(t,\theta)} \mathbb{S}^2$, tomamos la base usual en $\rdos$ y tenemos:
%
\begin{equation*}
    (dX)_{(t, \theta)}(1,0) = X_t(t, \theta) \qquad (dX)_{(t, \theta)}(0,1) = X_\theta(t, \theta).
\end{equation*}

Calculamos las derivadas y obtenemos:
%
\begin{align*}
    X_t(t, \theta) &= (-r\sen t\cos\theta, -r\sen t\sen\theta, r\cos t), \\
    X_\theta(t, \theta) &= (-r\cos t\sen\theta, r\cos t\cos\theta, 0),
\end{align*}
%
y por tanto, $\{ X_t(t, \theta), X_\theta(t, \theta) \}$ es base de $T_{X(t,\theta)} \mathbb{S}^2$, además $\langle X_t(t, \theta), X_\theta(t, \theta) \rangle = 0$ luego es una base ortogonal y solo nos falta dividir por su norma para obtener la base ortonormal:
%
\begin{align*}
    |X_t|^2 &= r^2 \\
    |X_\theta|^2 &= r^2\cos^2t
\end{align*}
%
por tanto, $|\text{Jac} \, X|(t,\theta) = |det \, M(t,\theta)|$  donde $M(t,\theta)$ es la matriz de $(dX)_{(t, \theta)}$ en las bases $\{e_1, e_2\}$ y $\left\{ \frac{X_t}{r}, \frac{X_\theta}{r\cos t} \right\}$.
%
\begin{equation*}
    M(t,\theta) = \left( {\begin{array}{cc}
        r & 0 \\
        0 & r\cos t
    \end{array} } \right)
\end{equation*}
%
Luego:
%
\begin{equation*}
    |\text{Jac} \, X|(t,\theta) = r^2\cos t
\end{equation*}
%
y, por tanto:
%
\begin{equation*}
    A(\mathbb{S}^2(0,r)) = \int_{(\frac{-\pi}{2}, \frac{\pi}{2}) \times (-\pi, \pi)} r^2\cos t \, dtd\theta = 4\pi r^2 
\end{equation*}
%
Hemos probado que:
\begin{equation*}
    A(\mathbb{S}^2(p_0,r)) = 4\pi r^2 
\end{equation*}
\end{remark}

\begin{remark}[Volumen de una bola]
Vamos a calcular el volumen de Lebesgue de una bola euclídea $B(p_0,r)=\{ p \in \rtres |p-p_0| < r \}$. Como en el cálculo del área de la esfera, consideramos la traslación, $T: B(0, r) \longrightarrow B(p_0, r)$ tal que $T(p) = p + p_0$. Aplicando la fórmula del cambio de variable y sabiendo que $|\text{Jac} \, T|=1$, tenemos:
%
\begin{equation*}
    vol \, B(p_0,r) = \int_{B(p_0,r)} 1 = \int_{B(0,r)} |\text{Jac} \, T| = \int_{B(0,r)} 1 = vol \, B(0,r).
\end{equation*}

Utilizando el teorema de la divergencia clásico \ref{divergencetheorem} con $S=\unitsphere(0,r)$ y $\Omega=B(0,r)$ y $X(p)=p$, tenemos que $(div \, X)(p)=3$, con $\forall t\in \rtres$, y nos queda:
%
\begin{align*}
    vol \, B(0,r) &= \frac{-1}{3} \int_{\mathbb{S}^2(0,r)}  \langle p, \frac{-p}{r} \rangle = \frac{-1}{3r} \int_{\mathbb{S}^2(0,r)}  \langle p, -p \rangle \\
    &= \frac{r}{3} \int_{\mathbb{S}^2(0,r)} 1
\end{align*}
%
ya que $ \langle p,p \rangle  = |p|^2 = r^2$ con $p \in \mathbb{S}^2(0,r)$.

Por tanto tenemos que $vol \, B(0,r) = \frac{r}{3}A(\unitsphere(0,r))$ utilizando la expresión para $A(\unitsphere(0,r))$ antes calculada, concluimos que:

\begin{equation*}
    vol \, B(p_0,r) = \frac{4}{3}\pi r^3.
\end{equation*}
\end{remark}

\begin{remark}[Volumen encerrado por una superficie paralela]
Sea $S$ una superficie compacta y $\epsilon > 0$ tal que $N_\epsilon(S)$ sea un entorno tubular. Hemos denotado a las superficies paralelas como $\{S_t\}_{t \in (-\epsilon, \epsilon)}$. Vamos a calcular el volumen comprendido entre $S$ y $S_t$ con $t\in (0, \epsilon)$. Sea $F: S \times (0, \epsilon) \longrightarrow V_t(S)$ definida como $F(p, t) = p + tN(p)$, donde $V_t(S)=F \big( S \times (0,\epsilon) \big)$. Sean $\Omega$ y $\Omega_t$ los dominios interiores determinados por $S$ y $S_t$, respectivamente. Teniendo en cuenta que $F$ es un difeomorfismo, tenemos:
%
\begin{equation*}
    vol \, \Omega - vol \, \Omega_t = vol \, V_t(S) = \int_{V_t(S)} 1 = \int_{S \times (0,t)} |\text{Jac} \, F|(p,t)
\end{equation*}

Calculemos $|\text{Jac} \, F|$. Dada una base ortonormal $\{e_1, e_2\}$ de direcciones principales en $p \in S$, sabemos que:
%
\begin{align*}
    (dF)_{(p,t)}(e_i,0) &= (1-tk_i(p))e_i, \qquad i = 1,2 \\
    (dF)_{(p,t)}(0,1) &= N(p)
\end{align*}

Tomando $\left\{(e_1,0), (e_2,0), (0,1)\right\}$ como base ortonormal de $T_pS \times \rmath$ y $\left\{e_1, e_2, N(p)\right\}$ como base ortonormal de $\rtres$, obtenemos:
%
\begin{align*}
    |\text{Jac} \, F|(p,t) &= \left|
  det \left( {\begin{array}{ccc}
   1 - tk_1(p) & 0 & 0 \\
   0 & 1-tk_2(p) & 0 \\
   0 & 0 & 1 \\
  \end{array} } \right) \right| \\
  &= 1 - 2tH(p) + t^2K(p)
\end{align*}

Utilizando el teorema de Fubini:

\begin{equation*}
    vol \, \Omega - vol \, \Omega_t = \int_0^t \left( \int_{S} \left( 1-2tH(p)+t^2K(p) \right) \, dp \right) dS
\end{equation*}

y, por tanto:

\begin{equation}\label{volumeparallelsurface}
    vol \, \Omega - vol \, \Omega_t = tA(S) - t^2\int_S H \, dS + \frac{t^3}{3}\int_S K \, dS.
\end{equation}
\end{remark}

\begin{remark}[Área de una superficie paralela]
Sea $S$ una superficie compacta y $\epsilon > 0$ tal que $N_\epsilon(S)$ sea un entorno tubular. Recordamos que hemos denotado a las superficies paralelas como $\{S_t\}_{t \in (-\epsilon, \epsilon)}$.

Sabemos que la función $F_t: S \longrightarrow S_t$ es un difeomorfismo, definido como $F_t(p, t) = p + tN(p)$. Utilizando el cambio de variable, tenemos:

\begin{equation*}
    A(S_t) = \int_{S_t} 1 \, dS_t = \int_S |\text{Jac} \, F_t|(p,t) \, dS
\end{equation*}

Por los cálculos previamente realizados:

\begin{align*}
    |\text{Jac} \, F_t|(p,t) &= \left|
  det \left( {\begin{array}{cc}
   1 - tk_1(p) & 0 \\
   0 & 1-tk_2(p) \\
  \end{array} } \right) \right| \\
  &= 1-2tH(p) + t^2K(p)
\end{align*}

Por tanto, tenemos que:

\begin{equation*}\label{areaparallelsurface}
    A(S_t) = A(S) -2t\int_{S} H(p) \, dS + t^2\int_{S} K(p) \, dS.
\end{equation*}
\end{remark}


\chapter{La desigualdad isoperimétrica en $\rtres$}
En esta sección vamos a definir el resultado que íbamos buscando y  más importante, la desigualdad isoperimétrica en $\rtres$. Para su demostración, necesitamos de algunos elementos que aún no hemos definido. Vamos a verlos como preliminares a la desigualdad.

Vamos a ver las fórmulas de Minkowski que nos serán de utilidad en lo sucesivo.
\begin{theorem}[Fórmulas de Minkowski]
Sea $S$ una superficie compacta, $N$ el dominio interior dado por su aplicación de Gauss y sean $H$ y $K$ las curvaturas media y de Gauss de la superficie. Tenemos:

\begin{enumerate}
    \item $\int_S (1+<p, N(p)>H(p))dp = 0$
    \item $\int_S (H(p)+<p, N(p)>K(p))dp = 0$
\end{enumerate}
\end{theorem}

\begin{definition}
Si $A, B$ son dos conjuntos de $\mathbb{R}^n$, definimos la \textbf{suma conjuntista} de $A+B$ como:

\begin{equation*}
    A+B = \{a+b; a \in A, b \in B\}
\end{equation*}
\end{definition}

\begin{lemma}
Sea $A,B \subset \mathbb{R}^n$. Si uno de los dos conjuntos es abierto, entonces la suma es un abierto.
\end{lemma}
\begin{proof}
Supongamos que $A$ es abierto.
Podemos notas la suma de los conjuntos como:
\begin{equation*}
    A + B = \{a+b; a \in A, b \in B\} = \bigcup_{b \in B} \{a+b; a \in A\}
\end{equation*}

Sabemos que la unión de abiertos es abierta, luego solo nos falta ver que $\{a+b; a \in A\}$ es abierto.

En efecto, tenemos que el conjunto $\{a+b; a \in A\}$ es una traslación del conjunto $A$, que es un movimiento rígido y por tanto conserva las propiedades del conjunto $A$. Concluimos que $A+B$ es un abierto.

Análogo para $B$ abierto.
\end{proof}


\begin{lemma}
Sea $A,B \subseteq \mathbb{R}^n$. Si ambos conjuntos están acotados, entonces la suma está acotada.
\end{lemma}
\begin{proof}
La demostración de esta propiedad es consecuencia directa de la desigualdad triangular.

Sea $L,K > 0$ las cotas de $A$ y $B$ respectivamente. Luego,
\begin{equation*}
    |a+b| \leq |a|+|b| \leq L + K \qquad \forall a \in A, b\in B
\end{equation*}

Luego $A+B$ está acotado.
\end{proof}


\begin{lemma}
Sea $A,B \subset \mathbb{R}^3$. Si ambos conjuntos son arco-conexos, entonces la suma es arco-conexo.
\end{lemma}
\begin{proof}
Veamos que $\forall c,d \in A+B$, $\exists \sigma: [0,1] \longrightarrow A+B$ arco, tal que $\sigma(0)=c$ y $\sigma(1)=d$.

Sea $c=a_1 + b_1$, $d=a_2+b_2$ cualesquiera pertenecientes a $A+B$. Por ser $A$ y $B$ arco-conexos, tenemos que existen $\sigma_1: [0,1] \longrightarrow A$ con $\sigma_1(0)=a_1$ y $\sigma_1(1)=a_2$ y $\sigma_2: [0,1] \longrightarrow B$ con $\sigma_2(0)=b_1$ y $\sigma_2(1)=b_2$. Luego podemos construir, $\sigma: [0,1] \longrightarrow A+B$ tal que $\sigma(0)=\sigma_1(0) + \sigma_2(0) = a_1+b_1 = c$ y $\sigma(1)=\sigma_1(1) + \sigma_2(1) = a_2+b_2 = d$ arco.
\end{proof}

\begin{lemma}\label{paralelepipedoslemma}
Sean $I_1, I_2, I_3$ y $J_1, J_2, J_3$ intervalos abierto de $\rmath$. Los subconjuntos de $\rtres$ dados por:

\begin{equation*}
    A = I_1 \times I_2 \times I_3, \qquad B = J_1 \times J_2 \times J_3
\end{equation*}

son llamados \textbf{paralelepípedos con los ejes paralelos a los ejes coordenados}. Entonces tenemos:

\begin{equation*}
    A + B = (I_1 + J_1) \times (I_2 + J_2) \times (I_3 + J_3)
\end{equation*}

Además, si A y B son disjuntos, tenemos que existe un plano paralelos a uno de los ejes coordenado que separando $A$ y $B$.
\end{lemma}

Con estos preliminares, vamos a demostrar la desigualdad de Brunn-Minkowski basada en la del libro \cite{montielrosbook}, que a su vez será pieza fundamental de la demostración del resultado más importarte de este capítulo, la desigualdad isoperimétrica.
\begin{theorem}[Desigualdad de Brunn-Minkowski]
Sean $A, B$ dos abiertos acotados del espacio euclídeo $\rtres$. Se cumple:

\begin{equation*}
    (vol A)^{\frac{1}{3}} + (vol B)^{\frac{1}{3}} \leq (vol (A+B))^{\frac{1}{3}}
\end{equation*}
\end{theorem}
\begin{proof}
Esta demostración la vamos a hacer en tres pasos. Vamos a probar la desigualdad cuando $A,B$ son paralelepípedos en primer lugar, cuando son unión finita de paralelepípedos abiertos disjuntos acotados cuyos lados son paralelos a los ejes coordenados y por último el caso de dos abiertos acotados cualesquiera.

Supongamos entonces el caso de paralelepípedos, sea $A = I_1 \times I_2 \times I_3$ y $B = J_1 \times J_2 \times J_3$ donde $I_i,J_i \quad \forall i=1,2,3$ son intervalos abiertos y acotados de $\rmath$. Luego:

\begin{equation*}
    \frac{ \left(vol A \right)^{1/3} + \left(vol B \right)^{1/3}}{\left(vol (A+B) \right)^{1/3}} = \frac{\left(\displaystyle\prod_{i=1}^3 a_i \right)^{1/3} + \left(\displaystyle\prod_{i=1}^3 b_i \right)^{1/3}}{\left(\displaystyle\prod_{i=1}^3 a_i+b_i \right)^{1/3}}
\end{equation*}

y por tanto:

\begin{equation*}
    \frac{ \left(vol A \right)^{1/3} + \left(vol B \right)^{1/3}}{\left(vol (A+B) \right)^{1/3}} = \left(\displaystyle\prod_{i=1}^3 \frac{a_i}{a_i+b_i} \right)^{1/3} + \left(\displaystyle\prod_{i=1}^3 \frac{b_i}{a_i+b_i} \right)^{1/3}
\end{equation*}

Además, por la desigualdad de las medias, sabemos que:

\begin{equation*}
    \frac{ \left(vol A \right)^{1/3} + \left(vol B \right)^{1/3}}{\left(vol (A+B) \right)^{1/3}} \leq
    \frac{1}{3} \displaystyle\sum_{i=1}^3 \frac{a_i}{a_i+b_i} + \frac{1}{3} \displaystyle\sum_{i=1}^3 \frac{b_i}{a_i+b_i} = 1
\end{equation*}

Para este caso, la desigualdad queda probada.

Supongamos ahora el segundo caso, sean $A = \displaystyle\bigcup_{i=1}^n A_i$ y $B = \displaystyle\bigcup_{j=1}^n B_j$, donde $A_i$, $B_j$ son paralelepípedos iguales a los del primer caso.

Vamos a probar la desigualdad por inducción sobre el número total de paralelepípedos, $t = n + m$.
El caso $t=2$ es claro, pues es el caso previamente demostrado.
Supongamos cierta la desigualdad cuando $t < n + m$ y veamos que es cierta para $t \geq n + m$. Como ya hemos visto que $t=2$ es cierto, podemos suponer que $n + m \geq 3$ luego es claro que $n > 1$ or $m > 1$.

Supongamos el primero de ellos, por el lema \autoref{paralelepipedoslemma} tenemos que existe un plano, $P$, que separando $A_1$ y $A_2$. Sea $P^+$ y $P^-$ los dos espacios en los que $P$ divide a $\rtres$ y sean $A^+ = A \cap P^+$ y $A^- = A \cap P^-$, además podemos notar a estos como:

\begin{equation}\label{equationuniones}
    A^+ = \displaystyle\bigcup_{i=1}^{n^+} A_i^+ \qquad A^- = \displaystyle\bigcup_{i=1}^{n^-} A_i^-
\end{equation}

una unión finita de de paralelepípedos cuyos ejes son paralelos a los ejes coordenados, donde $n^+ < n$ y $n^- < n$ ya que $P$ divide a $A_1$ y $A_2$.

Cogemos ahora un plano $Q$ que cumpla:

\begin{equation}\label{equationvolumen}
    \frac{vol A^+}{vol A} = \frac{vol B^+}{vol B}
\end{equation}

esto es posible ya que $\frac{vol A^+}{vol A} = c$ con $0 < c < 1$. Luego podemos tomar $Q$ a la izquierda de todos los paralelepípedos $B$ y desplazarlo a la derecha hasta encontrar la igualdad, gracias a que $\frac{vol B^+}{vol B}$ es una función continua.
Dado que $vol A = vol A^+ + vol A^-$ tenemos la misma \autoref{equationvolumen} para $A^-$ y $B^-$.

Además, $B = B^+ + B^-$ cumplen también la \autoref{equationuniones} para $B^+$ y $B^-$ con $m^+ \leq m$ y $m^- \leq m$, en este caso, no podemos suponer que son estrictos debido a que no tenemos garantizado el hecho de que $Q$ separe a dos paralelepípedos de $B$.

Estamos en condiciones de aplicar la hipótesis de inducción a los subconjuntos $A^+$, $A^-$ y $B^+$, $B^-$ ya que el total de paralelepípedos cumple la condición $n^+ + m^+ < n + m$ y $n^- + m^- < n + m$. Por tanto tenemos:

\begin{equation*}
    vol (A^+ + B^+) \geq \left[ (vol A^+)^{1/3} + (vol B^+)^{1/3} \right]^3
\end{equation*}
\begin{equation*}
    vol (A^- + B^-) \geq \left[ (vol A^-)^{1/3} + (vol B^-)^{1/3} \right]^3
\end{equation*}

Seguimos la deducción sabiendo que $A^+ \subset P^+$ y $B^+ \subset Q^+$, luego $A^+ + B^+ \subset P^+ + Q^+ = (P+Q)^+$ y análogamente $A^- + B^- \subset (P+Q)^-$. Teniendo en cuenta que $P+Q$ es otro plano de $\rtres$ tenemos que $A^+ + B^+$ y $A^- + B^-$ son disjuntos. En consecuencia:

\begin{equation*}
    vol (A+B) \geq vol(A^+ + B^+) + vol(A^- + B^-) \geq \left[ (vol A^+)^{1/3} + (vol B^+)^{1/3} \right]^3 + \left[ (vol A^-)^{1/3} + (vol B^-)^{1/3} \right]^3
\end{equation*}

Y teniendo en cuenta la \autoref{equationvolumen}:

\begin{equation*}
    vol (A+B) \geq vol A^+ \left[ 1 + \left( \frac{vol B}{vol A} \right)^{1/3}  \right]^3 + vol A^- \left[ 1 + \left( \frac{vol B}{vol A} \right)^{1/3}  \right]^3 \geq vol A \left[ 1 + \left( \frac{vol B}{vol A} \right)^{1/3}  \right]^3 = \left[ (vol A)^{1/3} + (vol B)^{1/3} \right]^3
\end{equation*}

Luego la desigualdad se cumple para este segundo caso.

Consideremos finalmente el tercer caso, con $A$ y $B$ dos abiertos acotados cualesquiera de $\rtres$. Usando la teoría de integración de Lebesgue, existen dos sucesiones de conjuntos abiertos de $\rtres$ como los del caso anterior, notémoslas como $A_n$ y $B_n$, con $n \in \mathbb{N}$, $A_n \subset A$, $B_n \subset B$ y:

\begin{equation*}
    \lim_{x \to \infty} vol A_n = vol A \qquad \lim_{x \to \infty} vol B_n = vol B
\end{equation*}

Luego tenemos que $A_n + B_N \subset A + B$ y para cada $n \in \mathbb{N}$:

\begin{equation*}
    vol (A+B)^{1/3} \geq vol (A_n+B_n)^{1/3} \geq vol (A_n) ^{1/3} + vol(B_n)^{1/3}
\end{equation*}

Basta con tomar límite con $n$ tendiendo a infinito y tenemos probada la desigualdad para todos los casos.
\end{proof}

Con la desigualdad de Brunn-Minkowski demostrada, vamos a probar el resultado más importante de este capítulo y sobre el que trabajaremos en el capítulo siguiente.
\begin{theorem}[Desigualdad Isoperimétrica]
Sea $S$ una superficie conexa y compacta con dominio interior $\Omega$. Se cumple:

\begin{equation*}
    A(S)^3 \geq 36\pi(vol \Omega)^2
\end{equation*}
\end{theorem}
\begin{proof}
Para demostrarlo, tenemos que buscar primero cierta situación donde podemos aplicar la desigualdad de Brunn-Minkowski y demostrar la desigualdad isoperimétrica.

Tomamos $\epsilon > 0$, tal que $N_\epsilon(S)$ es un entorno tubular. Para cada $t \in (0, \epsilon)$, tenemos que $S_t$ es una superficie paralela incluida en $\Omega$ y $\Omega_t$ es el dominio interior determinado por $S_t$, y además, vamos a tomar $B_t$ como la bola abierta centrada en el origen de $\rtres$ y radio $t$.

Consideremos el conjunto $S_t + B_t$, $\forall t \in (0, \epsilon)$. Veamos primero que los puntos de este conjunto no están en la superficie. Sea $p \in (S_t + B_t) \cap S$. Por definición, $p = q + v$ con $q \in S_t$ y $v \in B_t$, luego se cumple que $|p - q| = |v| < t$, luego $dist(q, S) < t$ lo que es una contradicción ya que el mínimo de la distancia de $q$ a $S$ se alcanza en un punto $p_0$ tal que $q$ está en la recta normal de $p_0$ y $S$, y teniendo en cuenta que $q \in S_t$, tenemos que $|p_0 - q| \geq t$.

Por tanto, concluimos que $\nexists p \in (S_t + B_t) \cap S$. Por otro lado, sabemos que $S_t$ y $B_t$ son arco-conexos, y hemos probado que la suma de arco-conexos es arco-conexa, por lo que $(S_t + B_t)$ es arco-conexo y por tanto tiene que estar incluido en una de las componentes arco-conexas de $\rmath - S$. Teniendo en cuenta que $\Omega_t \subset \Omega$, tenemos que $\Omega_t + B_t \subset \Omega$, $\forall t \in (0, \epsilon)$. Y tenemos la situación que buscábamos.

%NOTE: (?) con la inclusión suficiente para afirmar
Aplicando ahora la ya conocida, desigualdad de Brunn-Minkowski, tenemos que:

\begin{equation*}
    vol \Omega \geq vol (\Omega + B_t) \geq \left[ (vol \Omega)^{1/3} + vol (B_t)^{1/3} \right]^3
\end{equation*}

$\forall t \in (0, \epsilon)$. Desarrollamos la parte de la derecha y obtenemos:

\begin{equation*}
    vol \Omega \geq vol \Omega_t + 3(vol \Omega_t)^{2/3}(vol B_t)^{1/3} + 3(vol \Omega_t)^{1/3}(vol B_t)^{2/3} + vol B_t
\end{equation*}

y simplificando:

\begin{equation*}
    vol \Omega \geq vol \Omega_t + 3(vol \Omega_t)^{2/3}(vol B_t)^{1/3}
\end{equation*}

Sustituimos ahora el valor del volumen de la bola que hemos visto previamente:

\begin{align*}
    vol \Omega &\geq vol \Omega_t + 3(vol \Omega_t)^{2/3} \left( \frac{4\pi}{3}t^3 \right)^{1/3} \\
    vol \Omega &\geq vol \Omega_t + 3t \left( \frac{4\pi}{3} \right)^{1/3} (vol \Omega_t)^{2/3}
\end{align*}

y obtenemos:

\begin{equation*}
    \frac{vol \Omega - vol \Omega_t}{t} \geq 3 \left( \frac{4\pi}{3} \right)^{1/3} (vol \Omega_t)^{2/3}
\end{equation*}

Conociendo ahora el volumen encerrado entre superficies paralelas, nos queda:

\begin{align*}
    \frac{tA(S) - t^2\int_S H + \frac{t^3}{3}\int_S K}{t} &\geq 3 \left( \frac{4\pi}{3} \right)^{1/3} (vol \Omega_t)^{2/3} \\
    A(S) - t\int_S H + \frac{t^2}{3}\int_S K &\geq 3 \left( \frac{4\pi}{3} \right)^{1/3} (vol \Omega_t)^{2/3}
\end{align*}

Tomando límites en ambos lados con $t$ tendiendo a 0, tenemos:

\begin{equation*}
    A(S) \geq 3 \left( \frac{4\pi}{3} \right)^{1/3} (vol \Omega)^{2/3}
\end{equation*}

y elevando al cubo tenemos la desigualdad que buscamos:

\begin{equation*}
    A(S)^3 \geq 36\pi(vol \Omega)^2
\end{equation*}
\end{proof}

\begin{proposition}
Las esferas son superficies isoperimétricas que minimizan el área entre todas las superficies compactas y conexas con su mismo volumen.
\end{proposition}
\begin{proof}
Veamos en primer lugar que las esferas son una superficie isoperimétrica con área mínima, para ello, basta ver que las esferas cumplen la igualdad isoperimétrica.

Sea $S=\mathbb{S}^2$ superficie compacta y conexa y $\Omega = B$, bola abierta, el dominio interior determinado por $S$. Sabemos que $A(S)=4\pi r^2$ y $vol \Omega = \frac{4\pi r^3}{3}$, luego sustituyendo en la desigualdad isoperimétrica, tenemos:

\begin{align*}
    64 \pi^3 r^6 &\geq 36\pi \frac{16\pi^2r^6}{9} \\
    64 \pi^3 r^6 &\geq 64\pi^3r^6 \\
    64 \pi^3 r^6 &= 64\pi^3r^6
\end{align*}

Veamos ahora que es la única. Supongamos que existe otra $S'$ tal que $vol \Omega' = vol \Omega$, donde $\Omega'$ es el dominio interior determinado por la superficie $S'$, aplicando la desigualdad isoperimétrica:

\begin{equation*}
    A(S')^3 \geq 36\pi (vol \Omega')^2 = 36\pi (vol \Omega)^2 = A(S)^3
\end{equation*}

Luego las esferas son las superficies isoperimétricas que minimizan el área.
\end{proof}

\begin{corolario}
Sea $S \subset \rtres$ una superficie compacta y conexa y $\Omega$ su dominio interior. Entonces $S$ es isoperimétrica si y solo si $A(S)^3 = 36\pi (vol \Omega)^2$.
\end{corolario}


\chapter{Unicidad de las superficies isoperimétricas}
En el capítulo previo, hemos visto que las esferas son superficies isoperimétricas, superficies que encierran un volumen dado con área mínima. En este capítulo, vamos a ver que son las únicas. Para esto, vamos a comenzar demostrando que todas las superficies isoperimétricas tienen curvatura media constante. Nos ayudaremos de técnicas variacionales que introduciremos a continuación.

\section{Propiedad variacional de las superficies isoperimétricas}

Sea $S$ superficie compacta y conexa de $\rtres$, sea $\epsilon  \rangle  0$ suficientemente pequeño tal que $N_\epsilon$ sea un entorno tubular y sea $f: S \longrightarrow \rmath$ diferenciable. Consideremos la función $\Phi_t(p) = p + tf(p)N(p)$ con $p \in S$ y el conjunto:

\begin{equation*}
    S_t(f) = \left\{ x \in N_\epsilon(S) \ | \ x = p + tf(p)N(p); \ p \in S, t \in (-\delta, \delta) \right\}
\end{equation*}

La función $\Phi_t(s) = p + tf(p)N(p) = F(p, tf(p))$, donde $F$ es el difeomorfismo $F: S \times (-\epsilon, \epsilon) \longrightarrow N_\epsilon(S)$ dado por $F(p,t) = p + tN(p)$ con $N$ el normal interior, está bien definido por ser composición de funciones diferenciables.

Tenemos que $\Phi_t$ es inyectiva, por definición de inyectiva, Si $p,q \in S$ con $\Phi_t(p) = \Phi_t(q)$, entonces $F(p, tf(p)) = F(q, tf(q))$ lo que implica que $p=q$ al se $F$ inyectiva.

Veamos ahora, que $(d\Phi_t)_p: T_pS \longrightarrow \rtres$ es inyectiva. Dado $v \in T_pS$, se tiene:

\begin{align*}
    (d\Phi_t)_p(v) &= v + t((df)_p(v)N(p) + f(p)(dN)_p(v)) \\
    (d\Phi_t)_p(v) &= (v - tf(p)A_p(v)) + (t(df)_p(v)N(p))
\end{align*}

Nótese que tenemos la expresión dividida en dos partes, la primera, la parte tangente a $S$ y la segunda la parte normal. Supongamos que $(d\Phi_t)_p(v) = 0$, entonces ambas partes son 0, esto es, $v - tf(p)A_p(v) = 0$.

Recordemos que $F_u: S \longrightarrow \rtres$ dada por $F_u(p) = p + uN(p)$ tiene diferencial inyectiva si $|u|  \langle  \epsilon$. Tomando $u=tf(p)$ tenemos que, $v - tf(p)A_p(v) = (dF_u)_p(v)$, luego $v=0$. Con esto tenemos que $ker((d\Phi_t)_p) = \{0\}$ y por tanto la diferencial de $\Phi_t$ es inyectiva y podemos concluir que $\Phi_t: S \longrightarrow S_t(f)$ es un difeomorfismo y $S_t(f)$ es una superficie compacta por ser difeomorfa a $S$.

Se llama \textbf{variación de $S$ correspondiente a la función f} a la familia de superficies $S_t(f)$ con $t  \langle  \delta$.

\begin{remark}[Primera variación del área]
Sea $S_t(f)$, con $t \in (-\delta, \delta)$, la variación de la superficie $S$ para la función $f: S \longrightarrow \rmath$. Entonces, la función dada por:

\begin{equation*}
    t \longrightarrow A(t) = A(S_t(f))
\end{equation*}

es diferenciable y

\begin{equation}\label{variacionarea}
    \diff{}{t}{t=0} A(S_t(f)) = -2 \int_S f(p)H(p) dp
\end{equation}
\end{remark}
\begin{proof}
Sean $\{e_1, e_2\}$ una base ortonormal de curvaturas principales de $S$ en el punto $p \in S$, tenemos:

\begin{equation*}
    (d\Phi_t)_p(e_i) = (1 - tf(p)k_i(p))e_i + t(df)_p(e_i)N(p) \qquad i=1,2
\end{equation*}

y

\begin{equation}\label{productoescalar}
    (d\Phi_t)_p(e_1) \wedge (d\Phi_t)_p(e_2) = (1 - 2tf(p)H(p))N(p) - t(\triangledown f)_p + t^2G(p,t)
\end{equation}

donde $\triangledown f$ es el gradiente de la función $f$ y $G$ es una función diferenciable definida en $S \times (-\delta, \delta)$ tomando valores en $\rtres$. Aplicando el teorema del cambio de variable, tenemos:

\begin{equation*}
    A(t) = A(S_t(f)) = \int_{S_t(f)} 1 = \int_S |Jac \Phi_t|
\end{equation*}

Calculamos la diferencial en $t=0$:

\begin{equation*}
    A'(0) = \int_S \diff{}{t}{t=0} |(d\Phi_t)_p(e_1) \wedge (d\Phi_t)_p(e_2)|dp 
\end{equation*}

usando \autoref{productoescalar}:

\begin{equation*}
    \diff{}{t}{t=0} |(d\Phi_t)_p(e_1) \wedge (d\Phi_t)_p(e_2)|dp = -2f(p)H(p)
\end{equation*}
\end{proof}


\begin{remark}[Primera variación del volumen]
Sea $S_t(f)$, con $t \in (-\delta, \delta)$, la variación de la superficie $S$ para la función $f: S \longrightarrow \rmath$. Entonces, la función dada por:

\begin{equation*}
    t \longrightarrow V(t) = vol \Omega_t (f)
\end{equation*}

es diferenciable y

\begin{equation}\label{variacionvolumen}
    \diff{}{t}{t=0} vol \Omega_t (f) = - \int_S f(p) dp
\end{equation}
\begin{proof}
Sabemos por \autoref{volumensuperficiecompacta} que:

\begin{equation*}
    V(t) = - \frac{1}{3} \int_{S_t}  \langle p, N_t(p) \rangle  dp
\end{equation*}

donde $N_t$ es el normal interior de $S_t(f)$. Usando de nuevo el teorema del cambio de variable, tenemos:

\begin{equation*}
    V(t) = - \frac{1}{3} \int_{S}  \langle N_t \circ \Phi_t, \Phi_t \rangle |Jac \Phi_t|
\end{equation*}

Como el difeomorfismo $\Phi_t$ es la función identidad cuando $t=0$, teniendo en cuenta que:

\begin{equation*}
    (N_t \circ \Phi_t)(p) = \frac{(d\Phi_t)_p(e_1) \wedge (d\Phi_t)_p(e_2)}{|(d\Phi_t)_p(e_1) \wedge (d\Phi_t)_p(e_2)|} = \frac{(d\Phi_t)_p(e_1) \wedge (d\Phi_t)_p(e_2)}{|Jac \Phi_t|(p)}
\end{equation*}

y utilizando \autoref{productoescalar} tenemos:

\begin{align*}
    V(t) = - \frac{1}{3} \int_S (1-2tf(p)H(p)) \langle N(p), p \rangle  \\ - \frac{1}{3} \int_S tf(p) - t \langle (\triangledown f)_p, p \rangle  + t^2D(p,t) dS
\end{align*}

siendo $D$ una función diferenciable definida en $S \times (-\delta, \delta)$. Derivando cuando $t=0$, tenemos:

\begin{align*}
    V'(0) = \frac{1}{3} \int_S [-f(p) + 2f(p)H(p) \langle N(p),p \rangle  +  \langle (\triangledown f)_p, p \rangle ] dp
\end{align*}

Utilizando el corolario \autoref{corolariogradiente}:

\begin{align*}
    V'(0) &= \frac{1}{3} \int_S -f(p) dp + \frac{1}{3} \int_S [2f(p)H(p) \langle N(p),p \rangle  +  \langle (\triangledown f)_p, p \rangle ] dp \\
    V'(0) &= \frac{1}{3} \int_S -f(p) - 2 f(p)dp \\
    V'(0) &= - \int_S f(p) dp
\end{align*}
\end{proof}
\end{remark}

Veamos ahora el resultado que íbamos buscando con ayuda de las propiedades variacionales calculadas.

\begin{theorem}[Las superficies isoperimétricas tienen curvatura media constante]
Sea $S$ una superficie compacta y conexa. Si $S$ es isoperimétrica, entonces, $S$ tiene curvatura media constante.
\end{theorem}
\begin{proof}
Si $S$ es una superficie isoperimétrica, consideramos la función $f: S \longrightarrow \rmath$ diferenciable y la variación $S_t(f)$ de $S$ con respecto a $f$, con $t \in (-\delta, \delta)$. Entonces la función $h: (-\delta, \delta) \longrightarrow \rmath$ dada por $h(t) = A(S_t(f))^3 - 36\pi(vol \Omega_t(f))^2$ es diferenciable y alcanza su mínimo en $t=0$ ya que se $S_0=S$ y partimos de la suposición de que $S$ es isoperimétrica. Luego $h'(0)=0$ y por tanto, tenemos:

\begin{equation*}
    h'(0) = 3A(S_0(f))^2 \diff{}{t}{t=0} A(S_t(f)) - 72\pi vol \Omega_0(f) \diff{}{t}{t=0} vol \Omega_t(f)
\end{equation*}

Utilizando \autoref{variacionarea} y \autoref{variacionvolumen}:

\begin{equation*}
    0 = h'(0) = 3A(S)^2 \left( -2 \int_S f(p)H(p)dp \right) - 72\pi vol \Omega \left( - \int_S f(p) dp \right)
\end{equation*}

Sacando factor común obtenemos:

\begin{equation*}
    0 = 6 \int_S f(p)(-A(S)H(p) + 12 \pi vol \Omega)
\end{equation*}

Al estar igualado a 0, podemos omitir el producto por un número positivo. Además, esta igualdad de tiene que cumplir para toda función $f$ diferenciable en $S$, luego tomemos $f = 12 \pi vol \Omega - A(S)H(p)$ diferenciable:

\begin{equation*}
    0 = \int_S (12 \pi vol \Omega - A(S)H(p))^2
\end{equation*}

Luego $12 \pi vol \Omega(f) - A(S)H(p) = 0$ y despejando $H$:

\begin{equation*}
    H(p) = \frac{12 \pi vol \Omega}{A(S)^2}, \qquad p \in S
\end{equation*}


Luego $H$ es constante.
\end{proof}

\section{Unicidad de las esferas como superficies isoperimétricas}


\chapter{Conclusiones y vías futuras}
\section{Conclusiones y vías futuras}

En conclusión, hemos probado que las esferas son las únicas superficies isoperimétricas, lo que soluciona uno de los problemas clásicos de optimización estudiado desde hace mucho tiempo. Hemos probado este resultado para el espacio $\rtres$, pero es extensible para dimensión más alta, una demostración similar permite probar que toda hipersuficie compacta y conexa en $\rmath^n$ minimizando el área (n-$1$)-dimensional con volumen fijo debe ser una esfera, existen trabajos sobre esto como en \cite{paperchicago}.

Este resultado nos permite emplearlo en situaciones cotidianas cuando se necesitemos encerrar cierto volumen con área mínima, pues ya sabemos que la forma óptima es utilizando una esfera.

\part{Representación de información imprecisa en base de datos NoSQL}

\chapter{Descripción}
\section{Resumen}

En este trabajo estudiaremos teoría de conjuntos difusos para modelar matemáticamente información que utilizamos habitualmente para comunicarnos. A partir de ello, realizaremos una extensión de la funcionalidad de MongoDB para trabajar con información imprecisa.

Veremos una introducción a las bases de datos NoSQL, utilizadas frecuentemente para técnicas de Big Data, y su comparativa con los sistemas relacionales de bases de datos clásicos. Haremos una introducción a la teoría de conjuntos y las operaciones que se pueden realizar con ellos y veremos como incluir información imprecisa en la base de datos MongoDB.

Concluiremos con una propuesta software para extender los operadores de consulta y proyección de MongoDB.

\paragraph{Palabras clave} nosql, mongodb, conjuntos difusos, probabilidad, trapezoides, bases de datos, consulta difusa

\newpage

\section{Introducción}

En este trabajo estudiaremos teoría de conjuntos difusos para modelar matemáticamente información que utilizamos habitualmente para comunicarnos. A partir de ello, realizaremos una extensión de la funcionalidad de MongoDB para trabajar con información imprecisa.

\subsection{Contextualización y descripción del trabajo}

Las bases de datos tradicionales almacenan tipos de datos únicos, es decir, almacenan solo un tipo de dato concreto, número, cadena de texto, fecha... pero no están preparados para almacenar los tipos de datos difusos con las que las personas estamos acostumbrados a manejar. Por ejemplo, es muy común referirnos a la altura de una persona mediante etiqueta como "alto", "estatura media", etc, en una base de datos no es posible almacenar esta información ni realizar consultas como "obtén todos las personas altas". 

Otro aspecto a tener en cuenta, es la llegada de las bases de datos NoSQL, que vienen a dar soluciones donde las bases de datos relacionales no llegan. Cada vez es más frecuente el uso de las primeras debido a la gran cantidad de datos que se recopilan hoy en día y la necesidad de explotarlos y sacar información valiosa de ellos, las bases de datos NoSQL suelen ofrecer alternativas más eficientes para estos casos. Es por ello que bases de datos como MongoDB, basada en documentos, es cada vez más usada y ahora mismo una de las más populares, su estructura hace que se adapte perfectamente a las aplicaciones web y su facilidad y versatibilidad de modelado de datos hace que sea una alternativa a tener en cuenta. Además, tiene soporte y librerías para la mayoría de lenguajes de programación utilizados.

El presente trabajo pretende dar una solución de representación de conjuntos difusos en la base de datos NoSQL. Vamos a intentar dar una solución de operaciones ``fuzzy'' similar a la propuesta por Medina en \cite{tesismedina} para bases de datos relacionales, pero en base de datos NoSQL, intentando facilitar la extracción de información difusos y dotar a un sistema experto de la posibilidad de tomar mejores decisiones pudiendo interpretar información que para los humanos es común para expresarse.

\subsection{Estructura del trabajo}

Es trabajo está dividido en diferentes capítulos para mejorar la organización y poder separar conceptos, la distribución que sigue es la siguiente.

En el \autoref{chapter:nosqlmongodb} daremos una introducción a las bases de datos NoSQL, explicando los diferentes tipos y propuestas que existen. Veremos las principales diferencias con los SGBDR y nos centraremos en explicar la base de datos seleccionada para la propuesta, MongoDB.

En el \autoref{chapter:fuzzysets} veremos la teoría de conjuntos difusos. Se darán las definiciones y resultados necesarios para entender la teoría difusa y poder comprender cómo se va a almacenar en base de datos. Además, veremos distintas propuestas de bases de datos difusas y la definición de los operadores difusos que implementaremos en nuestra propuesta.

En el \autoref{propuesta} explicaremos la propuesta que hemos realizado para modelar conjuntos difusos en MongoDB. Explicaremos la función \texttt{fuzzy\_find} y los operadores utilizados en ella.

\subsection{Bibliografía fundamental}

Para este trabajo se han consultado muchas referencias, vamos a destacar entre todas ellas:

\begin{itemize}
    \item Bases de datos Objeto-relaciones difusas: Modelo, arquitectura y aplicaciones, Tesis doctoral por Carlos D. Barranco, contiene una propuesta de implementación difusa basada en propuesta de \cite{tesismedina}. Además ha sido fundamental en el \autoref{chapter:fuzzysets} para el entendimiento y redacción de la teoría difusa.
    \item Tratamiento de la imprecisión en bases de datos relacionales: Extensión del modelo y adaptación de los SGBD actuales, tesis doctoral por José Galindo. Fundamental junto con la anterior para los conjuntos y operadores difusos. Tiene unas gráficas para facilitar el entendimiento de los operadores.
    \item Manual de MongoDB. Utilizado para la implementación de la solución software propuesta.
    \item Evaluation of indexing strategies for possibilistic queries based on indexing techniques available in traditional RDBMS, por Juan Miguel Medina, Carlos D. Barranco y Olga Pons, en este artículo ha sido utilizado para enteder e implementar el operador \texttt{feq}. Además contiene la estrategia de indexación que se usa para este tipo de implementación.
    \item Indexing techniques to improve the performance of necessity-based fuzzy queries using classical indexing of RDBMS, por Juan Miguel Medina, Carlos D. Barranco y Olga Pons, el artículo es similar al anterior pero con la medida de necesidad, por lo que ha sido consultado para la implementación de los operadores basados en necesidad.
\end{itemize}


\subsection{Objetivos}

El objetivo que se planteó inicialmente para el trabajo fue hacer una prueba de concepto sobre una base de datos NoSQL para poder trabajar con información imprecisa. Además, se pretendía dar alguna utilidad implementada para la base de datos para poder realizar consultas difusas.

El objetivo se ha conseguido completamente, se ha realizado un estudio de las bases de datos NoSQL en el \autoref{chapter:nosqlmongodb}. De ahí se ha elegido MongoDB por su uso extendido en Big Data y en aplicaciones web, además de ser una de las más utilizadas actualmente. Se ha dado una visión completa de la teoría de conjuntos difusos en el \autoref{chapter:fuzzysets}, documentando los números difusos que utilizaremos para representar información difusa en la base de datos. Por último se ha dado la propuesta del operador \texttt{fuzzy\_find} en el \autoref{propuesta} que permite las consultas clásicas de MongoDB pero además ofrece una funcionalidad para realizar consultas con cláusulas difusas.

Durante la realización de la propuesta, comenzamos con una implementación sobre el operador de agregación de MongoDB, \textit{map-reduce}, que finalmente se descarta a favor de utilizar el operador de agregación \textit{pipeline}, que nos ofrecía mejor rendimiento y una mejor sintaxis.

Destacar que uno de los objetivos principales de la propuesta para trabajar con conjuntos difusos era que no se perdiese la forma de trabajar con MongoDB, de modo que la sintaxis fuese similar y trabajar con conjuntos difusos fuese similar a trabajar con tipos nativos. Como puede verse, este objetivo se ha cumplido ya que se ha realizado una extensión de operadores de MongoDB que funcionan exactamente igual que los nativos, de hecho nuestra función \texttt{fuzzy\_find} acepta consultas nativas sin perder generalidad.

A continuación se enumeran las materias del doble grado más relacionadas con este trabajo:
\begin{itemize}
    \item Fundamentos de bases de datos
    \item Inteligencia de negocio
    \item Inferencia estadística y probabilidad
\end{itemize}

\newpage

\begin{otherlanguage}{american}
\pdfbookmark[1]{Abstract}{Abstract}
\section{Abstract}

This work studies fuzzy set theory to get a mathematic model of imprecise information. Using this model, we will do an extension of MongoDB tools to work with imprecise information. MongoDB is a database categorized as NoSQL database.

We will give an introduction to NoSQL databases, these databases are usually used for Big Data techniques. Also, we will compare NoSQL databases with classic SQL databases. This work shows fuzzy sets theory, its operations and it gives a proposal to include imprecise information in the MongoDB database.

We will finish this work with a software implementation to extend MongoDB operators for match and projection stages. This solution adds support to MongoDB to save and query imprecise information.

\subsection{NoSQL}

NoSQL databases born to solve some performance issues that SQL databases have when information saved is big enough. In general, NoSQL databases are more varied than systems relational databases, that is because some companies are creating databases NoSQL types to solve its particular issues. That behavior, increase NoSQL databases and nowadays exists varied types of NoSQL databases depend on the type of data saved. NoSQL databases types are document-based, key-value, columns, graphs...

NoSQL databases do not have standard SQL language as its query language, instead of this, each NoSQL database has its own query language and its own query operators. Moreover, we will see that NoSQL databases do not guarantee the ACID operations (Atomic, Consistency, Isolate, Durability). We will show what is ACID and how can we categorized NoSQL databases though CAP theorem.

\subsection{MongoDB}

MongoDB is a document-based database, nowadays, it is one of the most used NoSQL databases. MongoDB has an extended use for many of Big Data techniques. MongoDB is a good choice for web applications because of its structure, easy to use and its query language. MongoDB allows to work with data based on JSON-like documents, it is easily modeled for any programming language, even more, the most of programming language has a compatible native type of data. MongoDB has libraries in almost all modern programming languages. This work introduces MongoDB databases and its main operators.

We will show the aggregation operators that MongoDB includes to work with information saved in a database. Our software proposal is based on one of them, the \textit{pipeline} operator. We will see how it works and the main operators that we need to implement our proposal.

\subsection{Fuzzy sets}

Fuzzy sets allow us to get a mathematic model of imprecise information. Current databases only allow to save types of data precise or ``crisp'', but daily, we could see different situations whose we need imprecise information. For instance, when we tag a person as ``young'', we understand that person do not have 80 years, but ``young'' as a concept depending on the context where it is used, the people which are used this terms or even depend on the point of view of the person that uses this term, that is, ``young'' is different for a 15 years person than a 50 years person.

Fuzzy sets try to do a mathematics solution to this different options. They try to give an option to save this information in a database. Due to fuzzy numbers, we could save imprecise information in a database and query it.

This work explains the main properties of fuzzy sets and its main operators to work with them.

\subsection{Fuzzy find}

The command \textbf{fuzzy\_find} is our software proposal to use imprecise information in MongoDB. It is developed in a script written in javascript programming language to store easily this command and its operator directly in MongoDB databases. The main goal of this command is to allow us to do generic queries of MongoDB and extend this functionality with fuzzy operators to query fuzzy data.

We will see the fuzzy operator algorithm implementations to match and projection stages of pipeline aggregation operator. The command \textbf{fuzzy\_find} keep the syntax proposed by MongoDB to its query language and operators. The command \textbf{fuzzy\_find} is simple to use and we will show it with some examples of its use.

Finally, this work will give the restrictions of the command, improvements and future jobs.


\paragraph{Keywords} databases, nosql, mongodb, fuzzy sets, possibility, necessity, trapezoidal fuzzy numbers, fuzzy query.


\end{otherlanguage}

\newpage


\chapter{NoSQL. MongoDB}
\section{NoSQL}

% NOSQL-MONGO:
    % 1. Historia
    % 2. Características
    % 3. SQL vs NoSQL
    % 4. Ejemplos y usos
    % 5. MongoDb
    
% Teoría difusa:
    % 1. Teoría de conjuntos difusos
    %   1.1. Concepto
    %   1.2. Operaciones
    % 2. Bases de datos difusas
    % 3. Propuesta Fuzzy-Mongo
    %   3.1. Operadores fuzzy
    %   3.2. Proyecciones fuzzy
    %   3.3. Fuzzy_find
    %   3.2. Ejemplos


\chapter{Conjuntos Difusos}
En este capítulo, haremos una introducción a la teoría de conjuntos difusos. Veremos qué son y las principales operaciones con estos conjuntos. Esta permite nos permite modelar matemáticamente información imperfecta, que es usada frecuentemente por los humanos en su comunicación y razonamiento.

Vamos a ver que bases de datos difusas existen y sus principales características para adentrarnos en el mundo de la representación difusa en base de datos. El objetivo de esta es dotar a los sistemas automáticos de una poderosa forma de toma de decisiones, similares a las que tomamos las personas, basada en información que no pretende ser del todo exacta.

\section{Teoría de conjuntos difusa}

La principal teoría para habla de datos imprecisos fue introducida por L.A Zadeh \cite{fuzzysetszadeh} en 1965. Vamos a basarnos en ella para explicar los conjuntos difusos.

La teoría de conjuntos difusos (\textit{fuzzy sets}) hace un generalización de la teoría de conjuntos clásicas, en la que se define un conjunto como un grupo de elementos que pertenecen o no a este conjunto. En la teoría de conjuntos difusa, se añade un elemento más a tener en cuenta, el grado de pertenencia al conjunto, esto es, cada elemento pertenece a un conjunto con un grado de pertenencia determinado, este grado de pertenencia suele representarse con un número $x \in [0,1]$. Haciendo una analogía, en la teoría de conjuntos clásica, los elementos pertenecen con solo dos posibles valores $0$ ó $1$, que indicarían si pertenece o no al conjunto.

Basándonos en los descrito anteriormente, vamos a dar una definición de conjunto difuso.

\begin{definition}[Conjunto difuso]
Sea $\Omega$ un dominio (de objetos), notemos a los elementos de $\Omega$ por $x$. Sea $f: \Omega \longrightarrow [0,1]$ una función. Definimos un \textbf{conjunto difuso}, $F$, como:

\begin{equation*}
    F = \left\{ f(x)/x \enspace | \enspace x\in\Omega, f(x) \in [0,1] \right\}
\end{equation*}
\end{definition}

Es común no dar una lista exhaustiva de elementos del conjunto difuso, basta con dar una definición de la función $f$, que se denomina \textbf{función de pertenencia}.

Estos conceptos suelen ser muy subjetivos y en la práctica, suelen depender del contexto en el que se encuentren, esto es, un mismo valor, depende del contexto, puede pertenecer a un conjunto u otro. Veamos un mismo ejemplo en dos contextos distintos para ilustrarlo, aprovecharemos el ejemplo para afianzar la definición de conjunto difuso.

\begin{example}
Supongamos que tenemos los datos de la tabla \ref{datatable} con la altura de 5 personas:

\begin{table}[h]
\centering
\begin{tabular}{|l|l|}
\hline
\textbf{Nombre} & \textbf{Altura} \\ \hline
Juan            & 175             \\ \hline
Pepe            & 195             \\ \hline
Marco           & 170             \\ \hline
Felipe          & 180             \\ \hline
Antonio         & 185             \\ \hline
\end{tabular}
\caption{Datos de altura}
\label{datatable}
\end{table}

Vamos a definir el conjunto difuso que representa a los jugadores \textit{altos} en dos deportes distintos:

Supongamos como primer deporte el baloncesto, y podríamos tener una estimación del conjunto difuso como sigue:
%TODO Mira el pdf, algún paquete te transforma los puntos decimales en comillas. Corrígelo!.
\begin{equation*}
    altos = \{0.8/195, 0.3/180, 0.35/185\}
\end{equation*}

sin embargo, para un deporte como el fútbol, podríamos tener el siguiente conjunto:

\begin{equation*}
    altos = \{0.2/175, 0.1/170, 1/195, 0.7/180, 0.82/185\}
\end{equation*}

\end{example}

Aprovechando el ejemplo previo, vamos a introducir un nuevo concepto:

\begin{definition}[Etiqueta lingüística]
Llamamos \textbf{etiqueta lingüística} a aquella palabra en lenguaje natural que describe un conjunto difuso.
\end{definition}

\subsection{Conceptos sobre conjuntos difusos}

Similar a los conceptos de la teoría clásica de conjuntos, vamos a dar una introducción a los conceptos sobre conceptos difusos, veremos como todos ellos depende de la función de pertenencia.

Sean $\Omega$ un dominio, y sean $F_1$ y $F_2$ dos conjuntos difusos con $f_1$ y $f_2$ sus funciones de pertenencia, respectivamente. Entonces, se tiene:

\begin{itemize}
    \item $F_1$ y $F_2$ son \textbf{iguales} si, y solo si, las funciones de pertenencia son iguales. Esto es, $F_1 = F_2 \Leftrightarrow f_1(x) = f_2(x) \forall x \in \Omega$.
    \item $F_1$ está \textbf{incluido} en $F_2$ si, y solo si, $f_1$ es menor o igual que $f_2$. Esto es, $F_1 \subseteq F_2 \Leftrightarrow f_1(x) \leq f_2(x) \forall x \in \Omega$.
    \item Se define el \textbf{soporte} de $F_1$ como el subconjunto de valores de $\Omega$ tal que la función de pertenencia es mayor que cero. Esto es, $sop(F_1) = \{ x \in \Omega \enspace | \enspace f_1(x) > 0 \}$
    \item Se define el \textbf{$\alpha$-corte} de $F_{1_\alpha}$ como el subconjunto de $\Omega$ tal que la función de pertenencia es mayor o igual que un valor dado, $\alpha$. Esto es, $F_{1_\alpha} = \{ x \in \Omega \enspace | \enspace f_1(x) \geq \alpha \}$.
    \item Se define el \textbf{núcleo} de $F_1$ como el subconjunto de $\Omega$ tal que el grado de pertenencia es 1. Esto es, $ker(F_1) = \{ x\in \Omega \enspace | \enspace f_1(x) = 1 \}$.
    \item Se define la \textbf{altura} de $F_1$ como el supremo de todos los grados de pertenencia. Esto es, $hgt(F_1) = \sup_{x\in \Omega} f_1(x)$.
    \item Se dice que un conjunto $F_1$ difuso está \textbf{normalizado} si, y solo si, la altura es igual a 1. Esto es, $\exists x \in \Omega$ tal que $f_1(x) = 1$.
\end{itemize}

\subsection{Números difusos}\label{fuzzynumbers}

Los números difusos fueron introducidos por Zadeh \cite{fuzzynumberszadeh} para poder trabajar con información imprecisa de forma práctica, posteriormente, otros trabajos han ido refinando y redefiniendo este concepto. En esta sección, vamos a definirlos y vamos a ver cómo trabajar con ellos. Es una sección muy importante para nuestro trabajo, ya que será con números difusos como representaremos la información en la base de datos. 

\begin{definition}[Número difuso]
Sea $F$ un conjunto difuso en $\Omega$ con $f$ su función de pertenencia. Diremos que $F$ es un \textbf{número difuso} si cumple:

\begin{enumerate}
    \item f es convexa
    \item f es semi-continua superiormente
    \item El soporte de $F$ está acotado 
\end{enumerate}
\end{definition}

Veamos ahora la forma general de un número difuso. Sean $\alpha, \beta, \gamma, \delta \in \Omega$ con $\alpha \leq \beta \leq \gamma \leq \delta$, entonces la \textbf{forma de pertenencia general de un número difuso} es:

\begin{equation*}
    f(x) = \left\{ { \begin{array}{cc}
                    0 & x < \alpha \\ 
                    r(x) & x\in [\alpha,\beta) \\
                    h & x\in [\beta,\gamma] \\
                    s(x) & x\in (\gamma,\delta] \\
                    0 & x > \delta \\ 
                    \end{array}  } \right.
\end{equation*}

donde $r: \Omega \longrightarrow [0,1]$ es no decreciente, $s: \Omega \longrightarrow [0,1]$ es no creciente y $h \in (0,1]$ con $r(\beta) = h = s(\gamma)$.

Los números difusos que nos interesan a nosotros especialmente, y que utilizaremos para dar nuestra solución son aquellos en los que $r,s$ son lineales. Estos son llamados los \textbf{trapezoides} y tienen la siguiente forma de pertenencia de número difuso.

\begin{equation}\label{trapezoidrepresentation}
    f(x) = \left\{ { \begin{array}{cc}
                    0 & x < \alpha \\ 
                    h + \frac{(x-\beta)h}{\beta-\alpha} & x\in [\alpha,\beta) \\
                    h & x\in [\beta,\gamma] \\
                    h + \frac{(x-\gamma)h}{\delta-\gamma} & x\in (\gamma,\delta] \\
                    0 & x > \delta \\ 
                    \end{array}  } \right.
\end{equation}

Además, habitualmente estará normalizado ($h=1$) y por tanto nos basta con una tupla $(\alpha, \beta, \gamma, \delta)$ para definir un número difuso.

\begin{remark}\label{notaciontrapezoide}
Notemos que esta representación nos vale incluso cuando se tengan solo 1,2 o 3 valores. Estos serán llamados \textit{crisp}, \textit{intervalo} y \textit{triangulares} respectivamente. Con un valor $a$, tendremos $\alpha = \beta = \gamma = \delta = a$, con dos valores, $a,b$ tendremos $\alpha = \beta = a$ y $\gamma = \delta = b$ y con tres valores, $a,b,c$ tendremos $\beta = \gamma = b$.
\end{remark}


\chapter{Propuesta difusa en MongoDB}
\label{propuesta}
Una vez introducida la teoría de conjuntos difusos en el capítulo previo y de dar una introducción a las bases de datos no relaciones, vamos a tratar de dar una solución para la base de datos NoSQL, MongoDB, dotándola con algunas utilidades para poder trabajar con estos conjuntos de datos.

En la literatura podemos encontrar trabajos con propuestas para bases de datos NoSQL, véase \cite{fuzzyquerygraph, fuzzyquerygraph2, fuzzyquerygraph3}, donde se pueden encontrar propuestas para bases de datos basadas en grafos o \cite{fuzzyqueryhbase} donde se habla de modelado de conjuntos difusos en la base de datos HBase. En \cite{fuzzyquerymongo} podemos encontrar una propuesta para realizar consultas difusas en la base de datos MongoDB, el cuál hemos utilizado para sacar algunas ideas sobre las que se basan nuestra propuesta.

El objetivo de la propuesta es mantener la compatibilidad con el funcionamiento de MongoDB común, adaptar el lenguaje al tipo de consultas que se realiza de forma nativa y adaptar la representación de los datos difusos a los tipos de datos nativos utilizados por MongoDB.

\section{Fuzzy Find}

En \cite{tesismedina} se expuso un módulo para permitir extender la capacidad de un SGBDR clásico para que pueda representar y manipular información imprecisa. En este trabajo se ha realizado una prueba de concepto sobre MongoDB con la implementación de la función \textbf{fuzzy\_find}, que nos permitirá realizar consultas sobre una abase de datos que contenga datos de tipo difuso.

Para llevar a cabo esto, hemos utilizado la utilidad que nos provee MongoDB para \href{https://docs.mongodb.com/manual/tutorial/store-javascript-function-on-server/}{almacenar funciones javascript en el servidor de MongoDB}, de forma que se permite su uso en cualquier contexto javascript.

Comenzamos una prueba de concepto con el operador de agregación \texttt{map-reduce} descrito en \ref{mapreduce}, pero tras unas pruebas con una cantidad de datos relativamente grande, no obteníamos un el rendimiento que esperábamos, las consultas eran demasiado lentas y descartamos esta opción a favor de utilizar el operador de agregación \texttt{pipeline}, véase la sección \ref{pipeline}. Este operador nos permite utilizar índices y nos ofrece para este problema en concreto un rendimiento muy superior a la opción \textit{map-reduce}.

\subsection{Sintaxis}

La cabecera de la función \texttt{fuzzy\_find} es la siguiente:

\begin{lstlisting}[numbers=none]
fuzzy_find(collection, filter, projection, count_name=null)
\end{lstlisting}

donde los parámetros son:

\begin{itemize}
    \item collection: Nombre de la colección sobre la que se quiere ejecutar la consulta.
    \item filter: JSON con la query que se quiere realizar. Conserva el formato propuesto por MongoDB además de los operadores que introducimos en este trabajo, explicaremos estos enla sección posterior.
    \item projection: JSON para la etapa de proyección.
    \item count\_name: Campo opcional. Si se especifica una cadena de texto se añade la etapa de \textit{count} para devolver el número de documentos en lugar de los propios documentos.
\end{itemize}

Veamos la sintaxis propuesta mediante un ejemplo:

\begin{example}

Supongamos una base de datos \texttt{test}, con la colección \texttt{example} y la siguiente información:

\begin{lstlisting}
{ _id: 1, item: { name: "ab", code: "123" }, qty: 15, tags: [ "A", "B", "C" ] }
{ _id: 2, item: { name: "cd", code: "123" }, qty: 20, tags: [ "B" ] }
{ _id: 3, item: { name: "ij", code: "456" }, qty: 25, tags: [ "A", "B" ] }
{ _id: 4, item: { name: "xy", code: "456" }, qty: 30, tags: [ "B", "A" ] }
{ _id: 5, item: { name: "mn", code: "000" }, qty: 20, tags: [ [ "A", "B" ], "C" ] }
\end{lstlisting}

podemos ejecutar la consulta utilizando la función \texttt{fuzzy\_find} como sigue:

\begin{lstlisting}[numbers=none]
fuzzy_find('example', { qty: { \$eq: 20 } }, {})
\end{lstlisting}

y obtenemos:

\begin{lstlisting}
{ _id: 2, item: { name: "cd", code: "123" }, qty: 20, tags: [ "B" ] }
{ _id: 5, item: { name: "mn", code: "000" }, qty: 20, tags: [ [ "A", "B" ], "C" ] }
\end{lstlisting}

si añadimos una proyección a la función:

\begin{lstlisting}[numbers=none]
fuzzy_find('example', { qty: { \$eq: 20 } }, {tags: 1})
\end{lstlisting}

obtendríamos:

\begin{lstlisting}
{ "_id" : 2, "tags" : [ "B" ] }
{ "_id" : 5, "tags" : [ [ "A", "B" ], "C" ] }
\end{lstlisting}

\end{example}

\subsection{Representación de información}

Dependiendo del tipo de dato, acordamos su representación como sigue:

\begin{itemize}
    \item Datos precisos: Hacen referencia a los datos comunes de MongoDB, enteros, fechas, cade de texto, arrays, documentos... Estos datos se seguiran representando con el tipo de dato correspondiente y se trabajará con ellos de forma nativa.
    \item Datos difusos: Como ya adelantamos en los capítulos previos, la forma de trabajar con datos difusos que hemos utilizado es mediante los números difusos \ref{fuzzynumbers} de tipo \textbf{trapezoides} \ref{trapezoidrepresentation}. Para representar esto en MongoDB hemos utilizado \texttt{arrays} de 4 posiciones, por tanto, la etapa previa a cualquier operación con un operador difuso será transformar el valor que nos den a un trapezoide siguiendo lo descrito en \ref{notaciontrapezoide}.
\end{itemize}



\chapter{Conclusiones y vías futuras}
\section{Conclusiones y vías futuras}

En conclusión, hemos visto cómo se puede extender la funcionalidad de MongoDB para trabajar con conjuntos difusos, que era el objetivo propuesto para este trabajo. Se ha dado una solución con la implementación de la función \texttt{fuzzy\_find} que permite realizar consultas con cláusulas difusas y proyectar el grado de pertenencia a la consulta pedida, pero aún con ciertas limitaciones.

Esta versión de \texttt{fuzzy\_find} solo acepta una cláusula difusa por atributo, además no se pueden combinar cláusulas difusas para el mismo atributo.

Las vías futuras para extender esta funcionalidad pueden resumirse:

\begin{itemize}
    \item Mejorar restricciones: Hay que ampliar la funcionalidad de la función \texttt{fuzzy\_find} para permitir varias cláusulas difusas para un mismo atributo. Además, si se consigue esto, hay que modificar la proyección para poder seleccionar cuales de las condiciones difusas se quiere proyectar.
    \item Implementación de operadores lógicos: Hay que implementar los operadores lógicos para utilizar con cláusulas difusas.
    \item Mejorar indexación: MongoDB permite la indexación en campos de tipo array, un posible estudio de la indexación podría mejorar los tiempos de consulta.
    \item Extender funcionalidad en clientes MongoDB: Ahora mismo, la función \texttt{fuzzy\_find} está probada desde la shell de MongoDB. Un posible trabajo futuro sería extender esta funcionalidad a las librerías para los distintos lenguajes de programación.
    \item Aumentar funcionalidad MongoDB: En esta propuesta nos hemos centrado en el operador de agregación \textit{pipeline}. Se podría estudiar para implementar el trabajo con información imprecisa con el resto de operadores.
\end{itemize}

% ********************************************************************
% Backmatter
%*******************************************************
%\renewcommand{\thechapter}{\alph{chapter}}
\appendix
\part*{Anexo}
\chapter{Implementación Fuzzy Find}
\input{chapters/Code.tex}

%********************************************************************
% Other Stuff in the Back
%*******************************************************
\cleardoublepage@book{montielrosbook,
  title={Curves and Surfaces},
  author={Sebastián Montiel and Antonio Ros},
  year={1998},
  publisher={American Mathematical Society}
}

@article{apuntesanalisis,
  title={Apuntes Análisis Matemático I},
  author={María D. Acosta, Camilo Aparicio, Antonio Moreno and Armando R. Villena},
  url={https://www.ugr.es/~dpto_am/docencia/Apuntes/Analisis_matematico_I_Matematicas.pdf},
}

@article{apuntesjoaquin,
  title={Curvas y Superficies},
  author={Joaquín Perez Muñoz},
  url={http://wpd.ugr.es/~jperez/wordpress/wp-content/uploads/raizCyS.pdf},
}

@article{paperchicago,
  title={THE JORDAN-BROUWER SEPARATION THEOREM},
  author={Wolfgang Schmaltz},
  url={http://www.math.uchicago.edu/~may/VIGRE/VIGRE2009/REUPapers/Schmaltz.pdf},
}

\cleardoublepage%*******************************************************
% Declaration
%*******************************************************
\refstepcounter{dummy}
\pdfbookmark[0]{Declaration}{declaration}
\chapter*{Declaración}
\thispagestyle{empty}

% TODO - DNI
Yo, \textbf{\myName}, alumno de la titulación \myDegree de la \textbf{\myFaculty} y de la \textbf{\myOtherFaculty} de la \textbf{\myUni}, con DNI XXXXXXXXX, asumo la originalidad de este trabajo, entendida en el sentido de que no se han utilizado fuentes sin citarlas debídamente.

Y para que conste, expido y firmo la presente declaración.

\bigskip
 
\noindent\textit{\myLocation, \today} 

\vspace{3cm}

\begin{flushright}
    \begin{tabular}{m{5cm}}
        %\\ \hline
        \centering\myName \\
    \end{tabular}
\end{flushright}
\cleardoublepage\include{FrontBackmatter/Colophon}
% ********************************************************************
% Game Over: Restore, Restart, or Quit?
%*******************************************************
\end{document}
% ********************************************************************