\documentclass[paper=a4, fontsize=11pt, spanish]{scrartcl}
\usepackage{fourier}
\usepackage{hyperref}
\usepackage{amsthm}
\usepackage{amsmath}
\usepackage{sectsty}
\allsectionsfont{\normalfont\scshape}
\setlength\parindent{0pt}

\usepackage{fancyhdr}
\pagestyle{fancyplain}
\fancyhead{}
\fancyfoot[L]{}
\fancyfoot[C]{}
\fancyfoot[R]{\thepage}
\renewcommand{\headrulewidth}{0pt}
\renewcommand{\footrulewidth}{0pt}
\setlength{\headheight}{13.6pt}

\usepackage[spanish,es-noquoting,es-lcroman]{babel}
\usepackage[utf8]{inputenc}
\usepackage[T1]{fontenc}
\usepackage{graphicx}
\usepackage{listings}
\selectlanguage{spanish}

%Includes "References" in the table of contents
\usepackage[nottoc]{tocbibind}

%----------------------------------------------------------------------------------------
%   ALIAS
%----------------------------------------------------------------------------------------
\newcommand{\rmath}{\mathbb{R}}
\newcommand{\rdos}{\mathbb{R}^2}
\newcommand{\rtres}{\mathbb{R}^3}
\newcommand{\rdostortres}{\rdos \longrightarrow \rtres}
\newcommand{\rtrestordos}{\rtres \longrightarrow \rdos}
\newcommand{\umath}{\mathds{U}}
\newcommand{\vmath}{\mathds{V}}
\newcommand{\tmath}{\mathds{T}}
\newcommand{\rtrestou}{\rtres \longrightarrow \umath}
\newcommand{\utortres}{\umath \longrightarrow \rtres}
\newcommand{\rtrestov}{\rtres \longrightarrow \vmath}
\newcommand{\vtortres}{\vmath \longrightarrow \rtres}
\newcommand{\utov}{\umath \longrightarrow \vmath}
\newcommand{\xmath}{\mathds{X}}
\newcommand{\cinf}{\mathds{C}^\infty}
\newcommand{\dx}{(d\mathds{X})}
\newcommand{\dxq}{(d\mathds{X})_{q}}
\newcommand{\cn}{\mathbb{C}^n}
\newcommand{\nmath}{\mathds{N}}
\newcommand{\unitsphere}{\mathbb{S}^2(1)}

%----------------------------------------------------------------------------------------
%   Formats
%----------------------------------------------------------------------------------------
\newtheorem{theorem}{Teorema}[section]
\newtheorem{corollary}{Corolario}[theorem]
\theoremstyle{definition}
\newtheorem{lemma}[theorem]{Lema}
\theoremstyle{definition}
\newtheorem{definition}{Definición}
\theoremstyle{definition}
\newtheorem{proposition}{proposition}

%----------------------------------------------------------------------------------------
%	TÍTULO
%----------------------------------------------------------------------------------------
% Título con las líneas horizontales, nombres y fecha.

\newcommand{\horrule}[1]{\rule{\linewidth}{#1}}

\title{
  \normalfont \normalsize
  \textsc{Universidad de Granada.} \\ [25pt]
  \horrule{0.5pt} \\[0.4cm]
  \huge Desigualdad Isoperimétrica en el espacio \\
  \horrule{2pt} \\[0.5cm]
}

\author{Sergio Padilla López}

\date{\normalsize\today}

%----------------------------------------------------------------------------------------
%	DOCUMENTO
%----------------------------------------------------------------------------------------

\begin{document}
\maketitle
\newpage

\tableofcontents

\newpage

\section{Preliminares}

\subsection{Superficie regular}

\begin{definition}
Sea un subconjunto $S \subseteq \rtres$, $S \neq \emptyset$, es una \textit{superficie regular} si:

$\forall p \in S$ $\exists V$ entorno abierto de $p$ en $S$ (con la topología inducida de $\rtres$) y una aplicación $\xmath: \utortres$ con $\umath \subseteq \rdos$ abierto, verificando:

\begin{enumerate}
    \item $\xmath \in \cinf(\umath, \rtres)$.
    \item $\dxq: \rdostortres$ es inyectiva $\forall q \in \umath$.
    \item $\xmath(\umath) = \vmath$ y $\xmath:\utov$ es un homeomorfismo.
\end{enumerate}
\end{definition}

\subsection{Curvatura de Superficies. Superficie orientable. Aplicación de Gauss}

Sea $S \subseteq \rtres$ una superficie. Un \textit{campo de vectores} (diferenciable) en $S$ es una aplicación diferenciable $\vmath: S \longrightarrow \rtres$. Si $\vmath_p := \vmath(p) \in \tmath_p S$ para todo $p \in S$, diremos que $\vmath$ es un \textit{campo tangente} a $S$. Si $\vmath_p \bot \tmath_p S$ para todo $p \in S$, diremos que $\vmath$ es un \textit{campo normal} a $S$. Un campo unitario es aquel que cumple $\parallel \vmath_p \parallel = 1$ para todo $p \in S$.

Sea $S \subseteq \rtres$ una superficio orientable, y $p \in S$. Se definen:

Se dice que $S$ es una \textbf{superficie orientable} si admite un campo normal unitario global $\nmath: S \longrightarrow \unitsphere$. A este campo $\nmath$ se le llama \texbf{aplicación de Gauss}.

Se llama \textbf{endomorfismo de Weingarten} de $S$ en $p$ al endomorfismo: $A_p = -(dN)_p$

\begin{definition}[Formas fundamentales]
La \texbf{primera forma fundamental} de $S$ en $p$ es: $I_p: T_pS \times T_pS \longrightarrow \rmath$ donde $I_p(u,v) = <u,v>$, $\forall u,v \in T_pS$.

La \texbf{segunda forma fundamental} de $S$ en $p$ es la forma bilineal: $II_p = \sigma_p: T_pS \times T_pS \longrightarrow \rmath$ donde $\sigma_p(u,v) = <A_pu,v> = I_p(Ap_u,v) = -<(dN)_p(u), v>$, $\forall u,v \in T_pS$.
\end{definition}

Notemos que $\sigma_p$ es simétrica y por tanto el endomorfismo de Weingarten ($A_p$) es autoadjunto.

Cómo $A_p$ es autoadjunto, entonces es diagonalizable mediante una base ortonormal y por tanto tiene valores propios reales, es decir, $\exists k_1(p), k_2(p) \in \rmath, k_1(p) \leq k_2(p)$, $\exists {e_1,e_2}$ base ortonormal en $(T_pS, I_p)$ de forma que $A_p(e_i) = k_i(p)e_i$, $\forall i = 1,2$

\begin{definition}[Curvaturas principales]
Los números $k_i(p)$ se llaman \textit{curvaturas principales} de $S$ en $p$.
\end{definition}
Los vectores propios, no nulos, de $A_p$ se llaman \textbf{direcciones principales} de $S$ en $p$.
Se define la \textbf{curvatura de Gauss} de $S$ en $p$ como el número real $K(p)=det A_p=k_1(p)k_2(p)$
Se define la \textbf{curvatura media} de $S$ en $p$ como el número real $H(p)=\frac{1}{2}tr A_p=\frac{k_1(p)+k_2(p)}{2}$
Se dice que $S$ es \textbf{una superficie llana} si $K(p)=0$, $\forall p \in S$

\begin{definition}[Superficie minimal y totalmente umbilical]
Se dice que $S$ es \textit{minimal} si $H(p)=0$, $\forall p \in S$

Un punto es \textit{umbilical} si $k_1(p)=k_2(p)$.

Sea $S \subseteq \rtres$, se dice que $S$ es \textit{totalmente umbilical} si todos son puntos son umbilicales.
\end{definition}

\begin{theorem}[Clasificación de las superficies totalmente umbilicales]
Sea $S \subseteq \rtres$ superficie conexa, cerrada y totalmente umbilical. Entonces es un plano o una esfera.
\end{theorem}

La demostración de este teorema se basa en la que la conexión nos impide que salgan uniones de varios planos y esferas y el cierre impide que S sea un abierto de plano o esfera. No lo vamos a demostrar.

\begin{theorem}[Teorema de Hilbert-Liebmann]
Sea $S \subseteq \rtres$ una superficie compacta y conexa con curvatura de Gauss $K$ constante, entonces S es una esfera.
\end{theorem}


\begin{theorem}[Teorema de Alexandrov]
Sea $S \subseteq \rtres$ una superficie compacta y conexa con curvatura media $H$ constante, entonces S es una esfera.
\end{theorem}

Generalización del teorema de Hilbert-Liebmann a superficies cerradas.
\begin{theorem}[Teorema de Bonnet]
Sea $S \subseteq \rtres$ una superficie cerrada y conexa con curvatura de Gauss $K=c>0$, constante positiva, entonces S es una esfera.
\end{theorem}


\section{Teorema de Brower-Samelson}

En esta sección vamos a dar los preliminares necesarios para la demostración del Teorema de Brower-Samelson. Comenzaremos con el teore de Jordan-Brower, que nos permitirá hablar del volumen encerrado por una superficie compacta. Finalmente, entraremos en la definición y propiedades de los entornos tubulares.


\subsection{Teorema de Brower-Samelson}
Cómo ya sabemos por el Teorema de Jordan clásico en el caso de $\rdos$, una curva cerrada y simple, delimita el plano en dos superficies conexas, una de ellas acotada. Este teorema, nos va a permitir tener tener una extensión de este resultado para el caso de $\rtres$. De hecho, esta generalización, aunque no vamos a entrar a verla, es válida para todo $\rmath^n$ con un hiperplano suyo, puede verse en \cite{paperchicago}. Para este caso, veámoslo en n=3.

\begin{theorem}[Teorema de separación de Jordan-Brower]
Sea $S \subseteq \rtres$ superficie compacta y conexa. Entonces $\rtres - S$ tiene exactamente dos componentes conexas cuya frontera común es $S$.
\end{theorem}
\begin{proof}
Veamos, que si $S$ es compacta, luego cerrada, y conexa, entonces $\rtres - S$ tiene exactamente dos componentes conexas cuya frontera es común.

Sea $C$ una componente conexa de $\rtres - S$. Partiendo de la suposición que $S$ es cerrada, entonces $\rtres - S$ es localmente conexo, luego cada componente conexa $C$ es abierta. Además, la frontera de $C$ es distinta del vacío ($Fr(C) \neq \emptyset$). Veámoslo.

Supongamos $Fr(C) = \emptyset$, entonces $\bar{C} = int(C)\bigcup Fr(C)=int(C)=C$ luego, $C$ sería abierto y cerrado en $\rtres$, por tanto, $C=\rtres$. Lo que es una contradicción ya que definimos $C$ como una componente conexa de $\rtres - S$.

Notemos, $\rtres - S = C \bigcup C'$, donde $C'$ es la unión de todas las componentes conexas distintas de $C$. Entonces, $C'$ es un conjunto abierto de $\rtres$, $\rtres - C' = C \bigcup S$ es cerrado y por tanto $\bar{C} \subset C\bigcup S$. Entonces, $Fr(C)=\bar{C}-C \subseteq S$.


Veamos ahora que $Fr(C) = S$. Para ello, sabemos que $Fr(C)$ es un subconjunto cerrado no vacio de $S$, veamos que también es abierto en S.

Sea $p \in Fr(C) \subseteq S$, sea $W$ un entorno abierto de $p$ en $\rtres$, de forma que $W-S$ tiene exactamente dos componentes conexas, $C_1, C_2$ cuya frontera común es $S \bigcap W$. Sabemos que $W-S \subseteq \rtres - S$ luego, $C_1, C_2$ están contenidas en componentes conexas distintas de $\rtres - S$. Esto es, o $C_1, C_2$ están contenidas en $C$, o no lo cortan. Si ninguna componente conexa corta a $C$, tenemos que $W\bigcap C = \emptyset$ y por tanto, $p \notin \bar{C}$, luego $p \notin Fr(C)$ y contradice nuestra suposición. Así, tenemos que al menos una de las componentes $C_1, C_2$ están contenidas en $C$. Supongamos que $C_1 \subseteq C$, análogo para $C_2$. Tenemos: $W \bigcap S = Fr(C_1) \bigcap W \subseteq \bar{C_1} \subseteq \bar{C}$, luego $W \bigcap S= Fr(C)$. Con esto tenemos que $p$ es un punto interior de la frontera de $C$ y por tanto, la frontera de $C$ es un abierto en $S$, como queríamos probar.

Además hemos probado, que $C$ contiene una de las dos componentes conexas de $W-S$, y como estas componentes son disjuntas, no pueden existir más de dos, por tanto, queda probado que $C'$ es la otra componente conexa de $\rtres - S$.

\end{proof}

\begin{definition}[Dominios interior y exterior]
Llamamos \textit{dominio interior} y notamos como $\Omega$ a la componente conexa acotada limitada por la superficie $S$. Llamamos \textit{dominio exterior} a la componente no acotada $\rtres - \Omega$.
\end{definition}

\begin{theorem}[Teorema de Brower-Samelson]
Toda superficie compacta en un espacio euclideo es orientable.
\end{theorem}
\begin{proof}

\end{proof}

\subsection{Entornos tubulares}

En este sección, vamos a definir los entornos tubulares. Veremos que dada una superficie, bajo determinadas hipótesis, existe un entorno que envuelve la superficie.

Sea $S$ una superficie de $\rtres$, como $\rtres$ es un espacio métrico, los entornos más sencillos de definir son los conocidos como \textbf{entornos métricos}. Definidos como los puntos cuya distancia a la superficie es menor que un delta dado, notamos:

\begin{equation*}
    B_\delta(S)=\{p\in \rtres | dist(p,S) < \delta\}
\end{equation*}

donde, $dist(p,S) = inf_{q\in S}|p-q|$.

\begin{lemma}
Sea $S$ una superficie cerrada de $\rtres$, el conjunto $B_\delta(S)$ definido anteriormente, coincide con el conjunto $N_\delta(S)=\bigcup_{p\in S}N_\delta(p)$, definido como el los segmentos abiertos en las normales a la superficie S con centro p ($p \in S$) y radio $\delta$.
\end{lemma}
\begin{proof}
Veámoslo por doble inclusión:

Sea $p \in S$, y sea $q \in N_\delta(p)$ para el $p$ dado y $\delta > 0$. Es directo que, $dist(q,S) \leq |q-p| < \delta$, luego $q \in B_\delta(S)$.

Supongamos ahora $q \in B_\delta(S)$. Tomamos la función distancia del punto $q$ a $S$. Como $S$ es cerrado, sabemos que existe un mínimo en un punto $p \in S$, además por la caracterización de los punto críticos de la función distancia al cuadrado, sabemso que el punto $q$ está en la normal a $S$ en $p$. Así, $|p-q| = dist(q,S) < \delta$, luego $q \in N_\delta(p)$.
\end{proof}

Sea $S$ una superficie orientable con la aplicación de Gauss $N: S \longrightarrow \mathbb{S}^2 \subset \rtres$, definimos:

\begin{align*}
    F: S x \mathbb{R} &\longrightarrow \rtres \\
    (p,t) &\longrightarrow p + tN(p)
\end{align*}

Esta aplicación,claramente diferenciable, envía cada punto $p$ de la superficie a la distancia $t$ en dirección de la recta normal a la superficie en el punto $p$. Luego tenemos:

\begin{equation*}
    F(S \times (-\delta, \delta)) = N_\delta(S)=\bigcup_{p\in S} N_\delta(p), \qquad \forall \delta > 0
\end{equation*}

\begin{definition}[Entornos tubulares]
La unión $N_\delta(S)$ de todos los segmentos normales de radio $\delta > 0$ centrados en los puntos de una superficie $S$ orientable es llamada \textit{entorno tubular} de radio $\delta$ si es un abierto como subconjunto de $\rtres$ y la función $F: S x (-\delta, \delta) \longrightarrow N(p)$ definida previamente es un difeomorfismo.
\end{definition}

% TODO: Demostrar
\begin{lemma}
Sea $S$ una superficie, $\forall p \in S$, $\exists V_p$, entorno abierto y orientable y $\delta > 0$ de forma que el conjunto $N_\delta(V_p)$ es un entorno tubular de $V$.
\end{lemma}

Veamos ahora la existencia de entornos tubulares para superficies compactas.
\begin{theorem}[Existencia de entornos tubulares]
Sea $S$ una superficie orientable y $R \subset S$ un subconjunto abierto relativamente compacto. Entonces $\exists \epsilon > 0$ tal que, el conjunto $N_\epsilon(R)$ es un entorno tubular de la superficie $R$, esto es, es un abierto de $\rtres$ y la función:

\begin{align*}
    F: S \times (-\epsilon, \epsilon) &\longrightarrow N_\epsilon(R) \\
    (p,t) &\longrightarrow p + tN(p)
\end{align*}

es un difeomorfismo.

En particular, cuando la superficie $S$ es compacta, $\exists \epsilon > 0$ tal que
$B_\epsilon(S)=N_\epsilon(S)$ es un entorno tubular de $S$.
\end{theorem}
\begin{proof}
Como tenemos que $R$ es relativamente compacto, esto es, $\bar{R}$ es compacto, existe un recubrimiento finito por abierto de $\bar{R}$ cada uno con un entorno tubular, por el lema previo. Sea $\delta > 0$ el menor radio de todos ellos. Si tomamos la función $F$ definida previamente, restringida al intervalo de definición $Rx(-\delta, \delta)$ es un difeomorfismo local. Vamos a buscar un $\epsilon \in (0,\delta)$ tal que la función $F$ restringida al intervalo $Rx(-\epsilon, \epsilon)$ es inyectiva. Veámoslo por reducción al absurdo.

Supongamos que $\exists \epsilon \in (0,\delta)$ tal que $F$ restringida al intervado de definición $Rx(-\epsilon, \epsilon)$ no es inyectiva, o lo que es lo mismo, los segmentos $N_\epsilon(p)$ de normales con $p \in R$ intersecan entre ellos. Tomamos $\epsilon=\frac{1}{n}$ y $\forall p \in \mathbb{N}$, $\exists p_n,q_n \in S$ con $p_n \neq q_n$ tal que $N_{\frac{1}{n}}(p_n) \bigcap N_{\frac{1}{n}}(q_n) \neq \emptyset$.
Como $R$ es relativamente compacto, podemos tomar sucesiones parciales que convergen en $\bar{R}$. Sean $\{p_n\}_{n \in \mathbb{N}}, \{q_n\}_{n \in \mathbb{N}}$ estas sucesiones y sean $p,q \in \bar{R}$ los puntos donde convergen, esto es:

\begin{equation*}
    \lim_{n\to\infty} p_n = p \qquad \lim_{n\to\infty} q_n = q
\end{equation*}

Sea $r_n \in N_{\frac{1}{n}}(p_n) \bigcap N_{\frac{1}{n}}(q_n)$, entonces:

\begin{equation*}
    |p_n - q_n| = |p_n - q_n + r_n - r_n| \leq |p_n-r_n| + |r_n-q_n| < \frac{1}{n}+\frac{1}{n} = \frac{2}{n}
\end{equation*}

Por tanto los límites coinciden.

Aplicando el lema previo al punto $p = q \in S$, tenemos $V$ y $\rho > 0$ tal que $N_\rho(V)$ es un entorno tubular. Además, $\exists N_0 \in \mathbb{N}$ tal que, $\forall n > N_0$, $p_n,q_n \in V$ y $1/n < \rho$. Por tanto llegamos a una contradicción:

\begin{equation*}
    N_{\frac{1}{n}}(p_n) \bigcap N_{\frac{1}{n}}(q_n) \subset N_\rho(p_n)\bigcap N_\rho(q_n) = \emptyset
\end{equation*}

ya que $N_\rho(V)$ es un entorno tubular. Por tanto, $F$ restringida al intervalo de definición $V \times (-\rho, \rho)$ es inyectiva y localmente difeomorfismo.

\end{proof}

A partir de los entornos tubulares, introduciremos el concepto de superficie paralela.

\begin{definition}[Superficie paralela]
Sea $\rho > 0$ y sea $N_\rho(S)$ un entorno tubular de la superficie $S$ compacta con $N$ su aplicación de Gauss. $\forall t \in (-\rho, \rho)$ se define $S_t={p + tN(p); p \in S}$ superficie compacta y la apliación $F_t: S \longrightarrow S_t$ dada por $F_t(p)=p+tN(p)$ es un difeomorfismo.
Llamamos a $S_t$ \textit{superficie paralela} a S a una distancia $t$.
\end{definition}


\section{Integración en Superficies}

\begin{definition}(Valor absoluto del Jacobiano)
Sea $\Phi: S \longrightarrow S'$ un difeomorfismo entre dos superficies. Se define el \textit{valor absoluto del jacobiano de $\Phi$} como la aplicación
\begin{align*}
    |Jac \Phi|: S &\longrightarrow \mathbb{R} \\
    p &\longrightarrow |det(d\Phi)_p| = |det(M_p)|
\end{align*}

donde $M_p=M((d\Phi)_p, (B_1)_p, (B_2)_{\Phi(p)})$, siendo $(B_1)_p$ base ortonormal de $T_pS$ y $(B_2)_{\Phi(p)}$ base ortonormal de $T_{\Phi(p)}S'$.
\end{definition}

\begin{definition}[Integral en $S \times \mathbb{R}$]
Sea $O$ un subconjunto abierto de $\rtres$, y sea el difeomorfismo $\phi: S \longrightarrow R \times (a,b)$ con $a < b$. Podemos decir que la función $h: R \times (a,b) \longrightarrow \mathbb{R}$ es integrable en su dominio cuando $(h \circ \phi)|Jac \phi|$ sea integrable en $O$ en el sentido de Lebesgue.

En ese caso, llamaremos \textit{integral de h en Rx(a,b)} al número real dado por:

\begin{equation*}
    \int_{R \times (a,b)} h = \int_{R \times (a,b)} h(p, t)dp dt = \int_{O} (h \circ \phi)|Jac \phi|
\end{equation*}
\end{definition}

\begin{definition}[Integral en $S$]
Sea $R$ una region de una superficie orientable $S$ y sea $f:R \longrightarrow \mathbb{R}$ una función. Diremos que $f$ es integrable en $R$ cuando la función $(p,t) \in Rx(0,1) \mapsto f(p)$ sea integrable en $R \times (0,1)$. Llamaremos \textit{integral de $f$ en $R$} al número real dado por:

\begin{equation*}
    \int_{R} f = \int_{R \times (0,1)} f(p)dp dt
\end{equation*}
\end{definition}

\begin{theorem}[Teorema del cambio de variable]
Sea $\phi:R_1 \longrightarrow R_2$ un difeomorfismo entre dos regiones de dos superficies orientables, y sea $f: R_2 \longrightarrow \mathbb{R}$ una función integrable. Entonces la función $(f \circ \phi)|Jac \phi|$ es integrable en $R_1$ y además:

\begin{equation*}
    \int_{R_2} f = \int_{R_1} (f \circ \phi)|Jac \phi|
\end{equation*}
\end{theorem}

\begin{theorem}[Teorema de Fubini]
Sea $R$ una región de una superficie orientable, $a,b \in \mathbb{r}, a < b$ y $h:R \times (a,b) \longrightarrow \mathbb{R}$ una función integrable en $R \times (a,b)$. Entonces, para casi todo $t \in (a,b)$, la función $p \in R \mapsto h(p,t)$ es integrable en $R$ y para casi todo $p \in R$ la función $t \in (a,b) \mapsto h(p,t)$ es integrable en $(a,b)$. Además, las funciones

\begin{equation*}
    p \in R \mapsto \int_a^b h(p,t), \quad  t \in (a,b) \mapsto \int_S h(p,t) dp
\end{equation*}

son integrales en $R$ y $(a,b)$ respectivamente.

Finalmente, tenemos que:

\begin{equation*}
    \int_{R \times (a,b)} h(p,t) dpdt = \int_R \left( \int_a^b h(p,t)dt \right) dp = \int_a^b \left( \int_R h(p,t)dp \right) dt
\end{equation*}
\end{theorem}


\subsection{Teorema de Divergencia}
En esta sección vamos a demostrar el Teorema de divergencia, para ello, vamos a utilizar la generalización del teorema del cambio de variable para las integrales de Lebesgue en el caso de que sea una función diferenciable cualquiera en lugar de un difeomorfismo. Esta generalización es la llamada fórmula del área.

\begin{theorem}[Fórmula del área]
Sea $O$ un abierto de $\rtres$ relativamente compacto, $\phi: O \longrightarrow \rtres$ una función diferenciable y $f: O \longrightarrow \mathbb{R}$ una función cualquiera. Supongamos que el producto $f|Jac(\Phi)|$ es integrable en $O$. Entonces tenemos que la función $n(\phi, f)$ definida como:

\begin{align*}
    n(\phi, f): \rtres - \phi(N) &\longrightarrow \mathbb{R} \\
    x &\longrightarrow \sum_{p \in \phi^{-1}(x)} f(p)
\end{align*}

donde $N \subset O$ es el subconjunto de punto críticos de $\phi$ en $O$ (esto es donde se anula el jacobiano), es integrable en $\rtres$ y

\begin{equation*}
    \int_{\rtres} n(\phi, f) = \int_O f(x)|Jac \phi|(x)dx
\end{equation*}
\end{theorem}

\begin{theorem}[Fórmula del área en productos]
Sea $\phi: S \times (a,b) \longrightarrow \rtres$ una apliación diferenciable, con $S$ una superficie compacta y $a,b \in \mathbb{R}$, $a < b$. Sea $f$ una función integrable en $S \times (a,b)$. Entonces, la función $n(\phi, f)$ dada por:

\begin{equation*}
    n(\phi, f) = \sum_{(p,t) \in \phi^{-1}(x)} f(p,t)
\end{equation*}

está bien definida para cada $x \in \rtres$ excepto en un conjunto de medida nula y:

\begin{equation*}
    \int_{\rtres} n(\phi, f) = \int_{S \times (a,b)} f(p,t)|Jac \phi|(p,t)dpdt
\end{equation*}
\end{theorem}

\begin{definition}[Campo vectorial diferenciable]
Sea $A$ un subconjunto de $\rtres$, se define un campo vectorial diferenciable como una aplicación diferenciable $X: A \longrightarrow \rtres$.
\end{definition}

\begin{definition}[Divergencia de un campo vectorial diferenciable]
Se define la divergencia de X como la función $div X: A \longrightarrow \mathbb{R}$ dada por $divX(p) = Traza(dX)_p$ con $\forall \in A$.
\end{definition}

\begin{theorem}[Teorema de divergencia]
Sea $S$ una superficie conexa y compacta y $\Omega$ el dominio interior determinado por $S$. Si $X: \bar{\Omega} \longrightarrow \rtres$ es un campo vectorial diferenciable, entonces:

\begin{equation*}
    \int_\Omega div X = -\int_S <X,N>
\end{equation*}

donde $N: S \longrightarrow \mathbb{S}^2$ es el campo normal interior a lo largo de $S$.
\end{theorem}

\begin{definition}[Volumen encerrado por una superficie compacta]
Si $S$ es una superficie compacta y $\Omega$ su dominio interior. Sea campo vectorial diferenciable identidad $X: \bar{\Omega} \longrightarrow \rtres$ definido por $X(x) = x \quad \forall p \in \bar{\Omega}$. Su divergencia es la función constante. Entonces por el teorema de divergencia:
\begin{equation*}
    vol \Omega = - \frac{1}{3} \int_S <p, N(p)> dp
\end{equation*}

donde $N$ es el interior normal unitario de $S$.
\end{definition}

\begin{theorem}[Teorema de divergencia en superficies]
Sea $S$ una superficie compacta y $V: S \longrightarrow \rtres$ un campo vectorial diferenciable en $S$. Entonces, tenemos:

\begin{enumerate}
    \item $\int_S div V = -2 \int_S<V,N>H$
    \item $\int_S [k_2(p)<(dV)_p(e_1),e_1> + k_1(p)<(dV)_p(e_2),e_2>]dp = -2 \int_S<V,N>K$ donde ${e_1,e_2}$ es una base ortonormal de direcciones principales en $p \in S$.
\end{enumerate}

donde $H, k_1 y k_2$ son las curvaturas media y principales de la superficie.
\end{theorem}

\section{Desigualdad Isoperimétrica}
Por último, en esta sección vamos a definir el resultado que íbamos buscando y  más importante, la desigualdad isoperimétrica en $\rtres$. Para su demostración, necesitamos de algunos elementos que aún no hemos definido. Vamos a verlos como preliminares a la desigualdad.

Vamos a ver las fórmulas de Minkowski que nos serán de utilidad en lo sucesivo.
\begin{theorem}[Fórmulas de Minkowski]
Sea $S$ una superficie compacta, $N$ el dominio interior dado por su aplicación de Gauss y sean $H$ y $K$ las curvaturas media y de Gauss de la superficie. Tenemos:

\begin{enumerate}
    \item \int_S (1+<p, N(p)>H(p))dp = 0
    \item \int_S (H(p)+<p, N(p)>K(p))dp = 0
\end{enumerate}
\end{theorem}




%----------------------------------------------------------------------------------------
%	BIBLIOGRAFIA
%----------------------------------------------------------------------------------------

\newpage

\begin{thebibliography}{9}
\bibitem{montielrosbook}
Sebastián Montiel and Antonio Ros.
\textit{Curves and Surfaces}.
España, 1998.

\bibitem{apuntesanalisis}
María D. Acosta, Camilo Aparicio, Antonio Moreno y Armando R. Villena
\\\texttt{https://www.ugr.es/$\sim$dpto\_am/docencia/Apuntes/Analisis\_matematico\_I\_Matematicas.pdf}

\bibitem{apuntesjoaquin}
Joaquín Perez Muñoz,
\\\texttt{http://wpd.ugr.es/$\sim$jperez/wordpress/wp-content/uploads/raizCyS.pdf}

\bibitem{paperchicago}
Wolfgang Schmaltz,
\\\texttt{http://www.math.uchicago.edu/$\sim$may/VIGRE/VIGRE2009/REUPapers/Schmaltz.pdf}
\end{thebibliography}

\end{document}
